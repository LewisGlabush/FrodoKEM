% IACR Communications in Cryptology template file
% This file shows how to use the iacrcc class to write a paper.

%% Document mode
% [version=xxx] where xxx is preprint, submission, or final. The default is preprint.
%               version=submission puts line numbers into the document and makes it
%               anonymous (which can be overridden).
%               version=final is for submitting the final version. This requires a license
%               to be specified and must be compiled with lualatex.
% [notanonymous]  Keep author names in submission mode
%% Package options
% [floatrow]      Load floatrow package with correct captions
% [biblatex]      Use the biblatex package instead of bibtex. Note that options
%                 may not be passed to biblatex at this time.

% Uncomment the next line if you have TeX Live 2021 or later
% and want to produce a PDF which complies with the PDF/A-2U standard
%\DocumentMetadata{pdfstandard=a-2u}
\documentclass[version=submission]{iacrcc}

% When the *final* document mode is used
% the authors need to provide a supported license.
% In all other modes this information is ignored.
% The currently provided ones are { CC-by }
\license{CC-by}

% Include LaTeX packages required by your paper

\newif\ifhidetodos
\hidetodosfalse
%\hidetodostrue
%\usepackage[letterpaper,margin=1in]{geometry}

\ifhidetodos
\newcommand{\NISTdescription}[1]{}
\newcommand{\assignto}[1]{}
\else
\newcommand{\NISTdescription}[1]{}
\newcommand{\assignto}[1]{{\Large\textbf{\textcolor{red}{[#1]}}}}
\fi

%%% FORMATTING %%%
\renewcommand{\topfraction}{0.99} % be more aggressive about text around floats
\renewcommand{\floatpagefraction}{0.99}
\usepackage{enumitem}
\setlist{noitemsep,topsep=0.03in} % reduce spacing in lists

%%% MISCELLANEOUS PACKAGES %%%
%\usepackage[utf8]{inputenc}

%%% DISPLAY PACKAGES %%%
\usepackage{booktabs}
\usepackage{graphicx}

\usepackage{cite}
\usepackage{microtype}
\usepackage{multicol}
\usepackage{multirow}
\usepackage[normalem]{ulem}
\usepackage{wrapfig}
\usepackage{xspace}
\usepackage{subfigure}
\usepackage{placeins}

%%% MATH PACKAGES %%%
%\usepackage{algorithm2e}
\usepackage{algorithm}
\usepackage[noend]{algorithmic}
\usepackage{amsfonts}
\usepackage{amsmath}
\usepackage{amssymb}
\usepackage{centernot}
\usepackage{nicefrac}
\usepackage{ulem}

%%% HYPERLINKS AND COLORS %%%
%\usepackage[table]{xcolor}
\definecolor{darkblue}{rgb}{0,0,0.5}
\definecolor{darkgreen}{rgb}{0,0.5,0}
\usepackage[colorlinks=true,linkcolor=darkblue,urlcolor=darkblue,citecolor=darkgreen]{hyperref}
\def\sectionautorefname{Section}
\def\subsectionautorefname{Section}
\def\subsubsectionautorefname{Section}
\def\algorithmautorefname{Algorithm}

%%% TODOS %%%
\ifhidetodos
\usepackage[disable]{todonotes}
\else
\usepackage{todonotes}
\fi

%%% FIXME notes %%%
% remove "draft" to turn off fixme notes
\ifhidetodos
\usepackage[final,multiuser,inline,nomargin]{fixme}
\else
\usepackage[draft,multiuser,inline,nomargin]{fixme}
\fi

% register commands for author(s)
\newcommand{\chris}[1]{\textcolor{red}{ \textbf{Chris:} #1 }}
\newcommand{\douglas}[1]{\textcolor{blue}{ \textbf{Douglas:} #1 }}
\newcommand{\patrick}[1]{\textcolor{orange}{ \textbf{Patrick:} #1 }}
\FXRegisterAuthor{jb}{ejb}{\color{red}Joppe}
\FXRegisterAuthor{cc}{ecc}{\color{green}Craig}
\FXRegisterAuthor{ld}{eld}{\color{purple}L{\'e}o}
\FXRegisterAuthor{pl}{epl}{\color{orange}Patrick}
\FXRegisterAuthor{im}{eim}{\color{red}Ilya}
\FXRegisterAuthor{mn}{emn}{\color{orange}Michael}
\FXRegisterAuthor{vn}{evn}{\color{green}Valeria}
\FXRegisterAuthor{cp}{ecp}{\color{red}Chris}
\FXRegisterAuthor{ar}{ear}{\color{darkgreen}Ananth}
\FXRegisterAuthor{ds}{eds}{\color{blue}Douglas}
\FXRegisterAuthor{lg}{elg}{\color{darkgreen}Lewis}
\fxusetheme{color}





%%%% From Lewis %%%%

\newcommand{\SFO}{\mathsf{SFO}}
\newcommand{\MINDCPA}{\ensuremath{\styleSecurityNotion{IND}_{n,u}\text{-}\styleSecurityNotion{CPA}}\xspace}
\newcommand{\MINDCCA}{\ensuremath{\styleSecurityNotion{IND}_{n,u}\text{-}\styleSecurityNotion{CCA}}\xspace}
\newcommand{\MOWPCA}{\ensuremath{\styleSecurityNotion{OW}_{n,u}\text{-}\styleSecurityNotion{PCA}}\xspace}
\newcommand{\SFOnotperp}{\styleScheme{SFO}^{\not\perp}}
\newcommand{\SFOnotperpprime}{\styleScheme{SFO}^{\not\perp\prime}}
\newcommand{\saltlist}{ \mathfrak{L}_{S}}
\newcommand{\M}{\mathcal{M}}
\newcommand{\lewis}[1]{\textcolor{blue}{Lewis: #1}}
\newcommand{\Lewis}[1]{\textcolor{blue}{Lewis: #1}}
\newcommand{\challoracle}{\mathsf{chall}}
\newcommand{\ST}{\styleScheme{ST}}
\newcommand{\kemnp}{\KEM^{\not\perp}}
\newcommand{\cipherlist}{ \mathfrak{L}_{C_j}}

%%% TIKZ %%%
\usepackage{tikz}
\usetikzlibrary{arrows,backgrounds,positioning,fit,shadows,shapes.geometric,decorations.pathreplacing}

%%% THEOREMS %%%
%\usepackage[amsmath,amsthm,thmmarks,hyperref]{ntheorem}
\usepackage[amsmath,amsthm,thmmarks,hyperref]{}
\usepackage{mathtools}
\usepackage{aliascnt}
%\newtheorem{theorem}{Theorem}[section]
% http://tex.stackexchange.com/questions/46258/how-to-get-correct-autoref-for-theorems
\def\NewTheorem#1#2{%
  \newaliascnt{#1}{theorem}
  \newtheorem{#1}[#1]{#2}
  \aliascntresetthe{#1}
  \expandafter\def\csname #1autorefname\endcsname{#2}
}
\theoremstyle{plain}            % following are theorem style
%\NewTheorem{lemma}{Lemma}
%\NewTheorem{corollary}{Corollary}
%\NewTheorem{proposition}{Proposition}
%\NewTheorem{claim}{Claim}
\NewTheorem{fact}{Fact}
\NewTheorem{maintheorem}{Main Theorem}
\NewTheorem{assumption}{Assumption}
\theoremstyle{definition}       % following are def style
%\NewTheorem{definition}{Definition}
%\NewTheorem{conjecture}{Conjecture}
\NewTheorem{construction}{Construction}
%\theoremstyle{remark}           % following are remark style
%\newtheorem*{remark}{Remark}    % I don't like numbered remarks.
%\NewTheorem{remark}{Remark}
%\NewTheorem{example}{Example}
%\NewTheorem{note}{Note}
%\NewTheorem{case}{Case}

%%% GENERAL TEXT MACROS %%%
\newcommand{\email}[1]{\href{mailto:#1}{\tt #1}}

%%% GENERAL MATH MACROS %%%
\newcommand{\getsr}{{\:\leftarrow\hspace*{-3pt}\raisebox{.5pt}{$\scriptscriptstyle\$$}\:}}
\newcommand{\tor}{{\:\raisebox{.5pt}{$\scriptscriptstyle\$$}\hspace*{-3pt}\rightarrow\:}}
\newcommand{\bit}{\{0,1\}}
\newcommand{\bits}{\bit^*}
\DeclareMathOperator{\dist}{dist}
\DeclareMathOperator*{\E}{\mathbb{E}}
\DeclareMathOperator*{\Var}{Var}
\DeclareMathOperator{\vol}{vol}

%%% "LEFT-RIGHT" PAIRS OF SYMBOLS %%%

%% NOTE: this requires \usepackage{mathtools} in the document preamble

% inner product
\DeclarePairedDelimiter\inner{\langle}{\rangle}
% absolute value
\DeclarePairedDelimiter\abs{\lvert}{\rvert}
% length
\DeclarePairedDelimiter\len{\lvert}{\rvert}
% norm
\DeclarePairedDelimiter\norm{\lVert}{\rVert}
% a set
\DeclarePairedDelimiter\set{\{}{\}}
% parens
\DeclarePairedDelimiter\parens{(}{)}
% tuple, alias for parens
\DeclarePairedDelimiter\tuple{(}{)}
% square brackets
\DeclarePairedDelimiter\bracks{[}{]}
% rounding off
\DeclarePairedDelimiter\round{\lfloor}{\rceil}
% floor function
\DeclarePairedDelimiter\floor{\lfloor}{\rfloor}
% ceiling function
\DeclarePairedDelimiter\ceil{\lceil}{\rceil}
% length of some vector, element
\DeclarePairedDelimiter\length{\lVert}{\rVert}
% "lifting" of a residue class
\DeclarePairedDelimiter\lift{\llbracket}{\rrbracket}

%%% ADVERSARIES %%%
\newcommand{\Adversary}{\mathcal{A}}
\newcommand{\Bdversary}{\mathcal{B}}

%%% EXPERIMENTS AND GAMES
\newcommand{\Adv}[2]{\mathrm{Adv}^{\styleSecurityNotion{#1}}_{#2}}
\newcommand{\Exp}[2]{\mathrm{Exp}^{\styleSecurityNotion{#1}}_{#2}}
\newcommand{\Advrm}{\mathrm{Adv}}
\newcommand{\qro}{q_{\mathsf{RO}}}

%%% SECURITY NOTIONS %%%
\newcommand{\styleSecurityNotion}[1]{\textsf{\upshape\mdseries #1}\xspace}
\newcommand{\eufcma}{\styleSecurityNotion{euf-cma}}
\newcommand{\EUFCMA}{\styleSecurityNotion{EUF-CMA}}
\newcommand{\indcca}{\styleSecurityNotion{ind-cca}}
\newcommand{\INDCCA}{\styleSecurityNotion{IND-CCA}}
\newcommand{\INDCCATwo}{\styleSecurityNotion{IND-CCA2}}
\newcommand{\indcpa}{\styleSecurityNotion{ind-cpa}}
\newcommand{\INDCPA}{\styleSecurityNotion{IND-CPA}}
\newcommand{\owcpa}{\styleSecurityNotion{ow-cpa}}
\newcommand{\OWCPA}{\styleSecurityNotion{OW-CPA}}
\newcommand{\intctxt}{\styleSecurityNotion{int-ctxt}}
\newcommand{\INTCTXT}{\styleSecurityNotion{INT-CTXT}}
\newcommand{\intptxt}{\styleSecurityNotion{int-ptxt}}
\newcommand{\INTPTXT}{\styleSecurityNotion{INT-PTXT}}
\newcommand{\sufcma}{\styleSecurityNotion{suf-cma}}
\newcommand{\SUFCMA}{\styleSecurityNotion{SUF-CMA}}
\newcommand{\ODecaps}{\mathcal{O}_{\mathrm{Decaps}}}
\newcommand{\OWPCA}{\styleSecurityNotion{OW-PCA}}
\newcommand{\owpca}{\styleSecurityNotion{ow-pca}}
\newcommand{\OWPCVA}{\styleSecurityNotion{OW-PCVA}}
\newcommand{\owpcva}{\styleSecurityNotion{ow-pcva}}
\newcommand{\OPco}{\mathcal{O}_{\mathrm{Pco}}}
\newcommand{\OCvo}{\mathcal{O}_{\mathrm{Cvo}}}

%%% ALGORITHMS %%%
\newcommand{\styleAlgorithm}[1]{\textrm{\upshape\mdseries #1}\xspace}
\newcommand{\dec}{\styleAlgorithm{Dec}}
\newcommand{\Dec}{\styleAlgorithm{Dec}}          %%% Patrick: duplicated because both versions are used (Lewis, please double check)
\newcommand{\Decaps}{\styleAlgorithm{Decaps}}
\newcommand{\enc}{\styleAlgorithm{Enc}}
\newcommand{\Enc}{\styleAlgorithm{Enc}}          %%% Patrick: duplicated because both versions are used (Lewis, please double check)
\newcommand{\Encaps}{\styleAlgorithm{Encaps}}
\newcommand{\Hash}{\styleAlgorithm{H}}
\newcommand{\HKDF}{\styleAlgorithm{HKDF}}
\newcommand{\HMAC}{\styleAlgorithm{HMAC}}
\newcommand{\KDF}{\styleAlgorithm{KDF}}
\newcommand{\KeyGen}{\styleAlgorithm{KeyGen}}
\newcommand{\MAC}{\styleAlgorithm{MAC}}
\newcommand{\PRF}{\styleAlgorithm{PRF}}
\newcommand{\SHAOne}{\styleAlgorithm{SHA-1}}
\newcommand{\SHATwoFiftySix}{\styleAlgorithm{SHA-256}}
\newcommand{\SHAThree}{\styleAlgorithm{SHA-3}}
\newcommand{\SHAThreeEightyFour}{\styleAlgorithm{SHA-384}}
\newcommand{\SHAFiveTwelve}{\styleAlgorithm{SHA-512}}
\newcommand{\Sign}{\styleAlgorithm{Sign}}
\newcommand{\Verify}{\styleAlgorithm{Vfy}}
\newcommand{\gen}{\styleAlgorithm{Gen}}
\newcommand{\encode}{\styleAlgorithm{ec}}
\newcommand{\decode}{\styleAlgorithm{dc}}
\newcommand{\Encode}{\styleAlgorithm{Encode}}
\newcommand{\Decode}{\styleAlgorithm{Decode}}
\newcommand{\Samp}{\styleAlgorithm{Sample}}
\newcommand{\sample}{\styleAlgorithm{Sample}}   %%% Patrick: I added this because it was missing, but not sure it's the right command (Lewis, please double check)
\newcommand{\SampV}{\styleAlgorithm{SampleMatrix}}
\newcommand{\rbg}{\styleAlgorithm{RBG}}
\newcommand{\SHAKE}{\styleAlgorithm{SHAKE}}
\newcommand{\AES}{\styleAlgorithm{AES}}
\newcommand{\AESOneTwoEight}{\styleAlgorithm{AES128}}
\newcommand{\AESOneNineTwo}{\styleAlgorithm{AES192}}
\newcommand{\AESTwoFiveSix}{\styleAlgorithm{AES256}}
\newcommand{\Pack}{\styleAlgorithm{Pack}}
\newcommand{\Unpack}{\styleAlgorithm{Unpack}}
\newcommand{\Sim}{\styleAlgorithm{S}}

%%% SCHEMES AND TRANSFORMS %%%
\newcommand{\styleScheme}[1]{\textsf{\upshape\mdseries #1}\xspace}
\newcommand{\KEM}{\styleScheme{KEM}}
\newcommand{\pke}{\styleScheme{PKE}}
\newcommand{\PKE}{\styleScheme{PKE}}        %%% Patrick: duplicated because both versions are used (Lewis, please double check)
\newcommand{\PKEOne}{\styleScheme{PKE}_1}
\newcommand{\FOnotperp}{\styleScheme{FO}^{\not\perp}}
\newcommand{\FOnotperpprime}{\styleScheme{FO}^{\not\perp\prime}}
\newcommand{\KEMnotperp}{\styleScheme{KEM}^{\not\perp}}
\newcommand{\KEMnotperpprime}{\styleScheme{KEM}^{\not\perp\prime}}
\newcommand{\T}{\styleScheme{T}}
\newcommand{\Unotperp}{\styleScheme{U}^{\not\perp}}
\newcommand{\Frodo}{\styleScheme{Frodo}}
\newcommand{\FrodoLOne}{\styleScheme{Frodo-640}}
\newcommand{\FrodoLThree}{\styleScheme{Frodo-976}}
\newcommand{\FrodoLFive}{\styleScheme{Frodo-1344}}
\newcommand{\eFrodoLOne}{\styleScheme{eFrodo-640}}
\newcommand{\eFrodoLThree}{\styleScheme{eFrodo-976}}
\newcommand{\eFrodoLFive}{\styleScheme{eFrodo-1344}}
\newcommand{\FrodoPKE}{\styleScheme{FrodoPKE}}
\newcommand{\FrodoPKERGauss}{\ensuremath{\styleScheme{FrodoPKE}_\Psi}}
\newcommand{\FrodoCCS}{\styleScheme{FrodoCCS}}
\newcommand{\FrodoKEM}{\styleScheme{FrodoKEM}}
\newcommand{\eFrodoKEM}{\styleScheme{eFrodoKEM}}
\newcommand{\FrodoPKELOne}{\styleScheme{FrodoPKE-640}}
\newcommand{\FrodoPKELThree}{\styleScheme{FrodoPKE-976}}
\newcommand{\FrodoKEMLOne}{\styleScheme{FrodoKEM-640}}
\newcommand{\FrodoKEMLThree}{\styleScheme{FrodoKEM-976}}
\newcommand{\FrodoKEMLFive}{\styleScheme{FrodoKEM-1344}}
\newcommand{\FrodoKEMLOneAES}{\styleScheme{FrodoKEM-640-AES}}
\newcommand{\FrodoKEMLOneSHAKE}{\styleScheme{FrodoKEM-640-SHAKE}}
\newcommand{\FrodoKEMLThreeAES}{\styleScheme{FrodoKEM-976-AES}}
\newcommand{\FrodoKEMLThreeSHAKE}{\styleScheme{FrodoKEM-976-SHAKE}}
\newcommand{\FrodoKEMLFiveAES}{\styleScheme{FrodoKEM-1344-AES}}
\newcommand{\FrodoKEMLFiveSHAKE}{\styleScheme{FrodoKEM-1344-SHAKE}}
\newcommand{\eFrodoKEMLOne}{\styleScheme{eFrodoKEM-640}}
\newcommand{\eFrodoKEMLThree}{\styleScheme{eFrodoKEM-976}}
\newcommand{\eFrodoKEMLFive}{\styleScheme{eFrodoKEM-1344}}
\newcommand{\eFrodoKEMLOneAES}{\styleScheme{eFrodoKEM-640-AES}}
\newcommand{\eFrodoKEMLOneSHAKE}{\styleScheme{eFrodoKEM-640-SHAKE}}
\newcommand{\eFrodoKEMLThreeAES}{\styleScheme{eFrodoKEM-976-AES}}
\newcommand{\eFrodoKEMLThreeSHAKE}{\styleScheme{eFrodoKEM-976-SHAKE}}
\newcommand{\eFrodoKEMLFiveAES}{\styleScheme{eFrodoKEM-1344-AES}}
\newcommand{\eFrodoKEMLFiveSHAKE}{\styleScheme{eFrodoKEM-1344-SHAKE}}

%%% VARIABLES %%%
\newcommand{\eps}{\ensuremath{\varepsilon}}
\newcommand{\pk}{p\mkern-0.5muk}
\newcommand{\sk}{s\mkern-1muk}
\newcommand{\pkh}{\bfp\bfk\bfh}
\newcommand{\ssk}{\bfs\bfs}
\newcommand{\sig}{\sigma}
\newcommand{\seedA}{{\mathsf{seed}_\bfA}}
\newcommand{\salt}{{\mathsf{salt}}}
\newcommand{\seedSE}{{\mathsf{seed}_{\bfS\bfE}}}
\newcommand{\nbar}{\overline{n}}
\newcommand{\mbar}{\overline{m}}
\newcommand{\Bbar}{\ensuremath{{\bar{B}}}}
\newcommand{\Cbar}{\ensuremath{{\bar{C}}}}
\newcommand{\lengthseedA}{{\mathsf{len}_{\seedA}}}
\newcommand{\lengthsalt}{{\mathsf{len}_{\salt}}}
\newcommand{\lengthseedSE}{{\mathsf{len}_{\seedSE}}}
\newcommand{\lengths}{{\mathsf{len}_{\bfs}}}
\newcommand{\lengthz}{{\mathsf{len}_{\bfz}}}
\newcommand{\lengthK}{{\mathsf{len}_{\bfk}}}
\newcommand{\lengthd}{{\mathsf{len}_{\bfd}}}
\newcommand{\lengthpkhash}{{\mathsf{len}_{\pkh}}}
\newcommand{\lengthss}{{\mathsf{len}_{\bfs\mkern-1mu\bfs}}}
\newcommand{\lengthm}{{\mathsf{len}_{\mu}}}
\newcommand{\lengthchi}{{\mathsf{len}_{\chi}}}
\newcommand{\lengthsec}{{\mathsf{len}_{\sec}}}
\newcommand{\bitstr}{{\mathsf{bits}}}
\newcommand{\numsamp}{\ensuremath{m_{\mathsf{samp}}}}

%%% SETS %%%
\newcommand{\KeySp}{\mathcal{K}}
\newcommand{\MsgSp}{\mathcal{M}}
\newcommand{\SigSp}{\mathcal{S}}
\newcommand{\RandSp}{\ensuremath{\mathcal{R}}}
\newcommand{\CtxtSp}{\ensuremath{\mathcal{C}}}

%%% FUNCTIONS %%%
\newcommand{\hint}[1]{\ensuremath{{\left\langle#1\right\rangle}}}
\newcommand{\rec}{{\mathop{\mathsf{rec}}}}
\newcommand{\renyi}{R\'enyi\xspace}
\newcommand{\RD}{\ensuremath{\mathrm{D}}}

%%% PROBLEMS %%%
\newcommand{\styleProblem}[1]{\ensuremath{\mathsf{#1}}\xspace}
\newcommand{\GapSVP}{\styleProblem{GapSVP}}
\newcommand{\SIVP}{\styleProblem{SIVP}}
\newcommand{\BDD}{\styleProblem{BDD}}
\newcommand{\DGS}{\styleProblem{DGS}}
\newcommand{\BDDwDGS}{\styleProblem{BDDwDGS}}
\newcommand{\LWE}{\styleProblem{LWE}}
\newcommand{\SLWE}{\styleSecurityNotion{SLWE}}
\newcommand{\nfSLWE}{\styleSecurityNotion{nf-SLWE}}
\newcommand{\DLWE}{\styleSecurityNotion{DLWE}}
\newcommand{\nfDLWE}{\styleSecurityNotion{nf-DLWE}}
\newcommand{\dlwe}{\styleSecurityNotion{dlwe}}
\newcommand{\slwe}{\styleSecurityNotion{slwe}}
\newcommand{\nfdlwe}{\styleSecurityNotion{nf-dlwe}}
\newcommand{\nfslwe}{\styleSecurityNotion{nf-slwe}}

%%% SHORTCUTS FOR \mathbb, \mathbf, and \mathcal %%%
\newcommand{\CC}{\mathbb{C}}
\newcommand{\FF}{\mathbb{F}}
\newcommand{\NN}{\mathbb{N}}
\newcommand{\QQ}{\mathbb{Q}}
\newcommand{\RR}{\mathbb{R}}
\newcommand{\ZZ}{\mathbb{Z}}
\newcommand{\bbA}{\mathbb{A}}
\newcommand{\bbB}{\mathbb{B}}
\newcommand{\bbC}{\mathbb{C}}
\newcommand{\bbD}{\mathbb{D}}
\newcommand{\bbE}{\mathbb{E}}
\newcommand{\bbF}{\mathbb{F}}
\newcommand{\bbG}{\mathbb{G}}
\newcommand{\bbH}{\mathbb{H}}
\newcommand{\bbI}{\mathbb{I}}
\newcommand{\bbJ}{\mathbb{J}}
\newcommand{\bbK}{\mathbb{K}}
\newcommand{\bbL}{\mathbb{L}}
\newcommand{\bbM}{\mathbb{M}}
\newcommand{\bbN}{\mathbb{N}}
\newcommand{\bbO}{\mathbb{O}}
\newcommand{\bbP}{\mathbb{P}}
\newcommand{\bbQ}{\mathbb{Q}}
\newcommand{\bbR}{\mathbb{R}}
\newcommand{\bbS}{\mathbb{S}}
\newcommand{\bbT}{\mathbb{T}}
\newcommand{\bbU}{\mathbb{U}}
\newcommand{\bbV}{\mathbb{V}}
\newcommand{\bbW}{\mathbb{W}}
\newcommand{\bbX}{\mathbb{X}}
\newcommand{\bbY}{\mathbb{Y}}
\newcommand{\bbZ}{\mathbb{Z}}
\newcommand{\bfa}{\mathbf{a}}
\newcommand{\bfb}{\mathbf{b}}
\newcommand{\bfc}{\mathbf{c}}
\newcommand{\bfd}{\mathbf{d}}
\newcommand{\bfe}{\mathbf{e}}
\newcommand{\bff}{\mathbf{f}}
\newcommand{\bfg}{\mathbf{g}}
\newcommand{\bfh}{\mathbf{h}}
\newcommand{\bfi}{\mathbf{i}}
\newcommand{\bfj}{\mathbf{j}}
\newcommand{\bfk}{\mathbf{k}}
\newcommand{\bfl}{\mathbf{l}}
\newcommand{\bfm}{\mathbf{m}}
\newcommand{\bfn}{\mathbf{n}}
\newcommand{\bfo}{\mathbf{o}}
\newcommand{\bfp}{\mathbf{p}}
\newcommand{\bfq}{\mathbf{q}}
\newcommand{\bfr}{\mathbf{r}}
\newcommand{\bfs}{\mathbf{s}}
\newcommand{\bft}{\mathbf{t}}
\newcommand{\bfu}{\mathbf{u}}
\newcommand{\bfv}{\mathbf{v}}
\newcommand{\bfw}{\mathbf{w}}
\newcommand{\bfx}{\mathbf{x}}
\newcommand{\bfy}{\mathbf{y}}
\newcommand{\bfz}{\mathbf{z}}
\newcommand{\bfA}{\mathbf{A}}
\newcommand{\bfB}{\mathbf{B}}
\newcommand{\bfC}{\mathbf{C}}
\newcommand{\bfD}{\mathbf{D}}
\newcommand{\bfE}{\mathbf{E}}
\newcommand{\bfF}{\mathbf{F}}
\newcommand{\bfG}{\mathbf{G}}
\newcommand{\bfH}{\mathbf{H}}
\newcommand{\bfI}{\mathbf{I}}
\newcommand{\bfJ}{\mathbf{J}}
\newcommand{\bfK}{\mathbf{K}}
\newcommand{\bfL}{\mathbf{L}}
\newcommand{\bfM}{\mathbf{M}}
\newcommand{\bfN}{\mathbf{N}}
\newcommand{\bfO}{\mathbf{O}}
\newcommand{\bfP}{\mathbf{P}}
\newcommand{\bfQ}{\mathbf{Q}}
\newcommand{\bfR}{\mathbf{R}}
\newcommand{\bfS}{\mathbf{S}}
\newcommand{\bfT}{\mathbf{T}}
\newcommand{\bfU}{\mathbf{U}}
\newcommand{\bfV}{\mathbf{V}}
\newcommand{\bfW}{\mathbf{W}}
\newcommand{\bfX}{\mathbf{X}}
\newcommand{\bfY}{\mathbf{Y}}
\newcommand{\bfZ}{\mathbf{Z}}
\newcommand{\cala}{\mathcal{a}}
\newcommand{\calb}{\mathcal{b}}
\newcommand{\calc}{\mathcal{c}}
\newcommand{\cald}{\mathcal{d}}
\newcommand{\cale}{\mathcal{e}}
\newcommand{\calf}{\mathcal{f}}
\newcommand{\calg}{\mathcal{g}}
\newcommand{\calh}{\mathcal{h}}
\newcommand{\cali}{\mathcal{i}}
\newcommand{\calj}{\mathcal{j}}
\newcommand{\calk}{\mathcal{k}}
\newcommand{\call}{\mathcal{l}}
\newcommand{\calm}{\mathcal{m}}
\newcommand{\caln}{\mathcal{n}}
\newcommand{\calo}{\mathcal{o}}
\newcommand{\calp}{\mathcal{p}}
\newcommand{\calq}{\mathcal{q}}
\newcommand{\calr}{\mathcal{r}}
\newcommand{\cals}{\mathcal{s}}
\newcommand{\calt}{\mathcal{t}}
\newcommand{\calu}{\mathcal{u}}
\newcommand{\calv}{\mathcal{v}}
\newcommand{\calw}{\mathcal{w}}
\newcommand{\calx}{\mathcal{x}}
\newcommand{\caly}{\mathcal{y}}
\newcommand{\calz}{\mathcal{z}}
\newcommand{\calA}{\mathcal{A}}
\newcommand{\calB}{\mathcal{B}}
\newcommand{\calC}{\mathcal{C}}
\newcommand{\calD}{\mathcal{D}}
\newcommand{\calE}{\mathcal{E}}
\newcommand{\calF}{\mathcal{F}}
\newcommand{\calG}{\mathcal{G}}
\newcommand{\calH}{\mathcal{H}}
\newcommand{\calI}{\mathcal{I}}
\newcommand{\calJ}{\mathcal{J}}
\newcommand{\calK}{\mathcal{K}}
\newcommand{\calL}{\mathcal{L}}
\newcommand{\calM}{\mathcal{M}}
\newcommand{\calN}{\mathcal{N}}
\newcommand{\calO}{\mathcal{O}}
\newcommand{\calP}{\mathcal{P}}
\newcommand{\calQ}{\mathcal{Q}}
\newcommand{\calR}{\mathcal{R}}
\newcommand{\calS}{\mathcal{S}}
\newcommand{\calT}{\mathcal{T}}
\newcommand{\calU}{\mathcal{U}}
\newcommand{\calV}{\mathcal{V}}
\newcommand{\calW}{\mathcal{W}}
\newcommand{\calX}{\mathcal{X}}
\newcommand{\calY}{\mathcal{Y}}
\newcommand{\calZ}{\mathcal{Z}}


% cost model names

\newcommand{\clsfsieve}{\textsf{C-LSF-Sieve}\xspace}
\newcommand{\qgroversieve}{\textsf{Q-Grover-Sieve}\xspace}
\newcommand{\qrandwalksieve}{\textsf{Q-RW-Sieve}\xspace}
\newcommand{\cmemsieve}{\textsf{C-2D-Sieve}\xspace}
\newcommand{\cparaenum}{\textsf{C-Para-Enum}\xspace}
\newcommand{\dff}{\ensuremath{\text{dff}}\xspace}

% Hide original text from specifications document
\newif\ifshoworiginal
%\showoriginaltrue
\showoriginalfalse

% Hide text for anonymous submission
\newif\iffullversion
%\fullversiontrue
\fullversionfalse

% Provide the title of the paper
% This should look like:
\title[running  = {FrodoKEM: A CCA-Secure Learning With Errors Key Encapsulation Mechanism}]
                  {FrodoKEM: A CCA-Secure Learning With Errors Key Encapsulation Mechanism}
% Where the options in square brackets “[ ]” are optional and control the following:
% running: the running title displayed in the headers

% Define authors and affiliations
% Authors are listed individually using the \addauthor tag followed by a list of affiliations.
% The idea is that every author makes a separate call to this command.
% This should look like:
% \addauthor[inst      = {1,2},
%            orcid     = 0000-0000-0000-0000,
%            footnote  = {Thanks to my supervisor for the support.},
%            onclick   = {https://www.mypersonalwebpage.com}
%           ]{Alice Accomplished}
% Where the options in square brackets “[ ]” are optional 
% and control the following:
% inst:     a numerical list pointing to the index of the institution 
%           in the affiliation array.
% orcid:    create a small clickable orcid logo next to the authors name 
%           linking to the authors ORCID iD see: orcid.org.
% footnote: create an author-specific footnote.
% onclick:  define what to do when clicking on the external link logo
%           next to the author name: e.g., can point to the academic webpage.
% email:    define the e-mail address of this author.
\iffullversion
\addauthor[inst     = {1},
%           footnote = {This is an example footnote.},
           email    = {lewis.glabush@epfl.ch},
           surname  = {Glabush}
          ]{Lewis Glabush}
\addauthor[inst     = {2},
%           footnote = {This is an example footnote.},
           email    = {plonga@microsoft.com},
           surname  = {Longa}
          ]{Patrick Longa}

\addauthor[inst     = {3},
%           footnote = {This is an example footnote.},
           orcid    = 0000-0003-0419-7501,
           email    = {cpeikert@umich.edu},
           surname  = {Peikert},
           onclick  = {https://web.eecs.umich.edu/~cpeikert/}
          ]{Chris Peikert}
\addauthor[inst     = {4},
%           footnote = {This is an example footnote.},
           email    = {dstebila@uwaterloo.ca},
           surname  = {Stebila}
          ]{Douglas Stebila}

%\addauthor[inst    = {2},
%           email   = {mccurley@digicrime.com},
%           surname = {McCurley},
%          ]{Kevin S. McCurley}

% The following command controls the running header for authors
% This is optional for <= 4 authors and mandatory for > 4 authors
% \authorrunning{Joppe W. Bos and Kevin S. McCurley}

% Affiliations are listed individually using the \addaffiliation command 
% *after* the (list of) authors using \addauthor
% This should look like (full example):
% \addaffiliation[ror        = 05f950310,
%                 department = {Computer Security and Industrial Cryptography},              
%                 street     = {Kasteelpark Arenberg 10, box 2452},
%                 city       = {Leuven},
%                 state      = {Vlaams-Brabant},
%                 postcode   = {3001},
%                 country    = {Belgium}
%                ]{KU Leuven}
% Where the options in square brackets “[ ]” are optional and control 
% the following (optional information is mainly used for meta-data collection):
% ror:        provide the Research Organization Registry (ROR) identifier 
%             for this affiliation (see: ror.org). This is used for meta-data 
%             collection only.
% department: department or suborganization name
% street:     street address
% city:       city name
% state:      state or province name
% postcode:   zip or postal code
% country:    country name

\addaffiliation[ 
%                ror     = 031v4g827,
%                street  = {Interleuvenlaan 80},
                city    = {Lausanne},
%                postcode= {3001},
                country = {Switzerland}
               ]{EPFL}
\addaffiliation[ 
%                ror     = 031v4g827,
%                street  = {Interleuvenlaan 80},
                city    = {Redmond},
%                postcode= {3001},
                country = {United States}
               ]{Microsoft Research}
\addaffiliation[ 
%                ror     = 031v4g827,
%                street  = {Interleuvenlaan 80},
                city    = {Ann Arbour},
%                postcode= {3001},
                country = {United States}
               ]{University of Michigan}
\addaffiliation[ 
%                ror     = 031v4g827,
%                street  = {Interleuvenlaan 80},
                city    = {Waterloo},
%                postcode= {3001},
                country = {Canada}
               ]{University of Waterloo}
\fi
% Authors should use the \addfunding macro to make sure that funding agencies
% can find papers published under their sponsorship.
% You can use the online tool at
% https://publish.iacr.org/funding
% to help you find fundref and ror identifiers.
% Note that \addfunding *does not* automatically create footnotes or
% an acknowledgements section to identify funding - it only collects the
% metadata for indexing.
% An example is:

% \addfunding[country  = {europe},
%             grantid  = {1234},
%             fundref = {100010661}
%            ]{Horizon 2020 Framework Programme}

% A footnote can be placed on the front page without a symbol / numbering using:
%\genericfootnote{This is the full version of our paper published at XX}

\begin{document}

\listoffixmes

\maketitle

% Provide the keywords *before* the abstract
% When keywords contain macros provide the text version as the optional argument
\keywords{Post-Quantum Cryptography, Lattice Cryptography, Key Exchange, Multi-ciphertext security, Implementation}

% Provide the abstract of your paper
\begin{abstract}
\plnote{Be careful when inserting notes. Some weird spacing shows up in the clean paper version when a blank space is left between a note and the text}
\plnote{Remember to provide a text-only (no macros) abstract in the ``textabstract'' environment following this one, for the final version}
  Large-scale quantum computers capable of implementing Shor's algorithm pose a significant threat to the security of the most widely used public-key cryptographic schemes.
  This risk has motivated substantial efforts by standards bodies and government agencies to identify and standardize quantum-safe cryptographic systems.
  Among the proposed solutions, lattice-based cryptography has emerged as the foundation for some of the most promising protocols.

  This paper describes \FrodoKEM, a family of conservative key-encapsulation mechanisms (KEMs) whose security is based on generic, “unstructured” lattices.
  \FrodoKEM is proposed as an alternative to the more efficient lattice schemes that utilize \emph{algebraically structured} lattices, such as the recently standardized ML-KEM scheme.
  By relying on generic lattices, \FrodoKEM minimizes the potential for future attacks that exploit algebraic structures while enabling simpler and more compact implementations.
  Our plain C implementations demonstrate that, despite its conservative design and parameterization, \FrodoKEM remains practical.
  For instance, the full protocol at NIST security level 1 runs in approximately 0.97~ms on a server-class processor, and 4.98~ms on a smartphone-class processor.

%The original version of FrodoKEM met it's security goal of single target IND-CCA security when submitted to the 2021 NIST post-quantum cryptography standardization project. However, there were concerns about security degradation when many keys would be generated for a single user. This turned out to be a generic problem for any KEM, built from the Fujisaki--Okamoto transform, that had a small message length. FrodoKEM now utilizes a new tool called the salted Fujisaki--Okamoto ($\SFO$) transform, which mitigates multi-challenge security degradation, without increasing the message length, and having a very small impact on performance. 

  \FrodoKEM obtains (single-target) \INDCCA security using a variant of the Fujisaki--Okamoto transform, applied to an underlying public-key encryption scheme called \FrodoPKE.\@
  In addition, using a new tool called the Salted Fujisaki--Okamoto ($\SFO$) transform, \FrodoKEM is also shown to \emph{tightly} achieve \emph{multi-challenge} security, without increasing the \FrodoPKE message length and with a negligible performance impact, based on the multi-challenge $\INDCPA$ security of \FrodoPKE.
\end{abstract}


\begin{textabstract}
  For the final version of your paper, you will need a text-only abstract. Do
  not use LaTeX macros inside this abstract.
\end{textabstract}

% The content of the paper starts here
\section{Introduction}%
\label{sec:introduction}

Quantum computing research has had significant implications for cryptography~\cite{You_2005, Kelly_2015}.
Currently, the most widely used asymmetric (i.e., public-key) cryptographic protocols rely on the conjectured intractability of number-theoretic problems like integer factorization and computing discrete logarithms.
However, these problems are known to be easy for large-scale quantum computers, so these computers (if they are ever built) would be able to completely break the world's most prevalent cryptography.

Motivated by this potentially catastrophic threat, standards bodies and government agencies have initiated efforts to standardize quantum-safe, or ``post-quantum,'' cryptography---i.e., systems that can be run on today's ordinary computers and networks, and are believed to be secure against quantum attacks.
In 2017, the National Institute of Standards and Technology (NIST) launched a large-scale project to select and standardize quantum-safe algorithms for digital signature, encryption, and key-establishment protocols~\cite{NIST17}.
% including the National Institute of Standards and Technology (NIST), the National Security Agency (NSA), and the PQCRYPTO project funded by the European Union \cite{Crpyto_today, NIST17}.
Among the candidates, schemes based on lattice problems---particularly the learning with errors~(LWE)~\cite{Reg09} and short integer solution~(SIS)~\cite{STOC:Ajtai96} problems, and their variants---emerged as especially promising.
% Lattice cryptography has been able to build practical, quantum safe counterparts to existing primitives, such as public key encryption schemes and key exchange mechanisms, and has also enabled new primitives, such as fully homomorphic encryption.
A significant milestone was achieved in 2022 when NIST selected two lattice-based schemes for standardization: CRYSTALS-Kyber, a key-encapsulation mechanism renamed as ML-KEM~\cite{MLKEM}, and CRYSTALS-Dilithium, a signature scheme renamed as ML-DSA~\cite{MLDSA}.
Both of these schemes are based on the Module-LWE problem, a variant of LWE with additional algebraic structure for efficiency purposes.
Additionally, NIST selected Falcon, another lattice-based signature scheme, and SPHINCS$^+$, a hash-based signature scheme.

Despite the promising security and efficiency profiles of the algorithms selected by NIST, several government agencies have expressed a desire for more conservative options with less underlying algebraic structure.
The two most notable examples are Classic McEliece~\cite{CME}, from the code-based family, and \FrodoKEM, from the lattice-based family.
These algorithms have been recommended as conservative alternatives by the German BSI~\cite{BSI}, the French ANSSI~\cite{ANSSI}, and the Dutch NLNCSA and AIVD~\cite{AIVD}.\footnote{In the latest edition of the PQC migration handbook~\cite{Dutch_HB}, the Dutch AIVD, CWI and TNO describe \FrodoKEM and Classic McEliece as more conservative options and ``strongly support ongoing initiatives aiming to standardise them''.
  The document classifies both algorithms as ``acceptable'' until their standardization is completed.}
Notably, Classic McEliece and \FrodoKEM, alongside ML-KEM, are currently undergoing standardization by the International Organization for Standardization (ISO)~\cite{ISO}.

%Given the high cost and slow deployment of entirely new cryptographic systems, the desired decades-long lifetime of such systems, and the unpredictable trajectory of quantum computing technology and quantum cryptanalysis over the coming years, we argue that any post-quantum standard should follow a conservative approach that errs comfortably on the side of security and simplicity over performance and (premature) optimization. This principle permeates the design choices behind FrodoKEM, as we now describe.

In this paper, we describe \FrodoKEM, a family of \INDCCA secure key-encapsulation mechanisms (KEMs).
\FrodoKEM is designed as a conservative yet practical post-quantum construction whose security derives from cautious parameterizations of the well-studied learning with errors (LWE) problem.
In turn, LWE has close connections to conjectured-hard problems on generic, “algebraically unstructured” lattices.

%$\FrodoKEM$ is an $\indcca$ secure key encapsulation mechanism, based on learning with errors. FrodoKEM is relies on plain LWE, and therefore is more conservative in it's assumptions than other lattice-based KEM's, which rely on structured variants of learning with errors (Ring LWE, and Module-LWE), but is still practical, and safe against known quantum threats.

\subsection{Pedigree}

The core of \FrodoKEM is a public-key encryption scheme called
\FrodoPKE, whose $\INDCPA$ security is tightly related to the
hardness of a corresponding learning with errors problem. Here
we briefly recall the scientific lineage of these systems.  See the
surveys~\cite{Micciancio10,RegevLWESurvey,DBLP:journals/fttcs/Peikert16}
for further details.

The seminal works of Ajtai~\cite{STOC:Ajtai96} (published in 1996) and
Ajtai--Dwork~\cite{STOC:AjtDwo97} (published in 1997) gave the first
cryptographic constructions whose security properties followed from
the conjectured \emph{worst-case} hardness of various problems on
point \emph{lattices} in $\bbR^{n}$.  In subsequent years, these works were
substantially refined and improved, e.g.,
in~\cite{EPRINT:GolGolHal96a,FOCS:CaiNer97,STOC:Micciancio02,DBLP:journals/jacm/Regev04,DBLP:journals/siamcomp/MicciancioR07}.
Notably, in work published in 2005, Regev~\cite{Reg09} defined the
\emph{learning with errors}~(LWE) problem, proved the hardness of
(certain parameterizations of) LWE assuming the hardness of various
worst-case lattice problems for \emph{quantum} algorithms, and
defined a public-key encryption scheme whose $\INDCPA$ security is
tightly related to the hardness of LWE.\footnote{As pointed out
  in~\cite{TCC:Peikert09_slides}, Regev's encryption scheme implicitly
  contains an (unauthenticated) ``approximate'' key-exchange protocol
  analogous to the classic Diffie--Hellman protocol~\cite{DifHel76}.}

Regev's initial work on LWE was followed by much more, which, among
other things:
\begin{itemize}
\item provided additional theoretical support for the hardness of
  various LWE parameterizations
  (e.g.,~\cite{STOC:Peikert09,C:ACPS09,STOC:BLPRS13,EC:DotMul13,C:MicPei13,STOC:PeiRegSte17}),
\item extensively analyzed the concrete security of LWE and closely
  related lattice problems
  (e.g.,~\cite{MR09:_post_quant_crypt,AC:CheNgu11,RSA:LiuNgu13,ICISC:AlbFitGop13,ChenThesis,EPRINT:ACFP14,PKC:AFFP14,LaarhovenThesis,C:KirFou15,albrecht15:_concrete_lwe,USENIX:ADPS16,CCS:BCDMNN16,EC:Albrecht17,EPRINT:AGVW17},
  among countless others), and
\item constructed LWE-based cryptosystems with improved efficiency or
  additional functionality
  (e.g.,~\cite{STOC:PeiWat08,C:PeiVaiWat08,STOC:GenPeiVai08,JC:CHKP12,FOCS:BraVai11,C:GenSahWat13,EC:BGGHNS14,C:GorVaiWee15}).
\end{itemize}
In particular, in work published in 2011, Lindner and
Peikert~\cite{RSA:LinPei11} gave a more efficient LWE-based public-key
encryption scheme that uses a square public matrix
$\bfA \in \bbZ_{q}^{n \times n}$ instead of an oblong rectangular one.
%  that relies on a similar proof
% structure as the ring-LWE-based scheme from~\cite{EC:LyuPeiReg10}.

\paragraph{Cryptographic schemes named ``Frodo''.}

Since there are now several cryptographic schemes incorporating the name ``Frodo'', we take a moment to clarify how they are related.

\begin{itemize}
\item \textbf{``Frodo'' / \FrodoCCS.}
  An LWE-based key exchange protocol called ``Frodo'' was published at the 2016 ACM CCS conference~\cite{CCS:BCDMNN16}, based on the Lindner--Peikert scheme~\cite{RSA:LinPei11} with some modifications, such as: pseudorandom generation of the public matrix~$\bfA$ from a small seed, more balanced key and ciphertext sizes, and new LWE parameters.
  For clarity, we refer to this as \FrodoCCS.
\item \textbf{\FrodoKEM (NIST submission versions).}
  An LWE-based KEM called \FrodoKEM was submitted to the NIST Post-Quantum Cryptography standardization project.
  At its heart is a public-key encryption scheme called \FrodoPKE, to which a Fujisaki--Okamoto-type transform~\cite{PKC:FujOka99} is applied to obtain a KEM with \INDCCA security.
  Some differences between \FrodoCCS and \FrodoKEM / \FrodoPKE (NIST submission versions) include:
  \begin{itemize}
  \item \FrodoCCS was described as an unauthenticated key-exchange protocol, which can equivalently be viewed as an \INDCPA-secure KEM, whereas \FrodoKEM is designed to be an \INDCCA-secure KEM.
  \item \FrodoCCS used a ``reconciliation mechanism'' to extract shared-key bits from approximately equal values (similarly to~\cite{EPRINT:DinXieLin12,PQCRYPTO:Peikert14,SP:BCNS15,USENIX:ADPS16}), whereas \FrodoKEM used simpler key transport via public-key encryption (as in~\cite{Reg09,RSA:LinPei11}).
  \item \FrodoKEM used significantly ``wider'' LWE error distributions than \FrodoCCS used, which conform to certain worst-case hardness theorems (see \autoref{sec:strength:lattice}).
  \item \FrodoKEM used different symmetric-key primitives than \FrodoCCS did.
  \end{itemize}
  Note that \FrodoPKE and \FrodoKEM changed between the first and second rounds of the NIST PQC standardization project.
\item \textbf{\FrodoKEM and \eFrodoKEM (ISO submission version~\cite{ISOdraft}, and this document).}
  In response to concerns about multi-ciphertext attacks (see \autoref{sec:intro:multi-challenge} below), a new version of \FrodoKEM was defined that included a \emph{salt} as a countermeasure.
  As of the ISO submission version, the name \eFrodoKEM (with the ``e'' meaning ``ephemeral'') refers to the un-salted version (which is equivalent to the NIST Round~3 version of \FrodoKEM), and the name \FrodoKEM refers to the salted version.
  The \FrodoPKE scheme remains unchanged since the Round 2 version.
\item We also use the name ``\FrodoKEM'' to refer to the overall family of schemes consisting of the ephemeral KEM \eFrodoKEM and the salted KEM \FrodoKEM.
\end{itemize}

\subsection{Structured versus unstructured lattices}

Problems that underlie the security of cryptographic protocols often have special ``structured'' instances, whose use may offer better efficiency or other advantages, but may also introduce security vulnerabilities that are not present in the general case.
In certain cases, the security/efficiency trade-off associated with special instances can be well understood and balanced\cpnote{is this example really so well understood? How about the range of exponents in dlog instead?}
(for example, small decryption exponents in RSA).
In other cases, the presence and degree of security loss from using structured instances remains unknown or not well understood.

In lattice-based cryptography, the (plain) LWE problem relates to solving a “noisy” linear system (modulo a known integer); it can also be interpreted as the problem of decoding a random “unstructured” lattice from a certain broad class.
There are also several variants of LWE where the linear system (and its associated lattice) has additional algebraic structure, which offers advantages in terms of computational efficiency and size.
These variants include Ring-LWE~\cite{DBLP:journals/jacm/LyubashevskyPR13}, Module-LWE~\cite{ITCS:BraGenVai12,DBLP:journals/dcc/LangloisS15} and the NTRU problem~\cite{HofPipSil98}.
Against the NTRU problem, there have been a number of attacks which exploit the algebraic structure~\cite{Soliloquy, EfficientClass}.\cpnote{These attacks do \emph{not} apply to NTRU, but only some ad-hoc ideal-lattice problems (principal with promised short generator).
  There are also the papers that obtain better subexp approx factors for worst-case ideal lattices, which we should cite here.}
Though these attacks do not extend to Ring-LWE or Module-LWE, they demonstrate that algebraic structure in lattices may allow for attacks that are not possible on generic lattices.

\subsection{Multi-challenge security}%
\label{sec:intro:multi-challenge}

\paragraph{Multi-key security.}

Multi-key (also known as multi-user) attacks aim to break security against \emph{any one} of many available public keys.\footnote{In the literature, multi-key attacks are sometimes also known as multi-\emph{target} attacks, but the latter term can also refer to multi-\emph{ciphertext} attacks.
  So, to avoid ambiguity, we eschew the term ``multi-target.''}
In its call for proposals for the post-quantum standardization process~\cite{NIST17}, NIST lists ``resistance to multi-key attacks'' as a ``desirable property.''

\FrodoKEM's primary security target of \INDCCA considers only a single public key, so multi-key attacks fall outside its immediate scope.
However, multi-key security generically follows from \INDCCA security by a routine hybrid argument, with linear concrete security loss in the number of keys.
In addition, all versions of \FrodoKEM (and \FrodoCCS before it) include a specific countermeasure against multi-key attacks, namely, a distinct LWE matrix~$\bfA$ for each public key.

\paragraph{Multi-ciphertext security.}

Multi-ciphertext attacks target a single public key, but aim to break \emph{any one} of many ciphertexts produced under that key.
As above, multi-ciphertext security falls outside the scope of \FrodoKEM's primary security target of \INDCCA, and (in contrast with multi-key security) NIST's call for proposals did not mention multi-ciphertext security.
However, it is a desirable feature in applications where a large number of ciphertexts will be encrypted under the same public key.

In 2021, a multi-ciphertext attack against the Round-3 \FrodoKEMLOne parameters (for NIST security level 1) and proposed fix was identified by NIST~\cite{perlner21}; the same attack was also identified by Bernstein in~\cite{EPRINT:Bernstein22d}.
Although the attack does not invalidate any of the security claims from the \FrodoKEM submission, it may be of concern for applications with long-lived public keys.
For this reason, subsequent versions of \FrodoKEM specifically target multi-ciphertext security (and multi-challenge security more generically; see below) via a \emph{salted} version of the Fujisaki--Okamoto transform.
For compatibility purposes, and for applications where the number of ciphertexts produced under any single public key is fairly small, there is \emph{ephemeral} \FrodoKEM (or \eFrodoKEM), which is identical to Round-3 \FrodoKEM.

The multi-ciphertext attack stems from \FrodoKEMLOne's message length of 128 bits, and the fact that encryption is deterministic, which is true of any KEM built from the Fujisaki--Okamoto transform.
More specifically, when there are~$n$ challenge ciphertexts available, an adversary can sample~$N$ distinct messages, and calculate the corresponding ciphertexts and KEM keys.
If a collision is found (between one of the challenge ciphertexts and one of the generated ciphertexts), the adversary succeeds in breaking one-wayness, and thereby indistinguishability.
By the birthday approximation, the probability of finding a collision is approximately $nN/\abs{M}$, where~$M$ is the message space and $n,N \ll \abs{M}$.
In particular, with \FrodoKEMLOne's message space of size $|M| = 2^{128}$, and (say) $n=2^{40}$ challenge ciphertexts, an adversary can break multi-ciphertext security by doing about $N = 2^{88}$ encryptions (and the collision search).

\paragraph{Multi-challenge security.}

Multi-challenge security unifies both multi-key and multi-ciphertext security.
Here, the adversary is given potentially several public keys and ciphertexts, where each ciphertext is generated under one of the public keys of the adversary's choice.\footnote{For a finer-grained notion, one can limit the total number of ciphertexts that may be generated under any single key.}
To obtain multi-challenge security, \FrodoKEM uses the above-mentioned Salted Fujisaki--Okamoto ($\SFO$) transform, and hashing of public keys.
The efficacy of these techniques was proven in~\cite{GlabushThesis}.

% CJP: suppressing this technical stuff about oracles and specific notation, which isn't used.
% The adversary is given access to a challenge oracle, which takes as input an index~$i$, and returns a ciphertext, encrypted under $pk_{i}$, and the corresponding KEM key.
% Denote $n_j$ as the number of queries the adversary makes under public key $pk_j$.
% We fix a threshold for $n$, say $2^{80}$, which models the total number of instances in which the protocol is used.
% Furthermore, it may be realistic to also restrict the number of queries the adversary can make to the challenge oracle for a single $j$, for example, by restricting $n_i \leq 2^{64}$ for all $i \in [u]$.

\subsection{Our contributions}

In this work, we present and analyze \FrodoKEM, a family of key-encapsulation mechanisms that rely on the learning with errors problem to obtain security against known quantum threats.
Our focus is on members of the \FrodoKEM family presented in the ISO submission~\cite{ISOdraft}, namely: \eFrodoKEM, the IND-CCA-secure KEM built from \FrodoPKE using the Fujisaki--Okamoto ($\FOnotperpprime$) transform; and \FrodoKEM, the multi-challenge-IND-CCA-secure KEM built from \FrodoPKE using the salted FO transform.
Supporting algorithms for \FrodoKEM are presented in \autoref{sec:algs}, \FrodoPKE is presented in \autoref{sec:cpa-pke}, and \FrodoKEM is presented in \autoref{sec:cca-kem}.

Given the existing work on past versions of Frodo in~\cite{CCS:BCDMNN16,NISTPQC-R1:FrodoKEM17,NISTPQC-R2:FrodoKEM19,NISTPQC-R3:FrodoKEM20,ISOdraft}, the specific contributions of this work are as follows.
\begin{itemize}
\item We apply the analysis of the salted FO transform \cpnote{update ref:} of~\cite{Multi-challenge} to derive results on the multi-challenge (i.e., multi-key and multi-ciphertext) security of \FrodoKEM, addressing the multi-ciphertext attack described in \autoref{sec:intro:multi-challenge}.
 These results, reported in \autoref{sec:strength:justification} and \autoref{sec:Security_appendix}, include bounds in the quantum random oracle model.
\item In \autoref{sec:params} and \autoref{sec:attack:cryptanalytic}, we incorporate recent cryptanalytic developments into our estimates for the security levels of the three parameterizations \FrodoLOne, \FrodoLThree, and \FrodoLFive.
\item In \autoref{sec:performance} we provide updated implementations and performance measurements for \FrodoKEM and \eFrodoKEM on Intel x64 and ARM platforms, with an optimized C implementation taking advantage of AES-NI and AVX2 instructions on Intel x64.
 These implementations are available anonymized as part of this submission at \url{https://doi.org/10.5281/zenodo.14633189}.
\end{itemize}

%%% Local Variables:
%%% mode: latex
%%% TeX-master: "Main"
%%% End:


\section{Background}%
\label{sec:background}

This section defines the cryptographic primitives and security notions that
are relevant to \FrodoPKE and \FrodoKEM, as well as the mathematical
background required to analyze their security.

\subsection{Notation}%
\label{sec:notation}

\ifshoworiginal
We use the following notation:

\begin{itemize}
\item Vectors are denoted with bold lower-case letters (e.g.,
  $\bfa, \bfb, \bfv$), and matrices are denoted with bold upper-case
  letters (e.g., $\bfA, \bfB, \bfS$). For a set~$D$, the set of
  $m$-dimensional vectors with entries in~$D$ is denoted by~$D^{m}$,
  and the set of $m$-by-$n$ matrices with entries
  in~$D$ is denoted by $D^{m \times n}$.
\item For an $n$-dimensional vector $\bfv$, its $i$th entry for
  $0\leq i < n$ is denoted by~$\bfv_i$.
\item For an $m$-by-$n$ matrix~$\bfA$, its $(i,j)$th entry (i.e., the
  entry in the $i$th row and $j$th column) for $0 \leq i <m$ and
  $0 \leq j < n$ is denoted by~$\bfA_{i,j}$, and its $i$th row is
  denoted by~$\bfA_i = (\bfA_{i,0},\bfA_{i,1},\ldots,\bfA_{i,n-1})$.
\item The transpose of a matrix~$\bfA$ is denoted by $\bfA^\text{T}$.  
\item An $m$-bit string $\bfk\in \set{0,1}^m$ is written as a vector over
  the set $\{0,1\}$ and its $i$th bit for
  $0 \leq i < m$ is denoted by~$\bfk_i$. 
\item The ring of integers is denoted by $\bbZ$, and, for a positive
  integer~$q$, the quotient ring of integers modulo~$q$ is denoted by
  $\bbZ_q = \bbZ/q\bbZ$.
\item For a probability distribution~$\chi$, the notation
  $e \getsr \chi$ denotes drawing a value~$e$ according to~$\chi$.
  The $n$-fold product distribution of~$\chi$ with itself is denoted
  by~$\chi^{n}$.
\item For a finite set~$S$, the uniform
  distribution on~$S$ is denoted by~$U(S)$.
\item The floor of a real number~$a$, i.e., the largest integer less
  than or equal to $a$, is denoted by~$\floor{a}$.
\item The closest integer to a real number~$a$ (with ties broken
  upward) is denoted by $\round{a} = \floor{a+1/2}$.
\item For a real vector $\bfv\in \bbR^n$, its Euclidean (i.e.,
  $\ell_{2}$) norm is denoted by~$\length{\bfv}$.
\item For two $n$-dimensional vectors $\bfa, \bfb$ over a common
  ring~$R$, their inner product is denoted by
  $\inner{\bfa, \bfb} = \sum_{i=0}^{n-1} \bfa_{i} \bfb_{i} \in
  R$.
\end{itemize}

\else

Vectors are denoted by bold lower-case letters (e.g., $\bfa, \bfb, \bfv$),
and matrices are denoted by bold upper-case letters (e.g., $\bfA, \bfB, \bfS$).
For a set~$D$, the set of $m$-dimensional vectors with entries in~$D$ is denoted by~$D^{m}$,
and the set of $m$-by-$n$ matrices with entries in~$D$ is denoted by $D^{m \times n}$.
For an $n$-dimensional vector $\bfv$, its $i$th entry for $0\leq i < n$ is denoted by~$\bfv_i$.
For an $m$-by-$n$ matrix~$\bfA$, its $(i,j)$th entry (i.e., the entry in the $i$th row and $j$th
column) for $0 \leq i <m$ and $0 \leq j < n$ is denoted by~$\bfA_{i,j}$, and its $i$th row is
denoted by~$\bfA_i = (\bfA_{i,0},\bfA_{i,1},\ldots,\bfA_{i,n-1})$.
The transpose of a matrix~$\bfA$ is denoted by $\bfA^\text{T}$.  

An $m$-bit string $\bfk\in \set{0,1}^m$ is written as a vector over the set $\{0,1\}$.
The ring of integers is denoted by $\bbZ$, and, for a positive integer~$q$, the quotient ring
of integers modulo~$q$ is denoted by $\bbZ_q = \bbZ/q\bbZ$.
For a probability distribution or randomized algorithm~$\chi$, the notation $e \getsr \chi$ denotes drawing a value~$e$
according to~$\chi$.
The $n$-fold product distribution of~$\chi$ with itself is denoted by~$\chi^{n}$.
For a finite set~$S$, the uniform distribution on~$S$ is denoted by~$U(S)$.
The floor of a real number~$a$, i.e., the largest integer less than or equal to $a$,
is denoted by~$\floor{a}$.
The closest integer to a real number~$a$ (with ties broken upward) is denoted by
$\round{a} = \floor{a+1/2}$.
For a real vector $\bfv\in \bbR^n$, its Euclidean (i.e.,~$\ell_{2}$) norm is denoted
by~$\length{\bfv}$.
For two $n$-dimensional vectors $\bfa, \bfb$ over a common ring~$R$, their inner product
is denoted by $\inner{\bfa, \bfb} = \sum_{i=0}^{n-1} \bfa_{i} \bfb_{i} \in R$.

\fi

\subsection{Cryptographic definitions}%
\label{sec:background:crypto}

This section recalls definitions of cryptographic primitives that are relevant to \FrodoKEM, along with their correctness and security notions.

\begin{definition}[Key encapsulation mechanism]
A \emph{key encapsulation mechanism} $\KEM$ is a tuple of algorithms
$(\KeyGen,\Encaps,\Decaps)$ along with a finite keyspace $\KeySp$:
\begin{itemize}
\item $\KeyGen() \tor (\pk, \sk)$: A probabilistic \emph{key generation algorithm} that outputs a public key $\pk$ and a secret key $\sk$.
\item $\Encaps(\pk) \tor (c, \ssk)$: A probabilistic \emph{encapsulation algorithm} that takes as input a public key $\pk$, and outputs an encapsulation $c$ and a shared secret $\ssk \in \KeySp$.  The encapsulation is sometimes called a ciphertext.
\item $\Decaps(c, \sk) \to \ssk'$: A (usually deterministic) \emph{decapsulation algorithm} that takes as input an encapsulation $c$ and a secret key $\sk$, and outputs a shared secret $\ssk' \in \KeySp$.
\end{itemize}
\end{definition}

The notion of $\delta$-correctness gives a bound on the probability of a legitimate protocol execution producing different keys in encapsulation and decapsulation.

\begin{definition}[$\delta$-correctness for KEMs]
A key encapsulation mechanism $\KEM$ is \emph{$\delta$-correct} if
\[ \Pr\left[ \ssk' \ne \ssk : (\pk, \sk) \getsr \KEM.\KeyGen(); (c, \ssk) \getsr \KEM.\Encaps(\pk);  \ssk' \gets \KEM.\Decaps(c, \sk) \right] \le \delta\enspace . \]
\end{definition}

%\begin{definition}[$\delta$-correctness of a PKE]
%A public key encryption scheme $\pke = (\gen, \enc, \dec)$ with message space $\M$ is called $\delta$-correct if

%\[\mathbb{E}_{(pk,sk)}[\max_{m \in \M} \Pr[\dec(sk, c) \neq m | c \leftarrow \enc(pk, m)] \leq \delta\]

%\noindent where the expectation is taken over all $(pk,sk) \sample \gen()$
%\end{definition}

%In settings where there are multiple users, we need a definition of multi-user correctness. We denote this as $\delta(u)$ which is the maximum $\delta$, taken over all $u$ users. 

%\begin{definition}[$\delta(u)$-correctness of a PKE]
%A public key encryption scheme $\pke = (\gen, \enc, \dec)$ with message space $\M$ is called $\delta(u)$-correct if

%\[\mathbb{E}[\max_{j \in [u]}\max_{m \in \M} \Pr[\dec(sk_j, c) \neq m | c \leftarrow \enc(pk_j, m)] \leq \delta(u)\]

%\noindent where the expectation is taken over all $(pk_1,sk_1),\ldots,(pk_u, sk_u) \sample \gen().$
%\end{definition}


The following defines indistinguishability under chosen-ciphertext attack (\INDCCA) for a key encapsulation mechanism.

\begin{definition}[$\INDCCA$ for KEMs]
  Let $\KEM$ be a key encapsulation mechanism (with keyspace $\KeySp$), and let~$\Adversary$ be an algorithm.
  The security experiment for \emph{indistinguishability under adaptive chosen ciphertext attack (\INDCCATwo, or just \INDCCA)} of $\KEM$ is $\Exp{\indcca}{\KEM}(\Adversary)$, as defined in \autoref{fig:kem-indcca}.
  The advantage of~$\Adversary$ in the experiment is
\[ \Adv{\indcca}{\KEM}(\Adversary) := \left| \Pr\left[ \Exp{\indcca}{\KEM}(\Adversary) \Rightarrow 1 \right] - \frac{1}{2} \right| \enspace . \]
\end{definition}

Note that $\Adversary$ can be a classical or quantum algorithm.
Even if $\Adversary$ is a quantum algorithm, we still require it to make classical queries to its $\ODecaps$ oracle.

\begin{figure}[h]
\centering
\fbox{
\begin{minipage}[t]{0.4\textwidth}
\underline{Experiment $\Exp{\indcca}{\KEM}(\Adversary)$:}
\vspace{-1em}
\begin{algorithmic}[1]
\STATE $(\pk, \sk) \getsr \KEM.\KeyGen()$
\STATE $b \getsr \{0,1\}$
\STATE $(c^*, \ssk_0) \getsr \KEM.\Encaps(\pk)$
\STATE $\ssk_1 \getsr U(\KeySp)$
\STATE $b' \getsr \Adversary^{\ODecaps(\cdot)}(\pk,\ssk_b, c^*)$
\IF {$b'=b$}
  \RETURN $1$
\ELSE
  \RETURN $0$
\ENDIF
\end{algorithmic}
\end{minipage}
~
\begin{minipage}[t]{0.4\textwidth}
\underline{Oracle $\ODecaps(c)$:}
\vspace{-1em}
\begin{algorithmic}[1]
\IF {$c = c^*$}
  \RETURN $\bot$
\ELSE
  \RETURN $\KEM.\Decaps(c, \sk)$
\ENDIF
\end{algorithmic}
\end{minipage}
}
\caption{Security experiment for indistinguishability under adaptive
  chosen ciphertext attack (\INDCCATwo, or just \INDCCA) of a key
  encapsulation mechanism $\KEM$ for an adversary $\Adversary$.}
\label{fig:kem-indcca}
\end{figure}

The key encapsulation mechanism presented in this work is obtained by transforming a \emph{public-key encryption} (PKE) scheme, which is formally defined as follows.

\begin{definition}[Public-key encryption scheme]
  A \emph{public-key encryption scheme} $\PKE$ is a tuple of
  algorithms $(\KeyGen,\Enc,\Dec)$ along with a message space
  $\MsgSp$:
\begin{itemize}
\item $\KeyGen() \tor (\pk, \sk)$: A probabilistic \emph{key generation algorithm} that outputs a public key $\pk$ and a secret key $\sk$.
\item $\Enc(m, \pk) \tor c$: A probabilistic \emph{encryption
    algorithm} that takes as input a message $m \in \MsgSp$ and public
  key $\pk$, and outputs a ciphertext $c$.  The deterministic form is
  denoted $\Enc(m, \pk; r) \to c$, where the randomness
  $r \in \RandSp$ is passed as an explicit input; $\RandSp$ is called
  the \emph{randomness space} of the encryption algorithm.
\item $\Dec(c, \sk) \to m'$ or $\bot$: A deterministic
  \emph{decryption algorithm} that takes as input a ciphertext $c$ and
  secret key $\sk$, and outputs either a message $m' \in \MsgSp$ or a
  special error symbol $\bot \notin \MsgSp$.
\end{itemize}
\end{definition}

The notion of $\delta$-correctness captures an upper bound on the
probability of decryption failure in a legitimate execution of the
scheme.
\begin{definition}[$\delta$-correctness for PKEs~\cite{TCC:HofHovKil17}]
  \label{def:delta-correct}
  A public-key encryption scheme $\PKE$ with message space $\MsgSp$ is
  \emph{$\delta$-correct} if
  \begin{equation}
    \label{eq:delta-correct}
    \E\left[ \max_{m \in \MsgSp} \Pr\left[ \PKE.\Dec(c, \sk)
        \neq m : c \getsr \PKE.\Enc(m, \pk) \right] \right] \leq
    \delta \enspace,
  \end{equation}
  where the expectation is taken over
  $(\pk, \sk) \getsr \PKE.\KeyGen()$.
\end{definition}
In our PKE, the probability expression in
\autoref{eq:delta-correct} has no dependence on~$m$, so the
condition simplifies to
\begin{equation}
  \label{eq:simpler-delta-correct}
  \Pr\left[ \PKE.\Dec(c, \sk)
    \neq m : (\pk, \sk) \gets \PKE.\KeyGen(); c \getsr \PKE.\Enc(m,
    \pk) \right]  \leq \delta \enspace ,
\end{equation}
which is what we analyze when calculating the probability of
decryption failure (see \autoref{sec:cpa-pke-correctness}).

The PKE scheme we use as the basis for the KEM transformation in
\autoref{sec:cca-transform} is required to satisfy the
notion of \INDCPA security, which is defined as follows.

\begin{definition}[$\INDCPA$ for PKE]%
  \label{def:IND-CPA}
  Let $\PKE$ be a public-key encryption scheme (with message space~$\MsgSp$), and let $\Adversary$ be an algorithm.
  The security experiment for \emph{indistinguishability under chosen plaintext attack (\INDCPA)} of $\PKE$ is $\Exp{\indcpa}{\PKE}(\Adversary)$ shown in \autoref{fig:pke-indcpa}.
  The advantage of~$\Adversary$ in the experiment is
  \cpnote{We define this and other IND advantages as (absolute) differences between ``correct guessing probability'' and $1/2$.
    But we define the LWE advantages as the (absolute) difference between acceptance probabilities given one of two distributions.
    These are equal, \emph{if} we multiply the former style by $2$ to get a number between $0$ and $1$.
    Otherwise, they have incompatible ``units'' things can get messed up switching between them.
    So I would favor including the factor of $2$ in these definitions, but we must check what definitions the external results we rely on use.
    And there is another issue: when game-hopping, we have to link the advantages in adjacent games to the advantage in the game that the reduction ``embeds,'' and this is more subtle when advantage is defined in terms of a ``guessing game'' versus with a difference of acceptance probabilities.}
\[ \Adv{\indcpa}{\PKE}(\Adversary) := \left| \Pr\left[ \Exp{\indcpa}{\PKE}(\Adversary) \Rightarrow 1 \right] - \frac{1}{2} \right| . \]
\end{definition}
% Note that $\Adversary$ can be a classical or quantum algorithm.

\begin{figure}[h]
\centering
\fbox{
\begin{minipage}[t]{0.4\textwidth}
\underline{Experiment $\Exp{\indcpa}{\PKE}(\Adversary)$:}
\vspace{-1em}
\begin{algorithmic}[1]
\STATE $(\pk, \sk) \getsr \PKE.\KeyGen()$
\STATE $(m_0, m_1, st) \getsr \Adversary(\pk)$
\STATE $b \getsr \{0,1\}$
\STATE $c^* \getsr \PKE.\Enc(m_b, \pk)$
\STATE $b' \getsr \Adversary(\pk, c^*, st)$
\IF {$b'=b$}
  \RETURN $1$
\ELSE
  \RETURN $0$
\ENDIF
\end{algorithmic}
\end{minipage}
}
\caption{Security experiment for indistinguishability under chosen plaintext attack (\INDCPA) of a public-key encryption scheme $\PKE$ against an adversary $\Adversary$.}
\label{fig:pke-indcpa}
\end{figure}

In the multi-user (or more generally, multi-challenge) setting, we need a suitable definition of multi-user correctness.
% \cnote{the next sentence doesn't really make sense: delta(u) is not the maximum $\delta$ (which doesn't mean anything), it is merely a *bound*.}
% We denote this as $\delta(u)$, which is the maximum $\delta$ taken over all $u$ users.

\begin{definition}[$\delta(u)$-correctness for PKEs]%
  \label{def:delta-correctness}
  A public-key encryption scheme $\pke$ with message space $\MsgSp$ is called $\delta(u)$-correct if for all~$u$,
  \[ \E\left[ \max_{j \in [u]} \max_{m \in \M} \Pr[\PKE.\dec(sk_j, \PKE.\enc(pk_j, m)) \neq m] \right] \leq \delta(u) \]
  where the expectation is taken over $(pk_1,sk_1),\ldots,(pk_u, sk_u) \getsr \PKE.\KeyGen()$.
\end{definition}

We note that in \FrodoPKE, the probability of incorrect decryption does not depend on the choice of message $m \in \M$, so the expression from \autoref{def:delta-correctness} simplifies to
\[ \E\left[ \max_{j \in [u]} \Pr[\PKE.\dec(sk_j, \PKE.\enc(pk_j, m)) \neq m] \right] ,
\]
where $m \in \M$ is some arbitrary message.
Also note that by restricting to $u=1$, the above definitions simplify to merely the single-user correctness~$\delta$~\cite{TCC:HofHovKil17}.
Because the maximum of several non-negative values is at most their sum, by linearity of expectation, any $\pke$ is $\delta(u)$-correct for $\delta(u) = \delta \cdot u$.
However, there is strong evidence that for \FrodoPKE and other natural lattice-based schemes, $\delta(u) \ll \delta \cdot u$; see~\cite[Table~1]{CCS:DHKLS21}.

In the multi-challenge security experiments, an oracle generates each ciphertext by encrypting under one of several (properly generated) public keys, as specified by the adversary.
Formally, we define \MINDCPA security for a PKE, and \MINDCCA security for a KEM.
Note that \INDCPA and \INDCCA are simply the special cases of these notions, respectively, for $n=u=1$.

\begin{definition}[\MINDCPA for PKE~\cite{GlabushThesis}]%
  \label{def:MIND-CPA}
  Let \PKE be a public-key encryption scheme and let~$\Adversary$ be an algorithm.  The \MINDCPA security
  experiment for~$\Adversary$ attacking \PKE is
  $\Exp{\MINDCPA}{\PKE}(\Adversary)$ from \autoref{fig:pke-mindcpa}.  The
  advantage of~$\Adversary$ in the experiment is
  \[ \Adv{\MINDCPA}{\PKE}(\Adversary) := \left| \Pr\left[ \Exp{\MINDCPA}{\PKE}(\Adversary) \Rightarrow 1 \right] - \frac{1}{2} \right| . \]
 
\end{definition}

\begin{figure}[h]
	\centering
	\fbox{
		\begin{minipage}[t]{0.4\textwidth}
			\underline{Experiment $\Exp{\MINDCPA}{\PKE}(\Adversary)$:}
			\vspace{-1em}
			\begin{algorithmic}[1]
                    \STATE $b \getsr \{0,1\}$
                    \FOR {$ j = 1,\ldots, u$} \STATE{$(\pk_j, \sk_j) \gets \PKE.\KeyGen()$} \ENDFOR
                    \STATE $\Vec{\pk} = (\pk_1,\ldots,\pk_u)$
				\STATE $b' \gets
                                \Adversary^{\challoracle}(\Vec{\pk})$
                                \IF {$b = b'$}
				\RETURN 1
				\ELSE
				\RETURN 0
				\ENDIF
			\end{algorithmic}
		\end{minipage}
		~
		\begin{minipage}[t]{0.4\textwidth}
                \underline{Oracle $\challoracle(m_0, m_1, j):$}
                \vspace{-1em}
                \begin{algorithmic}
                  \STATE [may be called up to $n$ times]
                \STATE $c \gets \PKE.\Enc(pk_j, m_b)$
                \RETURN $c$
                \end{algorithmic}
		\end{minipage}
	}
	\caption{Security experiment for \MINDCPA security of a public-key encryption scheme $\PKE$ against an adversary $\Adversary$.}%
	\label{fig:pke-mindcpa}
\end{figure}

\begin{definition}[\MINDCCA for KEM~\cite{CCS:DHKLS21}]%
  \label{def:MIND-CCA}
  Let \KEM be a key encapsulation mechanism (with key space $\KeySp$), and let~$\Adversary$ be an algorithm.
  The \MINDCCA security experiment for $\Adversary$ attacking \KEM is $\Exp{\MINDCCA}{\KEM}(\Adversary)$ from \autoref{fig:kem-mindcca}.
  The advantage of~$\Adversary$ in the experiment is
 \[ \Adv{\MINDCCA}{\KEM}(\Adversary) := \left| \Pr\left[ \Exp{\MINDCCA}{\KEM}(\Adversary) \Rightarrow 1 \right] - \frac{1}{2} \right| . \]
\end{definition}

\begin{figure}[h]
  \centering
  \fbox{
    \begin{minipage}[t]{0.4\textwidth}
      \underline{Experiment $\Exp{\MINDCCA}{\KEM}(\Adversary)$:}
      \vspace{-1em}
      \begin{algorithmic}[1]
        \STATE $b \getsr \{0,1\}$
        \FOR {$ j = 1,\ldots, u$} \STATE{$(\pk_j, \sk_j) \gets \KEM.\KeyGen()$} \ENDFOR
        \STATE $\Vec{\pk} = (\pk_1,\ldots,\pk_u)$
        \STATE $b' \gets \Adversary^{\ODecaps,\challoracle}(\Vec{\pk})$
        \IF {$b = b'$}
        \RETURN 1
        \ELSE
        \RETURN 0
        \ENDIF
      \end{algorithmic}
    \end{minipage}
    ~
    \begin{minipage}[t]{0.4\textwidth}
      \underline{Oracle $\ODecaps(j,c)$:}
      \vspace{-1em}
      \begin{algorithmic}[1]
        \IF {$c \in \cipherlist$}
        \RETURN $\bot$ 
        \ELSE
        \RETURN $\KEM.\Decaps(c, \sk_j)$
        \ENDIF
      \end{algorithmic}
      \underline{Oracle $\challoracle(j):$}
      \vspace{-1em}
      \begin{algorithmic}[1]
        \STATE [may be called up to $n$ times]
        \STATE $(c, k_0) \gets \KEM.\Encaps(pk_j)$
	\STATE $\cipherlist := \cipherlist \cup{c}$
        \STATE $k_1 \getsr \mathcal{K}$
        \RETURN $(c, k_b)$
      \end{algorithmic}
    \end{minipage}
  }
  \caption{Security experiment for \MINDCCA security of a key encapsulation mechanism $\KEM$ against an adversary $\Adversary$.}%
  \label{fig:kem-mindcca}
\end{figure}


\subsection{Learning with errors}%
\label{sec:lwe}

The security of our proposed PKE and KEM relies on the hardness of the
\emph{Learning With Errors} (LWE) problem, a generalization of the
classic Learning Parities with Noise problem (see,
e.g.,~\cite{C:BFKL93}) first defined by Regev~\cite{Reg09}. This
section defines the LWE probability distributions and computational
problems.

\begin{definition}[LWE distribution]%
  \label{def:lwe-distrib}
  Let $n, q$ be positive integers, and let $\chi$ be a distribution
  over $\bbZ$.  For an $\bfs \in \bbZ_q^n$, the \emph{LWE
    distribution} $A_{\bfs, \chi}$ is the distribution over
  $\bbZ_q^n \times \bbZ_q$ obtained by choosing $\bfa \in \bbZ_q^n$
  uniformly at random and an integer error $e \in \bbZ$ from~$\chi$,
  and outputting the pair
  $(\bfa, \inner{\bfa, \bfs} + e \bmod q) \in \bbZ_q^n \times \bbZ_q$.
\end{definition}

There are two main kinds of computational LWE problem: \emph{search},
which is to recover the secret~$\bfs \in \bbZ_{q}^{n}$ given a certain
number of samples drawn from the LWE distribution~$A_{\bfs, \chi}$;
and \emph{decision}, which is to distinguish a certain number of
samples drawn from the LWE distribution from uniformly random samples.
For both variants, one often considers two distributions of the
secret~$\bfs \in \bbZ_{q}^{n}$: the uniform distribution, and the
distribution~$\chi^{n} \bmod{q}$ where each coordinate is drawn from
the error distribution~$\chi$ and reduced modulo~$q$. The latter is
often called the ``normal form'' of LWE.

\begin{definition}[LWE Search Problem]%
  \label{def:slweproblem}
  Let $n, m, q$ be positive integers, and let $\chi$ be a distribution
  over~$\bbZ$.  The \emph{uniform-secret} (respectively,
  \emph{normal-form}) learning with errors \emph{search} problem with
  parameters $(n, m, q, \chi)$, denoted by $\SLWE_{n,m,q,\chi}$
  (respectively, $\nfSLWE_{n,m,q,\chi}$), is as follows: given~$m$
  samples from the LWE distribution $A_{\bfs, \chi}$ for uniformly
  random~$\bfs$ (resp, $\bfs \getsr \chi^{n} \bmod q$), find $\bfs$.
  More formally, for an adversary~$\Adversary$, define (for the
  uniform-secret case)
  \[ \Adv{\slwe}{n,m,q,\chi}(\Adversary) = \Pr \bracks*{
      \Adversary(((\bfa_i, b_{i}))_{i=1,\ldots,m}) \Rightarrow \bfs
      : \bfs \getsr U(\bbZ_q^n), (\bfa_{i}, b_{i}) \getsr A_{\bfs,
        \chi} \text{ for } i=1,\ldots,m } \enspace . \] Similarly,
  define (for the normal-form case)
  $\Adv{\nfslwe}{n,m,q,\chi}(\Adversary)$, where
  $\bfs \getsr \chi^{n} \bmod q$ instead of
  $\bfs \getsr U(\bbZ_{q}^{n})$.
\end{definition}

\begin{definition}[LWE Decision Problem]%
  \label{def:dlweproblem}
  Let $n, m, q$ be positive integers, and let~$\chi$ be a distribution
  over~$\bbZ$.  The \emph{uniform-secret} (respectively,
  \emph{normal-form}) \emph{learning with errors decision problem}
  with parameters $(n, m, q, \chi)$, denoted $\DLWE_{n,m,q,\chi}$ 
  (respectively, $\nfDLWE_{n,m,q,\chi}$), is as follows: distinguish~$m$
  samples drawn from the LWE distribution $A_{\bfs, \chi}$ from~$m$
  samples drawn from the uniform distribution
  $U(\bbZ_q^n \times \bbZ_q)$.  More formally, for an adversary
  $\Adversary$, define (for the uniform-secret case)
  \begin{align*}
    \Adv{\dlwe}{n,m,q,\chi}(\Adversary) = \Big|
      &\Pr \bracks*{
      \Adversary((\bfa_{i}, b_{i})_{i=1,\ldots,m}) \Rightarrow 1 :
      \bfs \getsr U(\bbZ_q^n), (\bfa_{i}, b_{i}) \getsr
      A_{\bfs,\chi} \text{ for } i=1,\ldots, m } \\
    - &\Pr \bracks*{
      \Adversary((\bfa_{i}, b_{i})_{i=1,\ldots,m}) \Rightarrow 1 :
      (\bfa_{i}, b_{i}) \getsr U(\bbZ_{q}^{n} \times \bbZ_{q})
      \text{ for } i=1,\ldots, m } \Big| \enspace .
  \end{align*}
  %
  Similarly, define (for the normal-form case)
  $\Adv{\nfdlwe}{n,m,q,\chi}(\Adversary)$,  where
  $\bfs \getsr \chi^{n} \bmod q$ instead of
  $\bfs \getsr U(\bbZ_{q}^{n})$.
\end{definition}

For all of the above problems, when~$\chi = \Psi_{\alpha q}$ is the
continuous Gaussian of parameter~$\alpha q$, rounded to the nearest
integer (see \autoref{def:rounded-gaussian} below), we often
replace the subscript~$\chi$ by~$\alpha$.

\subsection{Gaussians}%
\label{sec:gaussians}

For any real $s > 0$, the (one-dimensional) \emph{Gaussian function}
with parameter (or width)~$s$ is the function
$\rho_{s} \colon \bbR \to \bbR^{+}$, defined as
\[ \rho_s(\bfx) := \exp(-\pi \norm{\bfx}^2/s^2) \enspace .\]

\begin{definition}[Gaussian distribution]%
  \label{def:gaussian}
  For any real $s > 0$, the (one-dimensional) \emph{Gaussian
    distribution} with parameter (or width)~$s$, denoted~$D_{s}$, is
  the distribution over~$\bbR$ having probability density function
  $D_{s}(x) = \rho_{s}(x)/s$.
\end{definition}
Note that $D_{s}$ has standard deviation $\sigma = s / \sqrt{2 \pi}$.

\begin{definition}[Rounded Gaussian distribution]%
  \label{def:rounded-gaussian}
  For any real $s > 0$, the \emph{rounded Gaussian distribution} with
  parameter (or width)~$s$, denoted $\Psi_s$, is the distribution
  over~$\bbZ$ obtained by rounding a sample from~$D_{s}$ to the
  nearest integer:
  \begin{equation*}
    \Psi_s(x) = \int_{\{z\colon \lfloor z \rceil =
      x\}} D_s(z) \, dz \enspace .
  \end{equation*}
\end{definition}

\subsection{Lattices}%
\label{sec:lattices}

Here we recall some background on lattices that will be used when
relating LWE to lattice problems.

\begin{definition}[Lattice]%
  \label{def:lattice}
  A (full-rank) \emph{$n$-dimensional lattice} $\calL$ is a discrete
  additive subset of $\bbR^n$ for which
  $\text{span}_{\bbR}(\calL) = \bbR^{n}$. Any such lattice can be
  generated by a (non-unique) \emph{basis}
  $\bfB = \set{ \bfb_1, \dots, \bfb_n } \subset \bbR^n$ of linearly
  independent vectors, as
  \[ \calL = \calL(\bfB) := \bfB \cdot \bbZ^n = \set[\Big]{ \sum_{i =
        1}^n z_i \cdot \bfb_i\colon z_i \in \bbZ } \enspace . \] The
  \emph{volume}, or \emph{determinant}, of~$\calL$ is defined as
  $\vol(\calL) := \abs{\det(\bfB)}$. An \emph{integer lattice} is a
  lattice that is a subset of~$\bbZ^{n}$. For an integer $q$, a
  \emph{$q$-ary lattice} is an integer lattice that
  contains~$q\bbZ^{n}$.
\end{definition}

\begin{definition}[Minimum distance]%
  \label{def:minimum-dist}
  For a lattice~$\calL$, its \emph{minimum distance} is the length (in
  the Euclidean norm) of a shortest non-zero lattice vector:
  \[ \lambda_1(\calL) = \min_{\bfv \in \calL \setminus \set{
        \mathbf{0} }} \length{ \bfv } \enspace . \]
  More generally, its \emph{$i$th successive minimum}
  $\lambda_i(\calL)$ is the smallest real $r > 0$ such that~$\calL$
  has~$i$ linearly independent vectors of length at most~$r$.
\end{definition}

\begin{definition}[Discrete Gaussian]%
  \label{def:discrete-gaussian}
  For a lattice $\calL \subset \bbR^{n}$, the \emph{discrete Gaussian
    distribution} over~$\calL$ with parameter~$s$, denoted
  $D_{\calL,s}$, is defined as
  $D_s(\bfx) = \rho_s(\bfx) / \rho_s(\calL)$ for $\bfx \in \calL$ (and
  $D_{s}(\bfx)=0$ otherwise), where
  $\rho_s(\calL) = \sum_{\bfv \in \calL}\rho_s(\bfv)$ is a
  normalization factor.
\end{definition}

We now recall various computational problems on lattices. We stress
that these are \emph{worst-case} problems, i.e., to solve such a
problem an algorithm must succeed on \emph{every} input (and not just on a randomly chosen input from some probability distribution). The following
two problems are parameterized by an \emph{approximation factor}
$\gamma = \gamma(n)$, which is a function of the lattice dimension $n$.

\begin{definition}[Decisional approximate shortest vector problem
  ($\GapSVP_\gamma$)]%
  \label{def:GapSVP}
  Given a basis $\bfB$ of an $n$-dimensional lattice
  $\calL = \calL(\bfB)$, where $\lambda_1(\calL) \leq 1$ or
  $\lambda_1(\calL) > \gamma(n)$, determine which is the case.
\end{definition}

\begin{definition}[Approximate shortest independent vectors problem
  ($\SIVP_\gamma$)]%
  \label{def:SIVP}
  Given a basis $\bfB$ of an $n$-dimensional lattice
  $\calL = \calL(\bfB)$, output a set
  $\set{ \bfv_{1}, \ldots, \bfv_{n} } \subset \calL$ of~$n$ linearly
  independent lattice vectors where
  $\length{ \bfv_i } \leq \gamma(n) \cdot \lambda_n(\calL)$ for all~$i$.
\end{definition}

The following problem is parameterized by a function~$\varphi$ from
lattices to positive real numbers.

\begin{definition}[Discrete Gaussian Sampling ($\DGS_{\varphi}$)]%
  \label{def:DGS}
  Given a basis~$\bfB$ of an $n$-dimensional lattice $\calL =
  \calL(\bfB)$ and a real number $s \geq \varphi(\calL)$, output a sample
  from the discrete Gaussian distribution $D_{\calL,s}$.
\end{definition}

%%% Local Variables:
%%% mode: latex
%%% TeX-master: "Main"
%%% End:


\section{Auxiliary algorithms}%
\label{sec:algs}

This section describes the auxiliary algorithms used in $\FrodoPKE$ and $\FrodoKEM$.

%\lewis{Renaming this section to "Auxiliary algorithms" from "Algorithm description." Previously the beginning of this section read "This section specifies the algorithms comprising the $\FrodoKEM$ key encapsulation mechanism.  $\FrodoKEM$ is built from a public-key encryption scheme, $\FrodoPKE$, as well as several other components." }

\paragraph{Notation.}
In this work, the algorithms are described in terms of the following parameters:
\begin{itemize}
\item $\chi$, a probability distribution on $\bbZ$, and $T_\chi$, the corresponding distribution table for sampling;
\item $q=2^D$, a power-of-two integer modulus with exponent $D\leq 16$;
\item $n,\mbar,\nbar$, integer matrix dimensions with $n \equiv 0 \pmod 8$;
\item $B\leq D$, the number of bits encoded in each matrix entry;
\item $\ell=B\cdot \mbar\cdot\nbar$, the length of bit strings that are encoded as $\mbar$-by-$\nbar$ matrices;
\item $\lengthseedA$, the bit length of seeds used for pseudorandom matrix generation;
\item $\lengthseedSE$, the bit length of seeds used for pseudorandom bit generation for error sampling.
\end{itemize}

%Additional parameters for specific algorithms accompany the algorithm description.

\subsection{Matrix encoding of bit strings}
\label{sec:matrix-encoding}

This subsection describes how bit strings are encoded as mod-$q$ integer matrices.
Recall that $2^B \leq q$. The encoding function $\encode(\cdot)$ encodes an integer $0 \leq k < 2^B$ as an element in $\bbZ_q$ by multiplying it by $q/2^B=2^{D-B}$:
\[ \encode(k) := k\cdot  q/2^B \enspace . \]
This encoding function can be found in early works on LWE-based encryption, for example \cite{PKC:KawTanXag07,STOC:PeiWat08,C:PeiVaiWat08}.
Using this function, the function $\Frodo.\Encode$ encodes bit strings of
length $\ell=B\cdot \mbar\cdot\nbar$ as  $\mbar$-by-$\nbar$-matrices with
entries in $\bbZ_q$ by applying $\encode(\cdot)$ to $B$-bit sub-strings sequentially and filling the matrix row by row entry-wise. The function $\Frodo.\Encode$ is shown in \autoref{alg:encode}.
Each $B$-bit sub-string is interpreted as an integer $0 \leq k < 2^B$ and then encoded by $\encode(k)$, which means that $B$-bit values are placed into the $B$ most significant bits of the corresponding entry modulo $q$.  

The corresponding decoding function $\Frodo.\Decode$ is defined as shown in
\autoref{alg:decode}. It decodes the $\mbar$-by-$\nbar$ matrix $\bfK$ into a
bit string of length $\ell=B\cdot\mbar\cdot\nbar$. It extracts $B$ bits from
each entry by applying the function $\decode(\cdot)$:
\[ \decode(c) = \round{c \cdot 2^B/q} \bmod 2^B \enspace . \]
That is, the $\bbZ_{q}$-entry is interpreted as an integer, then
divided by $q/2^B$ and rounded. This amounts to rounding to the $B$
most significant bits of each entry. With these definitions, it is the case that $\decode(\encode(k)) = k$ for all $0\leq k < 2^B$.

\begin{figure}[h!]
\centering
\begin{minipage}[t]{0.48\textwidth}
\begin{algorithm}[H]
\caption{\label{alg:encode} $\Frodo.\Encode$}
% Ananth: I use phantom to make sure that the two Algorithm margins align in
% the final multicol
{\bf Input:} Bit string $\bfk\in \{0,1\}^\ell$, $\ell = B\cdot\mbar\cdot\nbar$.\phantom{$\bbZ_q^{\mbar\times\nbar}$}\\
{\bf Output:} Matrix $\bfK \in \bbZ_q^{\mbar \times \nbar}$.\\[-1.5ex]
\rule{\linewidth}{.5pt}
\vspace{-0.5cm}
\begin{algorithmic}[1]
    \FOR{($i = 0$; $i < \mbar$;  $i\gets i + 1$)}   
    \FOR{($j = 0$; $j < \nbar$;  $j\gets j+1$)}
    \STATE $k \gets \sum_{l=0}^{B-1}\bfk_{(i\cdot \nbar + j)B+l}\cdot 2^l$
    \STATE $\bfK_{i,j} \gets \encode(k)= k\cdot q/2^B$
    \ENDFOR
    \ENDFOR
    \RETURN$\bfK = (\bfK_{i,j})_{0\leq i < \mbar, 0 \leq j < \nbar}$
\end{algorithmic}
\end{algorithm}
\end{minipage}
~
\begin{minipage}[t]{0.49\textwidth}
\begin{algorithm}[H]
\caption{\label{alg:decode} $\Frodo.\Decode$}
{\bf Input:} Matrix $\bfK \in \bbZ_q^{\mbar\times\nbar}$.\\
{\bf Output:} Bit string $\bfk \in \{0,1\}^\ell$, $\ell = B\cdot\mbar\cdot\nbar$.\\[-1.5ex]
\rule{\linewidth}{.5pt}
\vspace{-0.5cm}
\begin{algorithmic}[1]
    \FOR{($i = 0$; $i < \mbar$;  $i\gets i+1$)}   
    \FOR{($j = 0$; $j < \nbar$;  $j\gets j+1$)}
    \STATE $k \gets \decode(\bfK_{i,j})= \round{\bfK_{i,j}\cdot 2^B/q} \bmod 2^B$
    \STATE $k = \sum_{l=0}^{B-1}k_l \cdot 2^l$ \text{ where } $k_l\in \{0,1\}$
    \FOR{($l = 0$; $l < B$; $l\gets l+1$)}
    \STATE $\bfk_{(i\cdot \nbar + j)B+l} \gets k_l$ 
    \ENDFOR
    \ENDFOR
    \ENDFOR
    \RETURN$\bfk$
\end{algorithmic}
\end{algorithm}
\vspace{-0.5em}
\textit{Note to implementers: recall $\round{x} = \lfloor x + \frac{1}{2} \rfloor$.}
\end{minipage}
\end{figure}

\subsection{Packing matrices modulo $q$}
\label{sec:pack}

This section specifies packing and unpacking algorithms to transform
matrices with entries in $\bbZ_q$ to bit strings and vice versa. The
algorithm $\Frodo.\Pack$ packs a matrix into a bit string by simply
concatenating the $D$-bit matrix coefficients, as shown in
\autoref{alg:frodopack}. Note that in the software implementation, the
resulting bit string is stored as a byte array, padding with zero bits
to make the length a multiple of~$8$. The reverse operation
$\Frodo.\Unpack$ is shown in \autoref{alg:frodounpack}.

\begin{figure}[h!]
\centering
\begin{minipage}[t]{0.45\textwidth}
\begin{algorithm}[H]
\caption{\label{alg:frodopack} $\Frodo.\Pack$}
{\bf Input:} Matrix $\bfC \in \bbZ_q^{n_1\times n_2}$.\\
{\bf Output:} Bit string $\bfb \in \{0,1\}^{D\cdot n_1\cdot n_2}$.\phantom{$\bbZ_q^{\mbar\times\nbar}$}\\[-1.5ex]
\rule{\linewidth}{.5pt}
\vspace{-0.5cm}
\begin{algorithmic}[1]
    \FOR{($i = 0$; $i < n_1$;  $i\gets i+1$)}   
    \FOR{($j = 0$; $j < n_2$;  $j\gets j+1$)}
    \STATE $\bfC_{i,j} = \sum_{l=0}^{D-1}c_l \cdot 2^l$ \text{ where } $c_l\in \{0,1\}$
    \FOR{($l = 0$; $l < D$; $l\gets l+1$)}
    \STATE $\bfb_{(i\cdot n_2  + j)D+l} \gets c_{D-1-l}$ 
    \ENDFOR
    \ENDFOR
    \ENDFOR
    %\FOR{($i = 0$; $i < D\cdot n_1\cdot n_2/8$;  $i\gets i+1$)}
    %\STATE $b_\bfC[i] \gets \sum_{l=0}^{7} \bfb_{8i + l} \cdot 2^{7-l}$
    %\ENDFOR
    \RETURN $\bfb$
\end{algorithmic}
\end{algorithm}
\end{minipage}
~
\begin{minipage}[t]{0.5\textwidth}
\begin{algorithm}[H]
\caption{\label{alg:frodounpack} $\Frodo.\Unpack$}
{\bf Input:}  Bit string $\bfb \in \{0,1\}^{D\cdot n_1\cdot n_2}$, $n_1$, $n_2$.\phantom{$\bbZ_q^{\mbar\times\nbar}$}\\
{\bf Output:} Matrix $\bfC \in \bbZ_q^{n_1\times n_2}$.\\[-1.5ex]
\rule{\linewidth}{.5pt}
\vspace{-0.5cm}
\begin{algorithmic}[1]
    %\FOR{($i = 0$; $i < D\cdot n_1\cdot n_2/8$;  $i\gets i+1$)}
    %\STATE $b_\bfC[i] = \sum_{l=0}^{7} c_l 2^l$, $c_l\in \{0,1\}$
    %\FOR{($l = 0$; $l < 8$; $l\gets l+1$)}
    %\STATE $\bfb_{8i+l} \gets c_{7-l}$
    %\ENDFOR
    %\ENDFOR
    \FOR{($i = 0$; $i < n_1$;  $i\gets i+1$)}   
    \FOR{($j = 0$; $j < n_2$;  $j\gets j+1$)}
    \STATE $\bfC_{i,j} \gets \sum_{l=0}^{D-1}\bfb_{(i\cdot n_2  + j)D+l}\cdot 2^{D-1-l}$
    \ENDFOR
    \ENDFOR
    \RETURN $\bfC$
\end{algorithmic}
\end{algorithm}
\end{minipage}
\end{figure}

\subsection{Deterministic random bit generation}\label{sec:rbg}

$\FrodoKEM$ requires the deterministic generation of random bit
sequences from a random seed value. This is done using the
SHA-3-derived extendable output function
$\SHAKE$~\cite{dworkin2015sha}. The function $\SHAKE$ is taken as
either $\SHAKE128$ or $\SHAKE256$ (indicated below for each
parameter set of $\FrodoKEM$), and takes as input a bit string $X$ and a
requested output bit length $L$.

\subsection{Sampling from the error distribution}\label{sec:sampling}

The error distribution~$\chi$ used in $\FrodoKEM$ is a discrete,
symmetric distribution on $\bbZ$, centered at zero and with small
support, which approximates a rounded continuous Gaussian
distribution.

The support of~$\chi$ is
$S_{\chi}=\set{-s, -s+1, \dots, -1, 0, 1, \dots, s-1, s}$ for a
positive integer $s$. The probabilities $\chi(z) = \chi(-z)$ for
$z \in S_{\chi}$ are given by a discrete probability density function,
which is described by a table
\[ T_{\chi} = (T_{\chi}(0), T_{\chi}(1), \dots, T_{\chi}(s)) \] of
$s+1$ positive integers related to the cumulative distribution function.  For a
certain positive integer $\lengthchi$, the table entries satisfy the
following conditions:
\[ T_{\chi}(0) = 2^{\lengthchi-1}\cdot \chi(0)-1 \quad\quad \textrm{and} \quad\quad
  T_{\chi}(z) = T_{\chi}(0) + 2^{\lengthchi}\sum_{i=1}^z\chi(i) \enspace \textrm{ for
  } 1 \leq z \leq s. \]

Since the distribution $\chi$ is symmetric and centered at zero, it is easy to verify that
$T_{\chi}(s) = 2^{\lengthchi-1}-1$.

Sampling from~$\chi$ via inversion sampling is done as shown in
\autoref{alg:samplechi}.  Given a string of~$\lengthchi$ uniformly
random bits~$\bfr \in \bit^{\lengthchi}$ and a distribution table
$T_{\chi}$, the algorithm $\Frodo.\Samp$ returns a sample~$e$ from the
distribution $\chi$. (Note that $T_\chi(s)$ is never accessed.)
We emphasize that it is important to perform
this sampling in constant time to avoid exposing timing side-channels,
which is why Step~\ref{alg:sample:loop} of the algorithm does a complete loop through the
entire table~$T_{\chi}$. The comparison in Step~\ref{alg:sample:cmp} needs
to be implemented in a constant-time manner.

\begin{algorithm}[H]
\caption{\label{alg:samplechi} $\Frodo.\Samp$}
{\bf Input:} A (random) bit string $\bfr = (\bfr_0, \bfr_1, \dots,
\bfr_{\lengthchi-1}) \in \bit^{\lengthchi}$, the table $T_{\chi} = (T_{\chi}(0), T_{\chi}(1), \dots, T_{\chi}(s))$.\\
{\bf Output:} A sample $e \in \bbZ$.\\[-1.5ex]
\rule{\linewidth}{.5pt}
\vspace{-0.5cm}
\begin{algorithmic}[1]
    \STATE $t \gets \sum_{i=1}^{\lengthchi-1} \bfr_i\cdot 2^{i-1}$
    \STATE $e\gets 0$
    \FOR{($z = 0$; $z < s$;  $z\gets z+1$)}\label{alg:sample:loop}
    \IF{$t > T_\chi(z)$}\label{alg:sample:cmp}
    \STATE $e\gets e+1$
    \ENDIF
    \ENDFOR
    \STATE $e\gets (-1)^{\bfr_0}\cdot e$
    \RETURN$e$
\end{algorithmic}
\end{algorithm}

An $n_1$-by-$n_2$ matrix of~$n_1n_2$ samples from the error
distribution is sampled on input of a 
$(n_1n_2 \cdot \lengthchi)$-bit string, here written as a sequence 
$(\bfr^{(0)}, \bfr^{(1)}, \dots, \bfr^{(n_1n_2-1)})$ of
$n_1n_2$ bit vectors of length
$\lengthchi$ each,
by sampling~$n_1n_2$ error terms through calls to $\Frodo.\Samp$ on a
corresponding $\lengthchi$-bit substring~$\bfr^{(i\cdot n_2 + j)}$ and
the distribution table $T_\chi$ to sample the matrix entry
$\bfE_{i,j}$. The algorithm $\Frodo.\SampV$ is shown in
\autoref{alg:samplevecchi}.

\begin{algorithm}[H]
\caption{\label{alg:samplevecchi} $\Frodo.\SampV$}
{\bf Input:} A (random) bit string $(\bfr^{(0)}, \bfr^{(1)}, \dots, \bfr^{(n_1n_2-1)}) \in \{0,1\}^{n_1n_2 \cdot \lengthchi}$ (here, each $\bfr^{(i)}$ is a vector of $\lengthchi$ bits), the
  table $T_{\chi}$.\\
  % = (T_{\chi}(0), T_{\chi}(1), \dots, T_{\chi}(s))$
{\bf Output:} A sample $\bfE \in \bbZ^{n_1\times n_2}$.\\[-1.5ex]
\rule{\linewidth}{.5pt}
\vspace{-0.5cm}
\begin{algorithmic}[1]
    \FOR{($i = 0$; $i < n_1$;  $i\gets i+1$)}
    \FOR{($j = 0$; $j < n_2$;  $j\gets j+1$)}
    \STATE $\bfE_{i,j} \gets \Frodo.\Samp(\bfr^{(i\cdot n_2 + j)}, T_\chi)$
    \ENDFOR
    \ENDFOR
    \RETURN$\bfE$
\end{algorithmic}
\end{algorithm}

\subsection{Pseudorandom matrix generation}
\label{sec:genA}

The algorithm $\Frodo.\gen$ takes as input a seed
$\seedA\in\{0,1\}^\lengthseedA$ and an implicit dimension $n\in\bbZ$, and outputs
a pseudorandom matrix $\bfA\in \bbZ_q^{n\times n}$. There are two
options for instantiating $\Frodo.\gen$. The first one uses
$\AESOneTwoEight$ and is shown in \autoref{alg:genA_AES}; the second
uses $\SHAKE128$ and is shown in \autoref{alg:genA_SHAKE}.

\paragraph{Using $\AESOneTwoEight$.}

\autoref{alg:genA_AES} generates a matrix
$\bfA \in \bbZ_{q}^{n\times n}$ as follows.  For each row
index $i=0,1,\ldots,n-1$ and column index $j=0,8,\ldots,n-8$, the
algorithm generates a 128-bit block, which it uses to set the matrix
entries $\bfA_{i,j}, \bfA_{i,j+1}, \ldots, \bfA_{i,j+7}$ as follows.  It
applies $\AESOneTwoEight$ with key $\seedA$ to the input block
$\inner{i} \| \inner{j} \| 0 \cdots 0 \in \bit^{128}$, where $i,j$ are
encoded as 16-bit strings, represented in little-endian byte order. It then splits the 128-bit AES output block
into eight 16-bit strings, which it interprets as nonnegative
integers~$c_{i,j+k}$ for $k=0,1,\ldots,7$ in little-endian byte order.  Finally, it sets
$\bfA_{i,j+k}= c_{i,j+k} \bmod q$ for all~$k$.

\paragraph{Using $\SHAKE128$.}

\autoref{alg:genA_SHAKE} generates a matrix
$\bfA \in \bbZ_{q}^{n \times n}$ as follows. For each row index
$i=0,1,\ldots,n-1$, it calls $\SHAKE128$ with a main input
of~$\seedA$, prefixed with a counter (represented as a 16-bit integer in little-endian byte order), to produce a
$16n$-bit output string. It splits this output into 16-bit integers (in little-endian byte order)
$c_{i,j}$ for $j=0,1,\ldots, n-1$, and sets
$\bfA_{i,j} = c_{i,j} \bmod q$ for all~$j$.

\begin{algorithm}[]
\caption{\label{alg:genA_AES} $\Frodo.\gen$ using $\AESOneTwoEight$}
{\bf Input:} Seed $\seedA \in \{0,1\}^\lengthseedA$.\\
{\bf Output:} Matrix $\bfA \in \bbZ_q^{n\times n}$.\\[-1.5ex]
\rule{\linewidth}{.5pt}
\vspace{-0.5cm}
\begin{algorithmic}[1]
    \FOR{($i = 0$; $i < n$;  $i\gets i+1$)}
    \FOR{($j = 0$; $j < n$;  $j\gets j+8$)}
    \STATE $\bfb \gets \inner{i} \| \inner{j} \|0 \cdots 0 \in
    \bit^{128}$ where $\inner{i}, \inner{j} \in \bit^{16}$
    \STATE $\inner{c_{i,j}} \| \inner{c_{i,j+1}} \| \cdots \| \inner{c_{i,j+7}} \gets
    \AESOneTwoEight_{\seedA}(\bfb)$ where each $\inner{c_{i,k}} \in \bit^{16}$
    \FOR{($k=0$; $k<8$; $k\gets k+1$)}
    \STATE $\bfA_{i,j+k} \gets c_{i,j+k} \bmod q$
    \ENDFOR
    \ENDFOR
    \ENDFOR
    \RETURN $\bfA$
\end{algorithmic}
\end{algorithm}

\begin{algorithm}[]
\caption{\label{alg:genA_SHAKE} $\Frodo.\gen$ using $\SHAKE128$}
{\bf Input:} Seed $\seedA \in \{0,1\}^\lengthseedA$.\\
{\bf Output:} Pseudorandom matrix $\bfA \in \bbZ_q^{n\times n}$.\\[-1.5ex]
\rule{\linewidth}{.5pt}
\vspace{-0.5cm}
\begin{algorithmic}[1]
    \FOR{($i = 0$; $i < n$;  $i\gets i+1$)}
    \STATE $\bfb \gets \inner{i} \| \seedA \in
    \bit^{16+\lengthseedA}$ where $\inner{i} \in \bit^{16}$
    \STATE $\inner{c_{i,0}} \| \inner{c_{i,1}} \| \cdots \|
    \inner{c_{i,n-1}} \gets \SHAKE128(\bfb, 16n)$
    where each $\inner{c_{i,j}} \in \bit^{16}$
    \FOR{($j = 0$; $j < n$;  $j\gets j+1$)}
    \STATE $\bfA_{i,j} \gets c_{i,j} \bmod q$
    \ENDFOR
    \ENDFOR
    \RETURN$\bfA$
\end{algorithmic}
\end{algorithm}


%\paragraph{Using other functions.}
%In principle, other functions could be used to pseudorandomly generate the matrix $\bfA$, such as a lightweight stream cipher for platforms without the hardware instructions that make fast $\AES$ and $\SHAKE$ implementations possible.  As NIST currently does not standardize such a primitive, and the call for proposals indicated that submissions should use NIST primitives, we do not currently propose such an alternate instantiation.








%%% Local Variables:
%%% mode: latex
%%% TeX-master: "Main"
%%% End:


\section{$\FrodoPKE$: \INDCPA-secure public-key encryption}%
\label{sec:cpa-pke}

This section describes $\FrodoPKE$, a public-key encryption scheme
with fixed-length message space, targeting \INDCPA security, that will
be used as a building block for $\FrodoKEM$. $\FrodoPKE$ is based on
the public-key encryption scheme by Lindner and Peikert~\cite{RSA:LinPei11}.

%with the following adaptations and
%specializations:
%\begin{itemize}
%\item The matrix $\bfA$ is generated from a seed using the function $\Frodo.\gen$ specified in \autoref{sec:genA}.
%\item Several ($\mbar$) ciphertexts are generated at once.
%\item The same Gaussian-derived error distribution is used for both
%  key generation and encryption.
%\end{itemize}

The PKE scheme is given by three algorithms
$(\FrodoPKE.\KeyGen, \FrodoPKE.\Enc,$ $\FrodoPKE.\Dec)$, defined
respectively in \autoref{alg:PKE:KeyGen}, \autoref{alg:PKE:Enc}, and
\autoref{alg:PKE:Dec}.  
$\FrodoPKE$ is parameterized by the parameters defined in \autoref{sec:algs}.
Additional parameters include the bit length of messages $\lengthm = \ell$, the message space $\MsgSp = \{0,1\}^\lengthm$,
and the matrix-generation algorithm $\Frodo.\gen$ (either \autoref{alg:genA_AES} or \autoref{alg:genA_SHAKE}).
%
%$\FrodoPKE$ is parameterized by the following parameters:
%\begin{itemize}
%\item $q=2^D$, a power-of-two integer modulus with exponent $D \leq 16$;
%\item $n, \mbar, \nbar$, integer matrix dimensions with $n \equiv 0 \pmod 8$;
%\item $B \leq D$, the number of bits encoded in each matrix entry;
%\item $\ell=B\cdot \mbar\cdot\nbar$, the length of bit strings that are encoded as $\mbar$-by-$\nbar$ matrices;
%\item $\lengthm = \ell$, the bit length of messages;
%\item $\MsgSp = \{0,1\}^\lengthm$, the message space;
%\item $\lengthseedA$, the bit length of seeds used for pseudorandom matrix generation;
%\item $\lengthseedSE$, the bit length of seeds used for pseudorandom bit generation for error sampling;
%\item $\Frodo.\gen$, the matrix-generation algorithm, either \autoref{alg:genA_AES} or \autoref{alg:genA_SHAKE};
%\item $T_\chi$, the distribution table for sampling.
%\end{itemize}
%
In the notation of~\cite{RSA:LinPei11}, their~$n_1$ and~$n_2$ both
equal~$n$ here, and their dimension~$\ell$ is~$\nbar$ here.


\begin{algorithm}[t]
\caption{\label{alg:PKE:KeyGen} $\FrodoPKE.\KeyGen$.}
{\bf Input:} None.\\
{\bf Output:} Key pair $(\pk, \sk) \in (\{0,1\}^\lengthseedA \times \mathbb{Z}_q^{n \times \nbar})\times \mathbb{Z}_q^{\nbar \times n}$.\\[-1.5ex]
\rule{\linewidth}{.5pt}
\vspace{-0.5cm}
\begin{algorithmic}[1]
    \STATE Choose a uniformly random seed $\seedA \getsr U(\{0,1\}^\lengthseedA)$
    \STATE Generate the matrix $\bfA \in \bbZ_q^{n \times n}$ via $\bfA\gets\Frodo.\gen(\seedA)$
    \STATE Choose a uniformly random seed $\seedSE \getsr U(\{0,1\}^\lengthseedSE)$
    \STATE Generate pseudorandom bit string $ (\bfr^{(0)}, \bfr^{(1)}, \dots, \bfr^{(2n\nbar-1)}) \gets \SHAKE(\mathtt{0x5F} \| \seedSE,$ $2n\nbar \cdot \lengthchi)$
    \STATE Sample error matrix $\bfS^\text{T} \gets \Frodo.\SampV((\bfr^{(0)}, \bfr^{(1)}, \dots, \bfr^{(n\nbar-1)}), \nbar, n, T_\chi)$
    \STATE Sample error matrix $\bfE \gets \Frodo.\SampV((\bfr^{(n\nbar)}, \bfr^{(n\nbar+1)}, \dots, \bfr^{(2n\nbar-1)}), n, \nbar, T_\chi)$
    \STATE Compute $\bfB = \bfA\bfS + \bfE$
    \RETURN public key $\pk \gets (\seedA, \bfB)$ and secret key $\sk \gets \bfS^\text{T}$
\end{algorithmic}
\end{algorithm}

\begin{algorithm}[]
\caption{\label{alg:PKE:Enc} $\FrodoPKE.\Enc$.}
{\bf Input:} Message $\mu \in \calM$ and public key $\pk = (\seedA, \bfB) \in \{0,1\}^\lengthseedA \times \bbZ_q^{n \times \nbar}$.\\
{\bf Output:} Ciphertext $c = (\bfC_1, \bfC_2) \in \bbZ_q^{\mbar\times n}\times \bbZ_q^{\mbar\times\nbar}$.\\[-1.5ex]
\rule{\linewidth}{.5pt}
\vspace{-0.5cm}
\begin{algorithmic}[1]
    \STATE Generate $\bfA\gets \Frodo.\gen(\seedA)$
    \STATE Choose a uniformly random seed $\seedSE \getsr U(\{0,1\}^\lengthseedSE)$
     \STATE Generate pseudorandom bit string $ (\bfr^{(0)}, \bfr^{(1)}, \dots, \bfr^{(2\mbar n+\mbar\nbar-1)}) \gets \SHAKE(\mathtt{0x96} \|$ $\seedSE, (2\mbar n+\mbar\nbar) \cdot \lengthchi)$
    \STATE Sample error matrix $\bfS' \gets \Frodo.\SampV((\bfr^{(0)}, \bfr^{(1)}, \dots, \bfr^{(\mbar n-1)}), \mbar, n, T_\chi)$
    \STATE Sample error matrix $\bfE' \gets \Frodo.\SampV((\bfr^{(\mbar n)}, \bfr^{(\mbar n+1)}, \dots, \bfr^{(2\mbar n-1)}), \mbar, n,$ $T_\chi)$
    \STATE Sample error matrix $\bfE'' \gets \Frodo.\SampV((\bfr^{(2\mbar n)}, \bfr^{(2\mbar n+1)}, \dots, \bfr^{(2\mbar n+\mbar\nbar-1)}),$ $\mbar, \nbar, T_\chi)$
    \STATE Compute $\bfB' = \bfS'\bfA + \bfE'$ and $\bfV = \bfS'\bfB + \bfE''$
    \RETURN ciphertext $c \gets (\bfC_1, \bfC_2)=(\bfB', \bfV + \Frodo.\Encode(\mu))$
\end{algorithmic}
\end{algorithm}

\begin{algorithm}[t]
\caption{\label{alg:PKE:Dec} $\FrodoPKE.\Dec$.}
{\bf Input:} Ciphertext $c=(\bfC_1, \bfC_2)  \in \bbZ_q^{\mbar\times n}\times \bbZ_q^{\mbar\times\nbar}$ and secret key $\sk = \bfS^\text{T} \in \mathbb{Z}_q^{\nbar \times n}$.\\
{\bf Output:} Decrypted message $\mu' \in \calM$.\\[-1.5ex]
\rule{\linewidth}{.5pt}
\vspace{-0.5cm}
\begin{algorithmic}[1]
    \STATE Compute $\bfM = \bfC_2 - \bfC_1\bfS$
    \RETURN message $\mu' \gets \Frodo.\Decode(\bfM)$
\end{algorithmic}
\end{algorithm}
   

\subsection{Correctness of \INDCPA PKE}\label{sec:cpa-pke-correctness}

The next lemma states bounds on the size of errors that can be handled
by the decoding algorithm.

\begin{lemma}\label{lem:CorrectnessDec}
  Let $q = 2^D$, $B \leq D$. Then $\decode(\encode(k)+e) = k$ for any
  $k,e\in \bbZ$ such that $0\leq k <2^B$ and
  $-q/2^{B+1} \leq e < q/2^{B+1}$.
\end{lemma}

\begin{proof}
  This follows directly from the fact that
  $\decode(\encode(k)+e) = \lfloor k + e2^B/q \rceil \bmod 2^B$.
\end{proof}

\paragraph{Correctness of decryption:}
The decryption algorithm $\FrodoPKE.\Dec$ computes 
\begin{align*}
\bfM 
&= \bfC_2 - \bfC_1\bfS \\
&= \bfV + \Frodo.\Encode(\mu) - (\bfS'\bfA + \bfE')\bfS \\
&= \Frodo.\Encode(\mu) + \bfS'\bfB + \bfE'' - \bfS'\bfA\bfS - \bfE'\bfS \\
&= \Frodo.\Encode(\mu) + \bfS'\bfA\bfS + \bfS'\bfE +  \bfE'' - \bfS'\bfA\bfS - \bfE'\bfS \\
&= \Frodo.\Encode(\mu) + \bfS'\bfE + \bfE''- \bfE'\bfS \\
&= \Frodo.\Encode(\mu) + \bfE'''
\end{align*}
for some error matrix $\bfE''' = \bfS' \bfE + \bfE'' - \bfE'
\bfS$. Therefore, any $B$-bit substring of the message $\mu$
corresponding to an entry of $\bfM$ will be decrypted correctly if the
condition in \autoref{lem:CorrectnessDec} is satisfied for the
corresponding entry of $\bfE'''$.

\paragraph{Failure probability.}

Each entry in the matrix $\bfE'''$ is the sum of~$2n$ products of two
independent samples from $\chi$, and one more independent sample from
$\chi$. Denote the distribution of this sum by $\chi'$. In the case of
a power-of-$2$ modulus $q$, the probability of decryption failure for
any single symbol is therefore the sum
\[ p = \sum_{e \notin [-q/2^{B+1},q/2^{B+1})}\chi'(e) \enspace. \]
The probability of decryption failure for the entire message can then
be obtained using the union bound.

For the distributions~$\chi$ we use, which have rather small
support~$S_{\chi}$, the distribution $\chi'$ can be efficiently
computed exactly. The probability that a product of two independent
samples from~$\chi$ equals~$e$ (modulo~$q$) is simply
\[ \sum_{(a,b)\in S_\chi\times S_\chi\; :\; a b = e \bmod q}
  \chi(a)\cdot \chi(b) \enspace. \] Similarly, the probability that
the sum of two entries assumes a certain value is given by the
standard convolution sum.  \autoref{sec:params:sets} reports the
failure probability for each of the selected parameter sets.

\subsection{Transform from $\MINDCPA$ PKE to $\MINDCCA$ KEM}\label{sec:cca-transform}

The Fujisaki--Okamoto transform \cite{C:FujOka99} constructs an \INDCCATwo-secure public-key encryption scheme, in the classical random oracle model, from a one-way-secure public-key encryption scheme (assuming the distribution of ciphertexts for each plaintext is sufficiently ``well spread'').
Targhi and Unruh \cite{TCC:TarUnr16} gave a variant of the Fujisaki--Okamoto transform and showed its \INDCCATwo security against a quantum adversary in the quantum random oracle model under similar assumptions.
The results of both \cite{C:FujOka99} and \cite{TCC:TarUnr16} proceed under the assumption that the public-key encryption scheme has perfect correctness, which is often not the case for lattice-based schemes (including ours).
Hofheinz, H{\" o}velmanns, and Kiltz (HHK)~\cite{TCC:HofHovKil17} gave a variety of constructions, in a modular fashion, that in particular allow for a small probability of incorrect decryption.

\cpnote{We need to be clearer about where SFO was defined, and what we've modified.}
In this work, we propose a variant called Modified Salted Fujisaki--Okamoto with implicit rejection ($\SFOnotperpprime$) transform, which concatenates a uniformly random, public salt of bit length $\ell = \lengthsalt$. The salt value is made
public as part of the ciphertext output by encapsulation. The security of the resulting KEM against multi-challenge attacks in the ROM was proven in a Master's thesis by Glabush~\cite{GlabushThesis}. In that same work, a QROM bound is proposed, with an accompanying proof sketch. A full proof was given in private communication~\cite{Multi-challenge}. Without the inclusion of a salt, with $n$ challenge ciphertexts, and $N$ ROM queries, an adversary can break $\MINDCCA$ security with probability $\frac{nN}{|\M|}$. With the inclusion of the salt, this advantage shrinks to $\frac{nN}{|\M||\saltlist|}$, where $|\saltlist|$ is the number of distinct salt values sampled by the challenge oracle.

We apply the $\SFOnotperpprime$ (``Modified Salted FO with implicit
rejection'') transform, which constructs an $\MINDCCA$-secure key
encapsulation mechanism from an $\MINDCPA$ public-key encryption scheme
and three hash functions; following~\cite{EuroSP:Kyber}, we make the
following modifications (see \autoref{fig:kem-qfo} for notation),
denoting the resulting transform $\SFOnotperpprime$:\plnote{It's not clear where these modifications are applied to to get the $\SFOnotperpprime$ transform.}
\begin{itemize}
\item A single hash function (with longer output) is used to compute $\bfr$ and $\bfk$.
\item The computation of $\bfr$ and $\bfk$ also takes the public key $\pk$ as input.
% Lewis says this is not a change:
% \item The computation of the shared secret $\ssk$ also takes the encapsulation $c$ as input.
\end{itemize}

\begin{definition}[$\SFOnotperpprime$ transform]
  \label{def:qfo}
  Let $\PKE=(\KeyGen,\Enc,\Dec)$ be a public-key encryption scheme
  with message space $\MsgSp$ and ciphertext space $\CtxtSp$, where
  the randomness space of $\Enc$ is $\RandSp$.  Let
  $\lengthsalt, \lengths, \lengthK, \lengthpkhash, \lengthss$ be parameters.  Let
  $G_1\colon \{0,1\}^* \to \{0,1\}^\lengthpkhash$,
  $G_2\colon \{0,1\}^* \to \RandSp \times \{0,1\}^\lengthK$, and
  $F\colon \{0,1\}^* \to \{0,1\}^\lengthss$ be hash functions.  Define
  $\KEMnotperpprime = \SFOnotperpprime[\PKE,G_1,G_2,F]$ to be the key
  encapsulation mechanism as shown in \autoref{fig:kem-qfo}.
\end{definition}

As observed by Guo, Johansson, and Nilsson \cite{C:GuoJohNil20}, a timing side-channel enables key recovery if \cpnote{use label and ref here:} step 5 of $\KEMnotperpprime.\Decaps$ is not performed in constant time.

\begin{figure}[h]
\centering
\fbox{
\begin{minipage}[t]{0.4\textwidth}
\underline{$\KEMnotperpprime.\KeyGen()$:}
\vspace{-1em}
\begin{algorithmic}[1]
\STATE $(\pk, \sk) \getsr \PKE.\KeyGen()$
\STATE $\bfs \getsr \{0,1\}^\lengths$
\STATE $\pkh \gets G_1(\pk)$
\STATE $\sk' \gets (\sk, \bfs, \pk, \pkh)$
\RETURN $(\pk, \sk')$
\end{algorithmic}

\medskip

\underline{$\KEMnotperpprime.\Encaps(\pk)$:}
\vspace{-1em}
\begin{algorithmic}[1]
\STATE $\mu \getsr \MsgSp$, {\color{black} $\salt \getsr \{0,1\}^{\lengthsalt}$}
\STATE $(\bfr, \bfk) \gets G_2(G_1(\pk) \| \mu {\color{black} \| \salt})$
\STATE $c \gets \PKE.\Enc(\mu, \pk; \bfr)$
\STATE $\ssk \gets F(c {\color{black} \| \salt} \| \bfk)$
\RETURN $(c {\color{black} \| \salt}, \ssk)$
\end{algorithmic}
\end{minipage}
~
\begin{minipage}[t]{0.5\textwidth}
\underline{$\KEMnotperpprime.\Decaps(c {\color{black} \| \salt}, (\sk, \bfs, \pk, \pkh))$:}
%\vspace{-1em}
\begin{algorithmic}[1]
\STATE $\mu' \gets \PKE.\Dec(c, \sk)$
\STATE $(\bfr', \bfk') \gets G_2(\pkh \| \mu' {\color{black} \| \salt})$
\STATE $\ssk_0' \gets F(c {\color{black} \| \salt} \| \bfk')$
\STATE $\ssk_1' \gets F(c {\color{black} \| \salt} \| \bfs)$
\STATE (in constant time) $\ssk' \gets \ssk_0'$ if $c = \PKE.\Enc(\mu', \pk; \bfr')$ else $\ssk' \gets \ssk_1'$
\RETURN $\ssk'$
\end{algorithmic}
\end{minipage}
}
\caption{Construction of an \INDCCA-secure key encapsulation mechanism
  $\KEMnotperpprime=\SFOnotperpprime[\PKE,G_1,G_2,F]$ from a public-key
  encryption scheme $\PKE$ and hash functions $G_1$, $G_2$, and $F$.}
\label{fig:kem-qfo}
\end{figure}

\begin{remark}
If the use of $\salt$ is removed from the $\SFOnotperpprime$ transform in \autoref{fig:kem-qfo},
it defaults to $\FOnotperpprime$ based on the standard Fujisaki--Okamoto transform~\cite{EuroSP:Kyber,NISTPQC-R3:FrodoKEM20}.
$\FOnotperpprime$ is used to build the \eFrodoKEM variant, which restricts the number of ciphertexts generated per public key (see \autoref{sec:cca-kem} and \autoref{sec:strength:justification}).
\end{remark}

%%% Local Variables:
%%% mode: latex
%%% TeX-master: "Main"
%%% End:


\section{$\FrodoKEM$: \INDCCA-secure key encapsulation}%
\label{sec:cca-kem}

This section defines $\FrodoKEM$, a key encapsulation mechanism that is derived from $\FrodoPKE$ by applying the $\SFOnotperpprime$ transform.
%Some readers may be familiar with the version of $\FrodoKEM$ submitted to the NIST post-quantum cryptography standardization project, available in \cite{spec}. We refer to that version as $\eFrodoKEM$ or "ephemeral" $\FrodoKEM$ that is intended for applications in which the number of ciphertexts produced relative to any single public key is small. We highlight changes from $\eFrodoKEM$ in red. 
$\FrodoKEM$ is parameterized by the following:
\begin{itemize}
\item $q=2^D$, a power-of-two integer modulus with exponent $D \leq 16$;
\item $n, \mbar, \nbar$, integer matrix dimensions with $n \equiv 0 \pmod {16}$;
\item $B \le D$, the number of bits encoded in each matrix entry;
\item $\ell=B\cdot \mbar\cdot\nbar$, the length of bit strings to be encoded in an $\mbar$-by-$\nbar$ matrix;
\item $\lengthm = \ell$, the bit length of messages;
\item $\MsgSp = \{0,1\}^\lengthm$, the message space;
\item $\lengthseedA$, the bit length of seeds used for pseudorandom matrix generation;
\item $\lengthseedSE$, the bit length of seeds used for pseudorandom bit generation for error sampling;
\item $\Frodo.\gen$, pseudorandom matrix generation algorithm, either \autoref{alg:genA_AES} or \autoref{alg:genA_SHAKE};
\item $T_\chi$, distribution table for sampling;
\item $\lengths$, the length of the bit vector $\bfs$ used for pseudorandom shared secret generation in the event of decapsulation failure in the $\SFOnotperpprime$ transform;
\item $\lengthz$, the bit length of seeds used for pseudorandom generation of $\seedA$;
\item {\color{black}$\lengthsalt$, the bit length of $\salt$;}
\item $\lengthK$, the bit length of intermediate shared secret $\bfk$ in the $\SFOnotperpprime$ transform;
\item $\lengthpkhash$, the bit length of the hash $G_1(\pk)$ of the public key in the $\SFOnotperpprime$ transform;
\item $\lengthss$, the bit length of shared secret $\ssk$ in the $\SFOnotperpprime$ transform;
\end{itemize}

\begin{algorithm}[t]
\caption{\label{alg:KEM:KeyGen} $\FrodoKEM.\KeyGen$.}
{\bf Input:} None.\\
{\bf Output:} Key pair $(\pk, \sk')$ with $\pk \in \{0,1\}^{\lengthseedA + D\cdot n \cdot \nbar}$, $\sk' \in \{0,1\}^{\lengths + \lengthseedA + D\cdot n \cdot \nbar}\times \bbZ_q^{\nbar\times n} \times \{0,1\}^{\lengthpkhash}$.\\[-1.5ex]
\rule{\linewidth}{.5pt}
\vspace{-0.5cm}
\begin{algorithmic}[1]
    \STATE Choose uniformly random seeds $\bfs \| \seedSE\| \bfz \getsr U(\{0,1\}^{\lengths + \lengthseedSE + \lengthz})$
    \STATE Generate pseudorandom seed $\seedA \gets \SHAKE(\bfz, \lengthseedA)$
    \STATE Generate the matrix $\bfA \in \bbZ_q^{n \times n}$ via $\bfA\gets\Frodo.\gen(\seedA)$
    \STATE Generate pseudorandom bit string $ (\bfr^{(0)}, \bfr^{(1)}, \dots, \bfr^{(2n\nbar-1)}) \gets \SHAKE(\mathtt{0x5F} \| \seedSE,$ $2n\nbar \cdot \lengthchi)$
    \STATE Sample error matrix $\bfS^\text{T} \gets \Frodo.\SampV((\bfr^{(0)}, \bfr^{(1)}, \dots, \bfr^{(n\nbar-1)}), \nbar, n, T_\chi)$
    \STATE Sample error matrix $\bfE \gets \Frodo.\SampV((\bfr^{(n\nbar)}, \bfr^{(n\nbar+1)}, \dots, \bfr^{(2n\nbar-1)}), n, \nbar, T_\chi)$
    \STATE Compute $\bfB \gets \bfA\bfS + \bfE$
    \STATE Compute $\bfb \gets \Frodo.\Pack(\bfB)$
    \STATE Compute $\pkh \gets \SHAKE(\seedA\| \bfb, \lengthpkhash)$
    \RETURN public key $\pk \gets \seedA\| \bfb$ and secret key $\sk' \gets (\bfs \| \seedA \| \bfb, \bfS^\text{T}, \pkh)$
\end{algorithmic}
\end{algorithm}

\begin{algorithm}[t]
\caption{\label{alg:KEM:Encaps} $\FrodoKEM.\Encaps$.}
{\bf Input:} Public key $\pk = \seedA\| \bfb \in \{0,1\}^{\lengthseedA + D\cdot n \cdot \nbar}$.\\
{\bf Output:} Ciphertext $\bfc_1\| \bfc_2{\color{black}\| \salt}\in \{0,1\}^{(\mbar\cdot n +\mbar\cdot\nbar)D{\color{black}+\lengthsalt}}$ and shared secret $\ssk \in \{0,1\}^\lengthss$.\\[-1.5ex]
\rule{\linewidth}{.5pt}
\vspace{-0.5cm}
\begin{algorithmic}[1]
    \STATE Choose uniformly random values $\mu \getsr U(\{0,1\}^\lengthm)$ {\color{black}and $\salt \getsr U(\{0,1\}^\lengthsalt$)}
    \STATE Compute $\pkh \gets \SHAKE(\pk, \lengthpkhash)$
    \STATE Generate pseudorandom values $\seedSE\| \bfk \gets \SHAKE(\pkh \| \mu{\color{black}\| \salt}, \lengthseedSE + \lengthK)$
    \STATE Generate pseudorandom bit string $ (\bfr^{(0)}, \bfr^{(1)}, \dots, \bfr^{(2\mbar n+\mbar\nbar-1)}) \gets \SHAKE(\mathtt{0x96} \|$ $\seedSE, (2\mbar n+\mbar\nbar) \cdot \lengthchi)$
    \STATE Sample error matrix $\bfS' \gets \Frodo.\SampV((\bfr^{(0)}, \bfr^{(1)}, \dots, \bfr^{(\mbar n-1)}), \mbar, n, T_\chi)$
    \STATE Sample error matrix $\bfE' \gets \Frodo.\SampV((\bfr^{(\mbar n)}, \bfr^{(\mbar n+1)}, \dots, \bfr^{(2\mbar n-1)}), \mbar, n,$ $T_\chi)$
    \STATE Generate $\bfA\gets \Frodo.\gen(\seedA)$
    \STATE Compute $\bfB' \gets \bfS'\bfA + \bfE'$
    \STATE Compute $\bfc_1 \gets \Frodo.\Pack(\bfB')$
    \STATE Sample error matrix $\bfE'' \gets \Frodo.\SampV((\bfr^{(2\mbar n)}, \bfr^{(2\mbar n+1)}, \dots, \bfr^{(2\mbar n+\mbar\nbar-1)}),$ $\mbar, \nbar, T_\chi)$
    \STATE Compute $\bfB \gets \Frodo.\Unpack(\bfb,n,\nbar)$
    \STATE Compute $\bfV \gets \bfS'\bfB + \bfE''$
    \STATE Compute $\bfC \gets \bfV + \Frodo.\Encode(\mu)$
    \STATE Compute $\bfc_2 \gets \Frodo.\Pack(\bfC)$
    \STATE Compute $\ssk \gets \SHAKE(\bfc_1 \| \bfc_2{\color{black}\| \salt} \| \bfk, \lengthss)$
    \RETURN ciphertext $\bfc_1 \| \bfc_2{\color{black}\| \salt}$ and shared secret $\ssk$
    \end{algorithmic}
\end{algorithm}

\begin{algorithm}
\caption{\label{alg:KEM:Decaps} $\FrodoKEM.\Decaps$.}
{\bf Input:} Ciphertext $\bfc_1 \| \bfc_2{\color{black}\| \salt} \in \{0,1\}^{(\mbar\cdot n +\mbar\cdot\nbar)D {\color{black}+\lengthsalt}}$, secret key $\sk' = (\bfs \| \seedA \| \bfb, \bfS^\text{T}, \pkh) \in  \{0,1\}^{\lengths + \lengthseedA +D\cdot n \cdot \nbar}\times \bbZ_q^{\nbar\times n} \times \{0,1\}^{\lengthpkhash}$.\\
{\bf Output:} Shared secret $\ssk \in \{0,1\}^\lengthss$.\\[-1.5ex]
\rule{\linewidth}{.5pt}
\vspace{-0.5cm}
\begin{algorithmic}[1]
    \STATE $\bfB' \gets \Frodo.\Unpack(\bfc_1, \mbar, n)$
    \STATE $\bfC \gets \Frodo.\Unpack(\bfc_2, \mbar, \nbar)$    
    \STATE Compute $\bfM \gets \bfC - \bfB'\bfS$
    \STATE Compute $\mu' \gets \Frodo.\Decode(\bfM)$
    \STATE Parse $\pk \gets \seedA \|\bfb$
    \STATE Generate pseudorandom values $\seedSE'\| \bfk' \gets \SHAKE(\pkh \| \mu'{\color{black}\| \salt}, \lengthseedSE + \lengthK)$
   \STATE Generate pseudorandom bit string $ (\bfr^{(0)}, \bfr^{(1)}, \dots, \bfr^{(2\mbar n+\mbar\nbar-1)}) \gets \SHAKE(\mathtt{0x96} \|$ $\seedSE', (2\mbar n+\mbar\nbar) \cdot \lengthchi)$
    \STATE Sample error matrix $\bfS' \gets \Frodo.\SampV((\bfr^{(0)}, \bfr^{(1)}, \dots, \bfr^{(\mbar n-1)}), \mbar, n, T_\chi)$
    \STATE Sample error matrix $\bfE' \gets \Frodo.\SampV((\bfr^{(\mbar n)}, \bfr^{(\mbar n+1)}, \dots, \bfr^{(2\mbar n-1)}), \mbar, n,$ $T_\chi)$
    \STATE Generate $\bfA\gets \Frodo.\gen(\seedA)$
    \STATE Compute $\bfB'' \gets \bfS'\bfA + \bfE'$
    \STATE Sample error matrix $\bfE'' \gets \Frodo.\SampV((\bfr^{(2\mbar n)}, \bfr^{(2\mbar n+1)}, \dots, \bfr^{(2\mbar n+\mbar\nbar-1)}),$ $\mbar, \nbar, T_\chi)$
    \STATE Compute $\bfB \gets \Frodo.\Unpack(\bfb,n,\nbar)$
    \STATE Compute $\bfV \gets \bfS'\bfB + \bfE''$
    \STATE Compute $\bfC' \gets \bfV + \Frodo.\Encode(\mu')$
    \STATE (in constant time) $\overline{\bfk} \gets \bfk'$ if $(\bfB' \| \bfC = \bfB'' \| \bfC')$ else $\overline{\bfk} \gets \bfs$
    \STATE Compute $\ssk \gets \SHAKE(\bfc_1 \| \bfc_2{\color{black}\| \salt} \| \overline{\bfk}, \lengthss)$
    \RETURN shared secret $\ssk$
    \end{algorithmic}
\end{algorithm}   

\subsection{Correctness of \INDCCA KEM}\label{sec:cca-kem-correctness}

For any KEM obtain from the FO transform, a correctness error occurs only on messages that exhibit a decryption error. Therefore, the failure probability $\delta$ of $\FrodoKEM$ is the same as the failure probability of the underlying $\FrodoPKE$ as computed in \autoref{sec:cpa-pke-correctness}.

\subsection{Interconversion to \INDCCA PKE}\label{sec:cca-pke}

\NISTdescription{As the KEM and public-key encryption functionalities can generally be interconverted, unless the submitter specifies otherwise, NIST will apply standard conversion techniques to convert between schemes if necessary.}

$\FrodoKEM$ can be converted to an \INDCCA-secure public-key encryption scheme using standard conversion techniques as specified by NIST.  In particular, shared secret $\ssk$ can be used as the encryption key in an appropriate data encapsulation mechanism in the KEM/DEM (key encapsulation mechanism / data encapsulation mechanism) framework~\cite{CraSho03}.

\subsection{Cryptographic primitives}\label{sec:primitives}

\NISTdescription{A complete submission shall specify any padding mechanisms and any uses of NIST-approved cryptographic primitives that are needed in order to achieve security. If the scheme uses a cryptographic primitive that has not been approved by NIST, the submitter shall provide an explanation for why a NIST-approved primitive would not be suitable.}

In $\FrodoKEM$ we use the following generic cryptographic primitives. We
describe their security requirements and instantiations with NIST-approved
cryptographic primitives. In what follows, we use
$\SHAKE128/256$ to denote the use of either $\SHAKE128$ or $\SHAKE256$;
which one is used with which parameter set for $\FrodoKEM$ is indicated in \autoref{tab:parameters-all}.

\begin{itemize}

\item $\Frodo.\gen$ in $\FrodoKEM.\KeyGen$: The security requirement on
  $\Frodo.\gen$ is that it is a public function that generates pseudorandom
  matrices $\bfA$. $\Frodo.\gen$ is instantiated using either \AESOneTwoEight
  (as in \autoref{alg:genA_AES}) or $\SHAKE128$ (as in
  \autoref{alg:genA_SHAKE}).

\item $G_1$, $G_2$, and $F$ in transform $\SFOnotperpprime$: these are
  modeled as independent random oracles (see below). We instantiate these using $\SHAKE128/256$.
\end{itemize}

%Overall, $\FrodoKEM$ has the following uses of $\SHAKE$:
%
%\begin{enumerate}
%\item $\Frodo.\gen$	using $\SHAKE128$, line 3: $\SHAKE128(\bfb, \dots)$, input $16+\lengthseedA$ bits
%\item $\FrodoKEM.\KeyGen$, line 2: $\SHAKE(\bfz, \dots)$, input $\lengthz$ bits
%\item $\FrodoKEM.\KeyGen$, line 4: $\SHAKE(\mathtt{0x5F}\|\seedSE, \dots)$, input $8+\lengthseedSE$ bits
%\item $\FrodoKEM.\KeyGen$, line 9: $\SHAKE(\seedA \| \bfb, \dots)$, input $\lengthseedA+D\cdot n \cdot \nbar$ bits
%\item $\FrodoKEM.\Encaps$, line 2: same as $\FrodoKEM.\KeyGen$, line 9
%\item $\FrodoKEM.\Encaps$, line 3: $\SHAKE(\pkh\|\mu{\color{black}\|\salt}, \dots)$, input length $\lengthpkhash+\lengthm{\color{black}+\lengthsalt}$ bits
%\item $\FrodoKEM.\Encaps$, line 4: $\SHAKE(\mathtt{0x96}\|\seedSE, \dots)$, input length $8+\lengthseedSE$ bits
%\item $\FrodoKEM.\Encaps$, line 15: $\SHAKE(\bfc_1\|\bfc_2{\color{black}\|\salt}\|\bfk, \dots)$, input length $(\mbar\cdot n+\mbar\cdot\nbar)D {\color{black}+\lengthsalt} + \lengthK$ bits
%\item $\FrodoKEM.\Decaps$, line 6: same as $\FrodoKEM.\Encaps$, line 3
%\item $\FrodoKEM.\Decaps$, line 7: same as $\FrodoKEM.\Encaps$, line 4
%\item $\FrodoKEM.\Decaps$, line 17: same as $\FrodoKEM.\Encaps$, line 15
%\end{enumerate}

\paragraph{Domain separation for $\SHAKE$.}
Each distinct use of $\SHAKE$ in \FrodoKEM should be cryptographically independent, which is achieved via one of two forms of domain separation.  

$\SHAKE$, and the underlying Keccak operation, employ padding to pad input strings to a multiple of the rate.  The specific padding used is appending the string $\mathtt{10^*1}$.  Thus, inputs of different lengths yield different padded strings.  

For uses of $\SHAKE$ where the inputs are of different lengths, we rely on Keccak's padding for domain separation.

For uses of $\SHAKE$ where the inputs are of the same length (i.e., line 4 in $\FrodoKEM.$ $\KeyGen$ and $\FrodoKEM.\Encaps$, and line 7 in $\FrodoKEM.\Decaps$), we prepend distinct bytes as domain separators.  These domain separators have bit patterns ($\mathtt{0x5F}=\mathtt{01011111}$, $\mathtt{0x96}=\mathtt{10010110}$) that were chosen to make it hard to use individual or consecutive bit flipping attacks to turn one into the other.

\subsection{\FrodoKEM variants}\label{sec:variants}

\FrodoKEM is parameterized by the pseudorandom generator (PRG) that is used for
the generation of the matrix $\bfA$ in $\Frodo.\gen$. As explained in \autoref{sec:genA}, there are two options:
using $\AESOneTwoEight$ (\autoref{alg:genA_AES}) and using $\SHAKE128$ (\autoref{alg:genA_SHAKE}).

In addition, \FrodoKEM consists of two main variants: an ``ephemeral'' variant, called \eFrodoKEM,
that is intended for applications in which the number of ciphertexts produced
relative to any single public key is fairly small (e.g., less than $2^8$), and a ``salted'' variant, simply called \FrodoKEM, that does
not impose any restriction on the reuse of key pairs.

In contrast to \eFrodoKEM, the salted KEM \FrodoKEM is constructed by applying the $\SFOnotperpprime$ transform
and incorporates some changes to address certain multi-ciphertext attacks. 
Specifically, salted \FrodoKEM doubles the length of the $\seedSE$ value
and incorporates a public random value $\salt$ into encapsulation (see \autoref{tab:parameters-additional}).

%%% Local Variables:
%%% mode: latex
%%% TeX-master: "Main"
%%% End:


\section{Justification of security strength}%
\label{sec:strength:justification}

The security of $\FrodoKEM$ is supported both by security reductions
and by analysis of the best known cryptanalytic attacks.
A full overview of the reductions supporting \FrodoKEM's security is given in \autoref{sec:Security_appendix},
while the analysis of the best known cryptanalytic attacks can be found in \autoref{sec:attack:cryptanalytic}.
A summary of the bit-security estimates based on these two methodologies is shown in \autoref{tab:security}.

In this section, we provide the main theorems supporting the security of \FrodoKEM
in the ROM and QROM. Refer to \autoref{sec:Security_appendix} for complete details.

%\subsection{Security results}
\subsection{\MINDCCA security in the random oracle model}
\label{sec:strength:cca-kem} \label{sec:renyi_loss}

The following theorem says that the transformation $\SFOnotperpprime$, which we use to construct \FrodoKEM from \FrodoPKE, generically yields an \MINDCCA-secure KEM (in the classical random oracle model) from an \MINDCPA-secure public-key encryption scheme, even if the KEM and PKE are parameterized by different distributions, provided that those distributions are sufficiently close in terms of \renyi divergence.
We present multi-challenge security bounds, parameterized by the number of challenge ciphertexts~$n$, and the number of users~$u$.
A detailed description of the proof steps is given in \autoref{sec:rom-mindcca}.

To specialize this result to obtain an ordinary (single-key, single-ciphertext) $\INDCCA$ security bound, one merely sets $|\saltlist| = n = u = 1$.
In this case, the final additive term in the security bound from~\eqref{eq:cca-to-cpa-renyi-bound} is zero, and the overall bound matches what was obtained in~\cite{NISTPQC-R3:FrodoKEM20}.

% \begin{remark}
% In addition, the term $n^2/|\MsgSp|2^{\lengthsalt}$ term represents the probability of a collision in the challenge ciphertexts, and can be removed when we only have 1 challenge. Although, for $n = 1$ this term is negligible.
    
% \end{remark}

% \cpnote{I don't see such a term, but I do see one that also has a $\abs{\MsgSp}$ factor in the denominator.
%   I wonder if this term actually has a numerator of $n(n-1)$, in which case we do not have to say that it can be removed (because $n-1=0$).
%   It is kind of cheating to say that it can be removed, because that's not a specialization of this theorem, so it requires a justification (which we haven't provided).}

\iffalse (Lewis and Chris:) removing this rushed remark but keeping the text for future reference

\begin{remark}
  The results below are not sufficient to establish \MINDCCA security for \FrodoKEM, as we do not establish the \MINDCPA security of \FrodoPKE.
  \cpnote{The previous statement is not true, and sells us somewhat short.
    We do (trivially) have an \MINDCPA bound for \FrodoPKE; it is just looser than the one for (single-challenge) \INDCPA.
    But it does establish \MINDCCA security for \FrodoKEM.}
  What they do establish is that \MINDCCA security for \FrodoKEM reduces \emph{tightly} to the \MINDCPA security of \FrodoPKE, which is not true for \eFrodoKEM.
  Notably, the following results prove that the multi-ciphertext attack described in \autoref{sec:intro:multi-challenge} to \emph{un-salted} \FrodoKEM is not possible against \emph{salted} \FrodoKEM.
  \cpnote{I don't think the previous sentence is true.
    What if the \MINDCPA insecurity of \FrodoPKE grows linearly with $n$; then according to the theorem, the \MINDCCA security of \FrodoKEM might also grow linearly with $n$, which puts us in the same situation as with the multi-ciphertext attack, right?}
  Furthermore, they show that any multi-challenge attack on \FrodoKEM would be the result of a multi-sample attack on LWE.
  \cpnote{I think the previous sentence is basically true, but it's vague; LWE is already a multi-sample problem.
    I think we're talking about an attack on LWE with a large number of samples \emph{and secrets} (shared across the samples), right?}
\end{remark}

\fi

\lewis{If we change our advantage definition in order to have an extra factor of 2, meaning a perfect attack has advantage 1, then for theorem 1, we simply double all intermediate terms. This has no impact on the distribution switch multiplicative term. }
\begin{theorem}[\MINDCPA PKE $\implies$ \MINDCCA KEM in classical ROM,
  with distribution switch]\label{thm:cca-kem-to-cpa-pke-rom-parameterized}
  \ \newline
  Let $\PKE_X = (\KeyGen,\Enc,\Dec)$ be a $\delta(u)$-correct public-key
  encryption scheme with message space $\MsgSp$ that is parameterized
  by a distribution~$X$, and let $s$ be an upper bound on the total
  number of samples drawn from~$X$ by $\KeyGen$ and $\Enc$ combined.
  Let $G_1$, $G_2$ and $F$ be independent random oracles, and let
  $\KEMnotperpprime_X=\SFOnotperpprime[\PKE_X,G_1,G_2,F]$ be the KEM
  obtained by applying the $\SFOnotperpprime$ transform from
  \autoref{def:qfo} to $\PKE_{X}$. Let $P,Q$ be any discrete
  distributions. There exists a classical algorithm (a reduction)
  $\Bdversary$ against the \INDCPA security of $\PKE_Q$, which uses as
  a ``black box'' subroutine any $\Adversary$ against the \INDCCA
  security of $\KEMnotperpprime_P$ that makes at most $\qro$ oracle
  queries, for which
  \cpnote{Changed the $n^{2}$ numerator to $n(n-1)$; check that this is correct in the context of the other paper and update if possible.}
  \begin{align}
        \label{eq:cca-to-cpa-renyi-bound}
        %\begin{split}  
        \Adv{\MINDCCA}{\KEMnotperpprime_P}(\Adversary) \leq &
        \bigg( \left(
        \frac{n\cdot (2\qro+1)}{|\MsgSp||\saltlist|}
        + \qro \cdot \delta(u)
        + 3 \cdot\Adv{\MINDCPA}{\PKE_Q}(\Bdversary) \right) \nonumber \\
        & \cdot \exp(s \cdot \RD_\alpha(P \| Q)) \bigg)^{1-1/\alpha}+\frac{\qro}{|\MsgSp|}+ \frac{n(n-1)}{|\MsgSp| \cdot 2^{\lengthsalt}} 
        %\end{split}
  \end{align}
  for any $\alpha > 1$, where the \renyi divergence $\RD_\alpha$ is defined in \autoref{def:renyi}.
  The total running time of $\Bdversary$ is about that of $\Adversary$ plus the time needed to simulate the random oracles.
\end{theorem}

We point out that when $P=Q$ are the same distribution, we have
$\exp(s \cdot \RD_{\alpha}(P \| Q)) = 1$ for any $\alpha > 1$ and
hence can take~$\alpha$ to be arbitrarily large, making the exponent
$1-1/\alpha$ approach~$1$ in the limit. This special case is a main
theorem from~\cite{TCC:HofHovKil17}, and it relates the \INDCCA
security of $\FrodoKEM$ to the \INDCPA security of $\FrodoPKE$ when
they use the same error distribution, e.g., ~$\chi_{\Frodo}$.

The proof of \autoref{thm:cca-kem-to-cpa-pke-rom-parameterized}
combines components from three separate works: the modular analysis of
the Fujisaki--Okamoto transform by Hofheinz, H{\"o}velmanns and Kiltz~\cite{TCC:HofHovKil17}, the work on tight multi-target security for key encapsulation in \cite{GlabushThesis}, and the
work of Langlois, Stehl{\' e} and Steinfeld relating the security of
search problems when one distribution is substituted by another via
analysis of the \renyi divergence~\cite{EC:LanSteSte14}.  More
specifically, the proof of the theorem proceeds in the following
steps:
\begin{enumerate}
\item We apply Theorem 3.2 by Hofheinz et al.~\cite{TCC:HofHovKil17}, which shows that their $\T$
  transform converts an \INDCPA-secure public-key encryption scheme
  $\PKE_{Q}$ into an \OWPCA-secure public-key encryption scheme with
  deterministic encryption (in the random oracle model). Likewise, Theorem 3.2 in~\cite{TCC:HofHovKil17} is adapted in~\cite{GlabushThesis} to the $\ST$ transform used by $\FrodoKEM$; see~\cite[Section 4.2]{GlabushThesis}.
\item Next, we apply distribution substitution for the \OWPCA security
  experiment (which represents a search problem), to switch from
  distribution~$Q$ to~$P$.
\item Finally, we apply Glabush's adaptation of Theorem 3.4 from~\cite{TCC:HofHovKil17}, which shows that their
  $\Unotperp$ transform converts an \MOWPCA-secure public-key
  encryption scheme into an \MINDCCA-secure KEM (in the random oracle
  model).
\end{enumerate}

Hofheinz et al.~\cite{TCC:HofHovKil17} denote the composition of the $\T$ and $\Unotperp$ transforms as
the $\FOnotperp$ transform.  As described in
\autoref{sec:cca-transform}, we use a variant of this
transform called $\SFOnotperpprime$, which differs from
$\FOnotperp$ as follows:
\begin{itemize}
\item The $\T$ transform is replaced with the $\ST$ transform.
\item $\SFOnotperpprime$ uses a single hash function (with longer
  output) to compute $r$ and $K$, whereas $\FOnotperp$ uses two
  separate functions, but these are equivalent when the hash functions
  are modeled as independent random oracles and have appropriate
  output lengths.
\item The~$\SFOnotperpprime$ computation of $r$ and $K$ also takes the
  hash $G_1(\pk)$ of the public key $\pk$ as input,
  whereas~$\FOnotperp$ does not; this change preserves the relevant
  theorems (with trivial changes to the proofs), and has the potential
  to provide stronger multi-target security.
\end{itemize}

If we apply all the changes above, excepting the replacement of $\T$ by $\ST$, this results in the $\FOnotperpprime$ transform~\cite{EuroSP:Kyber,NISTPQC-R3:FrodoKEM20}.
The $\FOnotperpprime$ transform is applied to \FrodoPKE to build \eFrodoKEM.


\subsection{\INDCCA security in the quantum random oracle model}
\label{sec:strength:cca-kem-qrom}

Jiang et al.~\cite{C:JZCWM18} show that the $\FOnotperp$ transform yields an \INDCCA-secure KEM from an \OWCPA-secure public-key encryption scheme, in the \emph{quantum} random oracle model.
In~\cite{GlabushThesis}, this result was extended as to show that the $\SFOnotperp$ transform generically yields an $\MINDCCA$-secure KEM from a $\MINDCPA$ secure PKE.
This result extends to the $\SFOnotperpprime$ transform.

\begin{theorem}
  Let $\PKE = (\KeyGen,\Enc,\Dec)$ be a $\delta(u)$-correct public-key
  encryption scheme with message space $\MsgSp$. For any quantum adversary $\Bdversary$ against the $\MINDCCA$ security of $\KEMnotperp := \SFOnotperpprime[\PKE, G, H, \lengthsalt]$, there exists a quantum adversary $\Adversary$ against the $\MINDCPA$ security of $\PKE$, with roughly the same running time, such that
  \begin{equation}\label{thm:cca-kem-to-cpa-pke-qrom}
    \begin{split}  
        \Adv{\MINDCCA}{\KEMnotperp}(\Adversary) \leq &\dfrac{u^2}{|\M|2^{\lengthsalt}} + 4(\qro + 1)\sqrt{\dfrac{u\cdot n}{|\M\|\saltlist|}} + 16(\qro + 1)^2\cdot \delta(u)\\ & +
        4(\qro+1)\sqrt{\dfrac{u}{\M}} + 2\sqrt{(\qro + 1)\Adv{\MINDCPA}{\PKE}(\Bdversary)}
    \end{split}
  \end{equation}
\end{theorem}

%%% Local Variables:
%%% mode: latex
%%% TeX-master: "Main"
%%% End:


\section{Parameters}%
\label{sec:params}

\ifshoworiginal
This section outlines our methodology for choosing tunable parameters of the proposed algorithms. 

\subsection{High-level overview}

Recall the main $\FrodoPKE$ parameters defined in
\autoref{sec:algs}:
\begin{itemize}
  \item $\chi$, a probability distribution on $\bbZ$;
  \item $q=2^D$, a power-of-two integer modulus with exponent $D \leq 16$;
  \item $n,\mbar,\nbar$, integer matrix dimensions with $n \equiv 0 \pmod 8$;
  \item $B\leq D$, the number of bits encoded in each matrix entry;
  \item $\ell=B\cdot \mbar\cdot\nbar$ the length of bit strings to be encoded in an $\mbar$-by-$\nbar$ matrix.
\end{itemize}

The task of parameter selection is framed as a combinatorial
optimization problem, where the objective function is the ciphertext's
size, and the constraints are dictated by the target security level,
probability of decryption failure, and computational efficiency. The
optimization problem is solved by sweeping the parameter space,
subject to simple pruning techniques.  We perform this sweep of the
parameter space using the Python scripts that accompany the
submission, in the folder
\texttt{Additional\_Implementations/Parameter\_Search\_Scripts}.

\else

Recall the \FrodoPKE parameters defined in \autoref{sec:algs}.
The task of parameter selection is framed as a combinatorial
optimization problem, where the objective function is the ciphertext's
size, and the constraints are dictated by the target security level,
probability of decryption failure, and computational efficiency. The
optimization problem is solved by sweeping the parameter space,
subject to simple pruning techniques.  We perform this sweep of the
parameter space using the Python scripts that accompany the
submission (folder \texttt{Parameter\_Search\_Scripts}):
\url{https://doi.org/10.5281/zenodo.14633189}.
\fi

\subsection{Parameter constraints}\label{sec:constraints}

Implementation considerations limit $q$ to be at most $2^{16}$ and $n$
to be a multiple of 16. Our cost function is the sum of the bit lengths of
$\FrodoPKE$'s ciphertext and its public key, which is $D\cdot (n\cdot (\mbar\ + \nbar)+\mbar\ \nbar)+\lengthseedA$.

The standard deviation~$\sigma$ of the Gaussian error distribution is
taken to exceed the ``smoothing parameter'' of the
integers~$\mathbb{Z}$, for a very small error parameter $\eps >
0$. The specific values of~$\sigma$ are chosen following the
methodology in \autoref{sec:strength:lattice}, which demonstrates that
our choices conform to a nontrivial reduction from the worst-case
\BDDwDGS problem to the corresponding average-case LWE decision
problem.

The complexity of the error-sampling algorithm
(\autoref{sec:sampling}) depends on the support of the distribution
and the number of uniformly random bits per sample. We bound the
number of bits per sample by 16. Since the distribution is symmetric,
the sample's sign ($\bfr_0$ in \autoref{alg:samplechi}) can be chosen
independently from its magnitude~$e$, which leaves 15 bits for
sampling from the non-negative part of the support. For each setting
of the variance $\sigma^2$ we find a discrete distribution subject to
the above constraints that minimizes its \renyi divergence (for
several integral orders) from the target ``ideal'' distribution, which
is the rounded Gaussian~$\Psi_{\sigma \sqrt{2\pi}}$.

We estimate the concrete security of parameters for our scheme based
on cryptanalytic attacks (\autoref{sec:attack:cryptanalytic}),
accounting for the loss due to substitution of a rounded Gaussian with
its discrete approximation (\autoref{sec:renyi_loss}).  The
probability of decryption failure is computed according to the
procedure outlined in \autoref{sec:cpa-pke}.

In case of ties, i.e., when different parameter sets result in
identical ciphertext sizes (i.e., the same $q$ and~$n$), we chose the
smaller~$\sigma$ for \FrodoKEMLOne and \FrodoKEMLFive (minimizing the probability of
decryption failure), and the larger~$\sigma$ for \FrodoKEMLThree
(prioritizing security).

\subsection{Selected parameter sets}\label{sec:params:sets}

We present three core parameter sets for \FrodoKEM:
\begin{itemize}
\item \FrodoLOne, matching or exceeding the brute-force security of $\AESOneTwoEight$,
\item \FrodoLThree, matching or exceeding the brute-force security of $\AESOneNineTwo$, and
\item \FrodoLFive, matching or exceeding the brute-force security of $\AESTwoFiveSix$,
\end{itemize}
\noindent which target Levels 1, 3 and 5, respectively,
in the NIST call for proposals~\cite{NIST17}.

We parameterize each core set by the PRG that is used for the generation
of matrix $\bfA$. 
As described in~\autoref{sec:genA}, \FrodoKEM allows two options for the PRG:
$\AESOneTwoEight$ and $\SHAKE128$. 
In addition, \FrodoKEM consists of two main variants: a \emph{salted} variant simply
called \FrodoKEM that does not impose any restriction on the reuse of key pairs,
and an \emph{ephemeral} variant called \eFrodoKEM that is intended for applications
in which the number of ciphertexts produced relative to any single public key is small (see \autoref{sec:variants}).
%More specifically, to address certain multi-ciphertext attacks, \eFrodoKEM doubles
%the length of the $\seedSE$ value, and incorporates a public random salt value into encapsulation.

Thus, in total, we propose twelve parameter sets. The variant \FrodoKEM includes the
parameter sets \FrodoKEMLOneAES, \FrodoKEMLThreeAES, \FrodoKEMLFiveAES,
\FrodoKEMLOneSHAKE, \FrodoKEMLThreeSHAKE and \FrodoKEMLFiveSHAKE, and the variant \eFrodoKEM
includes the parameter sets \eFrodoKEMLOneAES, \eFrodoKEMLThreeAES, \eFrodoKEMLFiveAES,
\eFrodoKEMLOneSHAKE, \eFrodoKEMLThreeSHAKE and \eFrodoKEMLFiveSHAKE.

\autoref{tab:parameters-all} and \autoref{tab:parameters-additional} summarize
the cryptographic parameters for all the parameter sets.
The corresponding error distributions appear in \autoref{tab:distribution}.
\autoref{tab:security} summarizes security claims we can make about \FrodoKEM
and its components. The columns LWE security C, Q and P respectively denote
security, in bits, for classical, quantum, and plausible attacks on $\mbar+\nbar$
instances of the normal-form (decisional) LWE problem with Gaussian error
distribution (\autoref{sec:lwe}) as estimated by the methodology of
\autoref{sec:attack:cryptanalytic}. The column IND-CCA security C denotes $\INDCCA$
security, in bits, for classical random oracle model attacks according to
\autoref{thm:cca-kem-to-cpa-pke-rom-parameterized}.

\begin{table}[h]
\caption{Cryptographic parameters for $\FrodoKEMLOne$, $\FrodoKEMLThree$, $\FrodoKEMLFive$, and
their corresponding ephemeral variants.
For each set, $\lengthm = \lengths = \lengthK = \lengthpkhash = \lengthss = \ell$.}\label{tab:parameters-all}
\begin{center}
\begin{tabular}{l|r|r|r }
\toprule
& $(\styleScheme{e})\FrodoKEMLOne$ & $(\styleScheme{e})\FrodoKEMLThree$ & $(\styleScheme{e})\FrodoKEMLFive$ \\
\midrule
$D$ & $15$ & $16$ & $16$ \\
$q$ & $32768$ & $65536$ & $65536$ \\
$n$ & $640$ & $976$ & $1344$ \\
$\mbar=\nbar$ & $8$ & $8$ & $8$ \\
$B$ & $2$ & $3$ & $4$ \\
$\lengthseedA$ & $128$ & $128$ & $128$ \\
$\lengthz$ & $128$ & $128$ & $128$ \\
$\ell$ & $128$ & $192$ & $256$ \\
$\lengthchi$ & $16$ & $16$ & $16$ \\
$\chi$ & $\chi_\FrodoLOne$ & $\chi_\FrodoLThree$ & $\chi_\FrodoLFive$ \\
$\SHAKE$ & $\SHAKE128$ & $\SHAKE256$ & $\SHAKE256$ \\
\bottomrule
\end{tabular}
\end{center}
\end{table}

\begin{table}[h]
\caption{Size (in bits) of $\lengthseedSE$ and $\lengthsalt$.}\label{tab:parameters-additional}
\begin{center}
\begin{tabular}{l|r|r|r }
\toprule
& $\FrodoKEMLOne$ & $\FrodoKEMLThree$ & $\FrodoKEMLFive$ \\
\midrule
$\lengthseedSE$ & $256$ & $384$ & $512$ \\
$\lengthsalt$ & $256$ & $384$ & $512$ \\
\toprule
& $\eFrodoKEMLOne$ & $\eFrodoKEMLThree$ & $\eFrodoKEMLFive$ \\
\midrule
$\lengthseedSE$ & $128$ & $192$ & $256$ \\
$\lengthsalt$ & $0$ & $0$ & $0$ \\
\bottomrule
\end{tabular}
\end{center}
\end{table}


\ifshoworiginal
The procedures outlined in this section can be adapted to support
alternative cost functions and constraints. For instance, an objective
function that takes into account computational costs or penalizes the
public key size would lead to a different set of outcomes. For
example, constraints can be also chosen to guarantee error-free
decryption, or to select parameters that allow for a bounded number of
homomorphic operations.

The three parameter sets are given in \autoref{tab:parameters}.  The
corresponding error distributions appear in
\autoref{tab:distribution}. \autoref{tab:security} summarizes security claims we can make about \FrodoKEM and its components. The columns LWE security C, Q and P respectively
denote security, in bits, for classical, quantum, and plausible attacks on $\mbar+\nbar$ instances of the normal-form (decisional) LWE problem with Gaussian error distribution (\autoref{sec:lwe}) as estimated by the methodology of \autoref{sec:attack:cryptanalytic}. The column $\indcca$ security C denotes $\indcca$ security, in bits, for classical random oracle model attacks according to \autoref{thm:cca-kem-to-cpa-pke-rom-parameterized}.

\begin{table}[h]
	\caption{\textbf{Parameters at a glance}}\label{tab:parameters}
	\begin{center}
		\begin{tabular}{l | c c c c c c c}
			\toprule
			& $n$ & $q$& $\sigma$ & \textbf{support} & $B$ & $\bar{m}\times \bar{n}$ & $c$ \textbf{size} \\
			& & & & \textbf{of} $\chi$ & & & \textbf{(bytes)}\\
			\midrule
			\FrodoLOne & \!\! 640 \!\! & \!\! $2^{15}$ \!\! &2.8 &$[-12\dots 12]$ &\!\!2\!\! & $8\times 8$ & \hphantom{0}9,720 \\
			\FrodoLThree & \!\! 976 \!\! & \!\! $2^{16}$ \!\! &2.3 &$[-10\dots 10]$ &\!\!3\!\! & $8\times 8$ & 15,744 \\
			\FrodoLFive & \!\! 1344 \!\! & \!\! $2^{16}$ \!\! &1.4 &$[-6\dots 6]$ &\!\!4\!\! & $8\times 8$ & 21,632 \\
\bottomrule
\end{tabular}
\end{center}
\end{table}
\fi

\begin{table}[h]
\caption{Error distributions.}\label{tab:distribution}
\begin{center}
\scalebox{0.78}{
\begin{tabular}{l|c| r r r r r r r r r r r r r|c c}
\toprule
 & $\sigma$ & \multicolumn{13}{c|}{\textbf{Probability of (in multiples of $2^{-16}$)}} & \multicolumn{2}{c}{\textbf{\renyi}} \\
             & & 0 &\!\!$\pm 1$\!\!&\!\!$\pm 2$\!\!&\!\!$\pm 3$\!\!&\!\!$\pm 4$\!\!&\!\!$\pm 5$&\!\!$\pm 6$\!\!&\!\!$\pm 7$\!\!&\!\!$\pm 8$\!\!&\!\!$\pm 9$\!\!&\!\!$\pm 10$\!\!&\!\!$\pm 11$\!\!&\!\!$\pm 12$\!\!& \textbf{order} & \textbf{divergence} 
\\ \midrule
$\chi_\FrodoLOne$ & 2.8 &\!\!9288\!\!&\!\!8720\!\!&\!\!7216\!\!&\!\!5264\!\!&\!\!3384\!\!&\!\!1918\!\!&\!\!958\!\!&\!\!422\!\!&\!\!164\!\!&\!\!56\!\!&\!\!17\!\!&\!\!4\!\!&\!\!1\!\!& 200 & $0.324\times 10^{-4}$ \\
$\chi_\FrodoLThree$ & 2.3 &\!\!11278\!\!&\!\!10277\!\!&\!\!7774\!\!&\!\!4882\!\!&\!\!2545\!\!&\!\!1101\!\!&\!\!396\!\!&\!\!118\!\!&\!\!29\!\!&\!\!6\!\!&\!\!1\!\!&&& 500 & $0.140\times 10^{-4}$ \\
$\chi_\FrodoLFive$ & 1.4 &\!\!18286\!\!&\!\!14320\!\!&\!\!6876\!\!&\!\!2023\!\!&\!\!364\!\!&\!\!40\!\!&\!\!2\!\!&&&&&&& 1000 & $0.264\times 10^{-4}$ \\
\bottomrule
\end{tabular}
}
\end{center}
\end{table}

\begin{table}[h]
	\caption{Security bounds.}\label{tab:security}
	\begin{center}
		\begin{tabular}{l | c c | c  c  c | c }
			\toprule
			& \textbf{target level} & \textbf{failure rate} & \multicolumn{3}{c}{\textbf{LWE security}} & \textbf{IND-CCA security}\\
			& &  & \textbf{C} & \textbf{Q} & \textbf{P} & \textbf{C} \\
			\midrule
				\FrodoLOne & 1 & $2^{-138.7}$ & 145 & 132 & 104 & 141 \\
				\FrodoLThree & 3 & $2^{-199.6}$ & 210 & 191 & 150 & 206 \\
				\FrodoLFive & 5 & $2^{-252.5}$ & 275 & 250 & 197 & 268 \\		
			\bottomrule
		\end{tabular}
	\end{center}
\end{table}


\ifshoworiginal
\subsection{Summary of parameters}

\autoref{tab:parameters-all} summarizes all cryptographic parameters for $\FrodoLOne$, $\FrodoLThree$ and $\FrodoLFive$.  $\FrodoKEMLOneAES$, $\FrodoKEMLThreeAES$ and $\FrodoKEMLFiveAES$ use $\AESOneTwoEight$ for generation of $\bfA$; $\FrodoKEMLOneSHAKE$, $\FrodoKEMLThreeSHAKE$ and $\FrodoKEMLFiveSHAKE$ 
use $\SHAKE$ for generation of $\bfA$.

\begin{table}[h]
\caption{\textbf{Cryptographic parameters for $\FrodoLOne$, $\FrodoLThree$, and $\FrodoLFive$}}\label{tab:parameters-all}
\begin{center}
\begin{tabular}{l|r|r|r }
\toprule
& $\FrodoLOne$ & $\FrodoLThree$ & $\FrodoLFive$ \\
\midrule
$D$ & $15$ & $16$ & $16$ \\
$q$ & $32768$ & $65536$ & $65536$ \\
$n$ & $640$ & $976$ & $1344$ \\
$\mbar=\nbar$ & $8$ & $8$ & $8$ \\
$B$ & $2$ & $3$ & $4$ \\
$\lengthseedA$ & $128$ & $128$ & $128$ \\
$\lengthz$ & $128$ & $128$ & $128$ \\
$\lengthm = \ell$ & $128$ & $192$ & $256$ \\
$\lengthseedSE$ & $128$ & $192$ & $256$ \\
$\lengths$ & $128$ & $192$ & $256$ \\
$\lengthK$ & $128$ & $192$ & $256$ \\
$\lengthpkhash$ & $128$ & $192$ & $256$ \\
$\lengthss$ & $128$ & $192$ & $256$ \\
$\lengthchi$ & $16$ & $16$ & $16$ \\
$\chi$ & $\chi_\FrodoLOne$ & $\chi_\FrodoLThree$ & $\chi_\FrodoLFive$ \\
$\SHAKE$ & $\SHAKE128$ & $\SHAKE256$ & $\SHAKE256$ \\
\bottomrule
\end{tabular}
\end{center}
\end{table}

\autoref{tab:size} summarizes the sizes, in bytes, of the different
inputs and outputs required by
$\FrodoKEM$. % and, for comparison, $\FrodoPKE$. The discrepancy between space complexity of the two schemes is due to the transform detailed in~\autoref{sec:spec:algs:cca-transform}.
Note that we also include the size of the public key in the secret key
sizes, in order to comply with NIST's API guidelines.  Specifically,
since NIST's decapsulation API does not include an input for the
public key, it needs to be included as part of the secret key.


\begin{table}[!ht]
\caption{\textbf{Size (in bytes) of inputs and outputs of $\FrodoKEM$.} 
Secret key size is the sum of the sizes of the actual secret value and of the public key (the NIST API does not include the public key as explicit input to decapsulation).} \label{tab:size}
\medskip
\centering
\renewcommand{\tabcolsep}{0.3cm}
\renewcommand{\arraystretch}{1.1}
\begin{tabular}{l|c c c c}
\toprule
\multirow{2}{*}{\textbf{Scheme}} & \textbf{secret key} & \textbf{public key} & \textbf{ciphertext} & \textbf{shared secret} \\
                                 & $\sk$                & $\pk$                & $c$                 & $\ssk$                 \\
\midrule
\FrodoKEMLOne & 19,888 & \hphantom{0}9,616 & \hphantom{0}9,720 & 16 \\ 
& {\scriptsize (10,272 + 9,616)}\\
\FrodoKEMLThree & 31,296 & 15,632 & 15,744 & 24 \\ 
& {\scriptsize (15,664 + 15,632)}\\
\FrodoKEMLFive & 43,088 & 21,520 & 21,632 & 32 \\ 
&{\scriptsize (21,568 + 21,520)}\\
\bottomrule
\end{tabular}
\end{table} 

\subsection{Provenance of constants and tables}\label{sec:constants}

\NISTdescription{To help rule out the existence of possible back-doors in an algorithm, the submitter shall explain the provenance of any constants or tables used in the algorithm.}

Constants used as domain separators in calls to $\SHAKE$ are described in Section~\ref{sec:spec:primitives}.

The constants in \autoref{tab:parameters} and \autoref{tab:distribution} were generated by search scripts following the methodology described in \autoref{sec:spec:params}.
\fi

\autoref{tab:size} summarizes the sizes, in bytes, of the different
inputs and outputs required by \FrodoKEM.
Note that the secret key sizes include the size of the public key,
in order to comply with NIST's API guidelines. Specifically,
since NIST's decapsulation API does not include an input for the
public key, it needs to be included as part of the secret key.


\begin{table}[!ht]
\caption{Size (in bytes) of inputs and outputs of \FrodoKEM and \eFrodoKEM.} \label{tab:size}
\medskip
\centering
\renewcommand{\tabcolsep}{0.3cm}
\renewcommand{\arraystretch}{1.1}
\begin{tabular}{l|c c c c}
\toprule
\multirow{2}{*}{\textbf{Scheme}} & \textbf{secret key} & \textbf{public key} & \textbf{ciphertext} & \textbf{shared secret} \\
                                 & $\sk$                & $\pk$                & $c$                 & $\ssk$               \\
\midrule
\FrodoKEMLOne & 19,888 & \hphantom{0}9,616 & \hphantom{0}9,752 & 16 \\
\FrodoKEMLThree & 31,296 & 15,632 & 15,792 & 24 \\ 
\FrodoKEMLFive & 43,088 & 21,520 & 21,696 & 32 \\ 
\midrule
\eFrodoKEMLOne & 19,888 & 9,616 & 9,720 & 16 \\
\eFrodoKEMLThree & 31,296 & 15,632 & 15,744 & 24 \\ 
\eFrodoKEMLFive & 43,088 & 21,520 & 21,632 & 32 \\ 
\bottomrule
\end{tabular}
\end{table} 

%%% Local Variables:
%%% mode: latex
%%% TeX-master: "Main"
%%% End:


\section{Implementation and performance analysis}%
\label{sec:performance}

\ifshoworiginal
The submission package includes:
\begin{itemize}
\item a reference implementation written exclusively in Python,
\item a reference implementation written exclusively in portable C,
\item an optimized implementation written exclusively in portable C that includes efficient algorithms to generate the matrix $\bfA$ and to compute the matrix operations $\bfA \bfS + \bfE$ and $\bfS' \bfA + \bfE'$, and
\item an additional, optimized implementation for x64 platforms that exploits Advanced Vector Extensions~2 (AVX2) intrinsic instructions.
\end{itemize}

The implementations in the submission package support all three security levels and both variants of matrix generation:
$\FrodoKEMLOneAES$, $\FrodoKEMLOneSHAKE$, $\FrodoKEMLThreeAES$, $\FrodoKEMLThreeSHAKE$, $\FrodoKEMLFiveAES$, and $\FrodoKEMLFiveSHAKE$. 
The only difference between the reference and the optimized implementation is that the latter includes two efficient functions to generate the public matrix $\bfA$ and to compute the matrix operations $\bfA \bfS + \bfE$ and $\bfS' \bfA + \bfE'$. Similarly, the only difference between the optimized and the additional implementation is that the latter uses AVX2 intrinsic instructions to speed up the implementation of the aforementioned functions. 
Hence, the different implementations share most of their codebase: this illustrates the simplicity of software based on $\FrodoKEM$. 

\else

An important feature of \FrodoKEM is that it is easy to implement and naturally facilitates
writing implementations that are compact and run in constant-time.
This latter feature aids to avoid common cryptographic implementation mistakes which can lead
to key-extraction based on, for instance, timing differences when executing the code.\footnote{Nonetheless,
care must be taken to avoid timing leaks. In 2020, Guo, Johansson, and Nilsson~\cite{C:GuoJohNil20}
demonstrated a key-recovery attack on the reference implementation in the Round 2 submission of \FrodoKEM
by exploiting branching in the computation of $\ssk$ in \FrodoKEM.\Decaps. This attack can be avoided
by ensuring the implementation reads both $\bfk'$ and $\bfs$, compares $\bfB' \| \bfC$ and $\bfB'' \| \bfC'$
in a constant-time way that avoids early termination, and sets $\overline{\bfk}$ using data-independent evaluation.}

Our compact implementation of the \FrodoKEM scheme consists of slightly more than 250 lines of plain C code.\footnote{Our reference and optimized implementations in C are available as part of this submission here: \url{https://doi.org/10.5281/zenodo.14633189}.}
This same code is used for all three security levels to implement \FrodoKEMLOne, \FrodoKEMLThree, and \FrodoKEMLFive,
with parameters changed by a small number of macros at compile-time.
Moreover, most of the code is either shared or reused for our implementation of \eFrodoKEM.
We remark that the separation in two implementations, one for \FrodoKEM and one for \eFrodoKEM, is only
done to provide a simpler and cleaner codebase supporting each API. 
In particular, the API for \eFrodoKEM has been customized to perform a \emph{single} key generation
per encapsulation execution.    

Computing on matrices---the basic operation in \FrodoKEM---allows for easy scaling to different dimensions $n$.
In addition, \FrodoKEM uses a modulus $q$ that is always equal or less than $2^{16}$. These two combined
aspects allow for the full reuse of the matrix functions for the different security levels by instantiating them
with the right parameters at build time. Since the modulus $q$ used is always a power of two, implementing
arithmetic modulo $q$ is simple, efficient and easy to do in constant-time in modern computer architectures:
for instance, computing modulo $2^{16}$ comes for free when using 16-bit data-types. Moreover, the dimension
values were chosen to be divisible by 16 in order to facilitate vectorization optimizations and to simplify the
use of $\AESOneTwoEight$ for the generation of the matrix $\bfA$.

Also the error sampling is designed to be simple and facilitates code reuse: for any security level, \FrodoKEM
requires 16 bits per sample, and the tables $T_\chi$ corresponding to the discrete cumulative density functions
always consist of values that are less than $2^{15}$. Hence, a simple function applying inversion sampling
(see~\autoref{alg:samplechi}) can be instantiated using precomputed tables $T_\chi$. Moreover, due to
the small sizes of these precomputed tables constant-time table lookups, needed to protect against attacks
based on timing differences, can be implemented almost for free in terms of effort and performance impact.
\fi

All our implementations avoid the use of secret address accesses and secret branches and, hence,
are protected against timing and cache attacks at the software level.


\subsubsection{Performance analysis on x64 Intel}\label{sec:results_x64}

\ifshoworiginal
In this section, we summarize the results of our performance evaluation using a machine equipped with a 3.4GHz Intel Core i7-6700 (Skylake) processor and running Ubuntu 16.04.3 LTS. As standard practice, TurboBoost was disabled during the tests. For compilation we used GNU GCC version 7.2.0 with the command {\tt gcc -O3 -march=native}. 
The generation of the matrix $\bfA$ is the most expensive part of the computation. As described in \autoref{sec:spec:genA}, we support two ways of generating $\bfA$: one using $\AESOneTwoEight$ and one using $\SHAKE128$.


\subsubsection{Performance using AES128}\label{sec:results_aes}

\autoref{tab:results_x64_aes} details the performance of the optimized implementations and the additional x64 implementations when using $\AESOneTwoEight$ for the generation of the matrix $\bfA$. The top two sets of results correspond to performance when using OpenSSL's AES implementation\footnote{Note that in order to enable AES-NI instructions in OpenSSL, we use the \texttt{EVP\_aes\_128\_ecb} interface in OpenSSL.} and the bottom set presents the results when using a standalone AES implementation using Intel's Advanced Encryption Standard New Instructions (AES-NI).

As can be observed, the different implementation variants have similar performance, even when using hand-optimized AVX2 intrinsic instructions. This illustrates that $\FrodoKEM$'s algorithms, which are mainly based on matrix operations, facilitate automatic parallelization using vector instructions. Hence, the compiler is able to achieve close to ``optimal'' performance with little intervention from the programmer.
The best results for $\FrodoKEMLOneAES$, $\FrodoKEMLThreeAES$ and $\FrodoKEMLFiveAES$ (i.e., 0.93~ms, 1.75~ms and 2.91~ms, respectively, obtained by adding the times for encapsulation and decapsulation) are achieved by the additional implementation using AVX2 intrinsics. However, the difference in performance between the different implementations reported in \autoref{tab:results_x64_aes} is, in all the cases, less than~0.5\%.

We note that the performance of $\FrodoKEM$ using AES on Intel platforms greatly depends on AES-NI instructions. For example, when turning off the use of these instructions the computing cost of the optimized implementation of $\FrodoKEMLOneAES$ (resp.,~$\FrodoKEMLThreeAES$) is 26.4~ms (resp.,~60.9~ms), which is roughly a 28-fold (resp.,~35-fold) degradation in performance.


\begin{table}[t]
\caption{\textbf{Performance (in thousands of cycles) of $\FrodoKEM$ on a 3.4GHz Intel Core i7-6700 (Skylake) processor with matrix $\bfA$ generated using $\AESOneTwoEight$.} Results are reported using OpenSSL's AES implementation and using a standalone AES implementation, all of which exploit AES-NI instructions. Cycle counts are rounded to the nearest $10^3$ cycles.
\patrick{POSSIBLY, UPDATE RESULTS.}}\label{tab:results_x64_aes}
\medskip
\centering
\renewcommand{\tabcolsep}{0.4cm}
\renewcommand{\arraystretch}{1.1}
\begin{tabular}{l|c c c|c}
\toprule
\multirow{2}{*}{\textbf{Scheme}}     &     \multirow{2}{*}{\textbf{KeyGen}}      &    \multirow{2}{*}{\textbf{Encaps}}   &    \multirow{2}{*}{\textbf{Decaps}}   &    \textbf{Total}        \\ 
                                                       &                                                            &                                                         &                                                       &    \textbf{(Encaps + Decaps)}   \\
\midrule
\multicolumn{5}{l}{\bf Optimized Implementation (AES from OpenSSL)} \\
\midrule
$\FrodoKEMLOneAES$                               &            1,389                &            1,637                   &                 1,534         &                3,171             \\
$\FrodoKEMLThreeAES$                             &            2,831                &            3,047                   &                 2,904       &                5,951             \\
$\FrodoKEMLFiveAES$                             &            4,775                &            5,063                   &                 4,840       &                 9,903             \\
\midrule
\multicolumn{5}{l}{\bf Additional implementation using AVX2 intrinsic instructions (AES from OpenSSL)} \\
\midrule
$\FrodoKEMLOneAES$                               &            1,387                &            1,634                   &                 1,531       &                3,165             \\
$\FrodoKEMLThreeAES$                             &            2,846                &            3,047                   &                2,894       &                5,941             \\
$\FrodoKEMLFiveAES$                             &            4,779                &            5,051                   &                 4,849       &                 9,900             \\
\midrule
\multicolumn{5}{l}{\bf Additional implementation using AVX2 intrinsic instructions (standalone AES)} \\
\midrule
$\FrodoKEMLOneAES$                               &            1,398                &            1,644                   &                 1,540       &                3,184             \\
$\FrodoKEMLThreeAES$                             &            2,874                &            3,054                   &                 2,908       &                5,962             \\
$\FrodoKEMLFiveAES$                             &            4,765                &            5,069                   &                 4,867       &                9,936             \\
\bottomrule
\end{tabular}
\end{table}


\subsubsection{Performance using SHAKE128}\label{sec:results_shake}

\autoref{tab:results_x64_shake} outlines the performance figures of the optimized implementations and the additional x64 implementations when using $\SHAKE128$ for the generation of the matrix $\bfA$. 
The top set of results shows the performance of the optimized implementation written in C only, while the bottom set presents the results when using a 4-way implementation of SHAKE using AVX2 instructions (``SHAKE4x using AVX2'').
Note that the use of such a vectorized implementation of SHAKE is necessary to boost the practical performance. In our use-case, it results in a two-fold speedup when compared to the version using a SHAKE implementation written in plain C.

Comparing \autoref{tab:results_x64_aes} and \autoref{tab:results_x64_shake}, $\FrodoKEM$ using AES, when implemented with AES-NI instructions, is around 2.6--3.1$\times$ faster than the vectorized SHAKE implementation. Nevertheless, this comparative result could change drastically if hardware-accelerated instructions such as AES-NI are not available on the targeted platform, or if support for hardware-accelerated instructions for SHA-3 is added in the future. 


\begin{table}[t]
\caption{\textbf{Performance (in thousands of cycles) of $\FrodoKEM$ on a 3.4GHz Intel Core i7-6700 (Skylake) processor with matrix $\bfA$ generated using $\SHAKE128$.} Results are reported for two test cases: (i) using a SHAKE implementation written in plain C and, (ii) using a 4-way implementation of SHAKE using AVX2 instructions. Cycle counts are rounded to the nearest $10^3$ cycles.}\label{tab:results_x64_shake}
\medskip
\centering
\renewcommand{\tabcolsep}{0.4cm}
\renewcommand{\arraystretch}{1.1}
\begin{tabular}{l|c c c|c}
\toprule
\multirow{2}{*}{\textbf{Scheme}} & \multirow{2}{*}{\textbf{KeyGen}} & \multirow{2}{*}{\textbf{Encaps}} & \multirow{2}{*}{\textbf{Decaps}} & \textbf{Total}             \\ 
                                 &                                  &                                  &                                  & \textbf{(Encaps + Decaps)} \\ 
\midrule
\multicolumn{5}{l}{\bf Optimized Implementation (plain C SHAKE)} \\
\midrule
$\FrodoKEMLOneSHAKE$            &           7,649                  &           7,847                  &                7,743             &              15,590        \\
$\FrodoKEMLThreeSHAKE$          &          16,874                  &          17,101                  &               16,971           &              34,072        \\
$\FrodoKEMLFiveSHAKE$          &          30,465                  &          30,626                  &               30,451             &              61,077        \\
\midrule
\multicolumn{5}{l}{\bf Additional implementation using AVX2 intrinsics (SHAKE4x using AVX2)} \\
\midrule
$\FrodoKEMLOneSHAKE$            &           4,031                  &           4,218                  &                4,116             &               8,334        \\
$\FrodoKEMLThreeSHAKE$          &           8,599                  &           8,799                  &                8,659             &              17,458        \\
$\FrodoKEMLFiveSHAKE$          &          15,067                  &          15,338                  &               15,170             &              30,508        \\
\bottomrule
\end{tabular}
\end{table}
\fi

\autoref{tab:results_x64} summarizes the results of our performance evaluation using a machine equipped
with a 3.2GHz Intel Core i7-8700 (Coffee Lake) processor and running Ubuntu 22.04.2 LTS. Following standard practice,
TurboBoost was disabled during the tests. For compilation we used GNU GCC version 15.0.1 with the command
{\tt gcc -O3 -march=native}. 
%The generation of the matrix $\bfA$ is the most expensive part of the computation. As described in
%\autoref{sec:spec:genA}, we support two ways of generating $\bfA$: one using $\AESOneTwoEight$ and
%one using $\SHAKE128$.

For the case of generating the matrix $\bfA$ using $\AES128$, we present the results when using
an AES implementation that exploits the cryptographic extension set AES-NI.
The corresponding running times for $\FrodoKEMLOneAES$, $\FrodoKEMLThreeAES$ and $\FrodoKEMLFiveAES$ are
0.67~ms, 1.28~ms and 2.17~ms, respectively, obtained by adding the times for encapsulation and decapsulation.
This performance is expected to be typical in static key exchange applications where the cost of key generation
is amortized across many key encapsulation executions. 
For the full KEM, the running times are 0.97~ms, 1.91~ms and 3.22~ms, respectively. 
These timings roughly match the cost of \eFrodoKEM in an ephemeral setting (the overhead is about 1\% or less).

Our implementation also includes the optional use of AVX2 intrinsic instructions. In our experiments, we observed that
this optimization offers a very small performance improvement compared to the plain C implementation. 
This illustrates that $\FrodoKEM$'s algorithms, which are mainly based on matrix operations, facilitate
automatic parallelization using vector instructions. Hence, the compiler is able to achieve close to ``optimal''
performance with little intervention from the programmer.

We note that the performance of $\FrodoKEM$ using AES on Intel platforms greatly depends on AES-NI instructions.
For example, when turning off the use of these instructions the computing cost of the optimized implementations
suffers a more than 20-fold increase.

\autoref{tab:results_x64} also outlines the performance figures of our implementation when using $\SHAKE128$
for the generation of $\bfA$. 
In this case, we use a 4-way implementation of SHAKE that exploits AVX2 instructions.
In our tests, we observed that this approach results in a two-fold speedup when compared to a version using a SHAKE implementation
written in plain C.  
%The middle set of results shows the performance of the implementation written in C only, while the bottom
%set presents the results when using a 4-way implementation of SHAKE using AVX2 instructions.
%Note that the use of such a vectorized implementation of SHAKE is necessary to boost the practical performance.
%In our use-case, it results in a two-fold speedup when compared to the version using a SHAKE implementation
%written in plain C.

Comparing the use of $\AESOneTwoEight$ and $\SHAKE128$, $\FrodoKEM$ using AES, when implemented with AES-NI instructions,
is around 3$\times$ faster than $\FrodoKEM$ using SHAKE with a vectorized implementation. Nevertheless, this comparative
result may change drastically if hardware-accelerated instructions such as AES-NI are not available on the
targeted platform, or if support for hardware-accelerated instructions for SHA-3 is added in the future. 

\begin{table}[t]
\caption{Performance (in thousands of cycles) of $\FrodoKEM$ on a 3.2GHz Intel Core i7-8700 (Coffee Lake) processor.
For the variants using $\AESOneTwoEight$, results are reported using an AES implementation that exploits AES-NI instructions. 
For the variants using $\SHAKE128$, results are reported using a 4-way vectorized implementation of SHAKE using AVX2 instructions.
Cycle counts are rounded to the nearest $10^3$ cycles.}\label{tab:results_x64}
\cpnote{see about using siunitx to get nicer display and alignment of numbers here and in other tables}
\medskip
\centering
\renewcommand{\tabcolsep}{0.25cm}
\renewcommand{\arraystretch}{1.1}
\begin{tabular}{l|c c c|c}
\toprule
\multirow{2}{*}{\textbf{Scheme}}     &     \multirow{2}{*}{\textbf{KeyGen}}      &    \multirow{2}{*}{\textbf{Encaps}}   &    \multirow{2}{*}{\textbf{Decaps}}   &    \textbf{Total}        \\ 
                                   &                                             &                                       &
                                   &    \textbf{(Encaps + Decaps)}   \\
\midrule
\multicolumn{5}{l}{\bf AES using AES-NI} \\
\midrule
$\FrodoKEMLOneAES$                               &            938                &            1,105                   &                 1,044         &                2,149             \\
$\FrodoKEMLThreeAES$                             &            2,017                &            2,105                   &                 1,983       &                4,088             \\
$\FrodoKEMLFiveAES$                             &            3,353                &            3,597                   &                 3,326       &                 6,923             \\
%\midrule
%\multicolumn{5}{l}{\bf Plain C SHAKE} \\
%\midrule
%$\FrodoKEMLOneSHAKE$            &           7,649                  &           7,847                  &                7,743             &              15,590        \\
%$\FrodoKEMLThreeSHAKE$          &          16,874                  &          17,101                  &               16,971           &              34,072        \\
%$\FrodoKEMLFiveSHAKE$          &          30,465                  &          30,626                  &               30,451             &              61,077        \\
\midrule
\multicolumn{5}{l}{\bf Vectorized SHAKE using AVX2} \\
\midrule
$\FrodoKEMLOneSHAKE$            &           2,806                  &           2,941                  &                2,877             &               5,818        \\
$\FrodoKEMLThreeSHAKE$          &           6,026                  &           6,096                  &                5,994             &              12,090        \\
$\FrodoKEMLFiveSHAKE$          &          10,725                  &          10,643                &               10,497             &              21,140        \\
\bottomrule
\end{tabular}
\end{table}


\subsubsection{Performance analysis on ARM}\label{sec:results_arm}

\ifshoworiginal
In this section, we summarize the results of our performance evaluation using a device powered by a 1.992GHz 64-bit ARM Cortex-A72 (ARMv8) processor and running Ubuntu 16.04.2 LTS. For compilation we used GNU GCC version 5.4.0 with the command {\tt gcc -O3 -march=native}. 

\autoref{tab:results_arm} details the performance of the optimized implementations when using $\AESOneTwoEight$ and $\SHAKE128$. Similar to the case of the x64 Intel platform, the overall performance of $\FrodoKEM$ is highly dependent on the performance of the primitive that is used for the generation of the matrix $\bfA$. Hence, the best performance in this case is achieved when using OpenSSL's AES implementation, which exploits the efficient NEON engine. On the other hand, $\SHAKE$ performs significantly better when there is no support for specialized instructions in the targeted platform: using a plain C version of $\SHAKE$ is more than 3 times faster than using a plain C version of AES.

\else

\autoref{tab:results_arm} details the performance of our implementations on a device powered by a 1.992GHz 64-bit ARM Cortex-A72 (ARMv8)
processor and running Ubuntu 24.04.2 LTS.
We provide three options which, again, are determined by the way we generate matrix $\bfA$. The first option uses OpenSSL's AES engine for implementing and accelerating $\AES128$ with the AES cryptographic extensions available on the targeted ARMv8 processor. This implementation uses OpenSSL version 3.0.13 and was compiled with GNU GCC version 14.2.0.
The other two options use plain C implementations of AES and SHAKE for generating $\bfA$ with $\AES128$ and $\SHAKE128$, respectively. 
For these cases, we use OpenSSL version 3.0.2 and compiled the implementations with GNU GCC version 13.1.0.
For all the options we used the command {\tt gcc -O3 -march=native}. 
Similar to the case of the x64 Intel platform, the overall performance of $\FrodoKEM$ is highly
dependent on the performance of the primitive that is used for the generation of the matrix $\bfA$.
Hence, the best performance in this case is achieved when using an AES implementation that
exploits the hardware acceleration provided by the ARMv8 cryptographic extensions.
The respective running times for the full KEM are 4.98~ms, 9.99~ms and 17.24~ms for $\FrodoKEMLOneAES$, $\FrodoKEMLThreeAES$ and $\FrodoKEMLFiveAES$, respectively
(as above, these timings only have a negligible overhead in comparison to the cost of \eFrodoKEM in an ephemeral setting).
On the other hand, $\SHAKE$ performs significantly better when
there is no support for specialized instructions in the targeted platform: using a plain C version
of $\SHAKE$ is more than 3$\times$ faster than using a plain C version of AES.
\fi

\begin{table}[t]
\caption{Performance (in thousands of cycles) of $\FrodoKEM$ on a 1.992GHz 64-bit ARM Cortex-A72 (ARMv8) processor. Results are reported for three test cases: (i) using an AES implementation exploiting cryptographic extensions, (ii) using an AES implementation written in plain C, and (iii) using a $\SHAKE$ implementation written in plain C. Results have been scaled to cycles using the nominal processor frequency. Cycle counts are rounded to the nearest $10^3$ cycles.}\label{tab:results_arm}
\medskip
\centering
\renewcommand{\tabcolsep}{0.25cm}
\renewcommand{\arraystretch}{1.1}
\begin{tabular}{l|c c c|c}
\toprule
\multirow{2}{*}{\textbf{Scheme}}     &     \multirow{2}{*}{\textbf{KeyGen}}      &    \multirow{2}{*}{\textbf{Encaps}}   &    \multirow{2}{*}{\textbf{Decaps}}   &    \textbf{Total}        \\ 
                                                       &                                                            &                                                         &                                                       &    (Encaps + Decaps)   \\
\midrule
\multicolumn{5}{l}{\bf AES using cryptographic extensions} \\
\midrule
$\FrodoKEMLOneAES$                               &            3,109                &            3,402                   &                 3,414         &                6,816             \\
$\FrodoKEMLThreeAES$                             &            6,515                &            6,702                   &                6,678       &                13,380             \\
$\FrodoKEMLFiveAES$                             &            11,010                &            11,796                   &                11,527       &                23,323             \\
\midrule
\multicolumn{5}{l}{\bf Plain C AES} \\
\midrule
$\FrodoKEMLOneAES$                               &            47,776                &           48,007                   &               47,958       &              95,965             \\
$\FrodoKEMLThreeAES$                             &         109,922                &         110,695                   &             110,387       &               221,082             \\
$\FrodoKEMLFiveAES$                             &            207,752                &           209,724                   &                209,164       &               418,888             \\
\midrule
\multicolumn{5}{l}{\bf Plain C SHAKE} \\
\midrule
$\FrodoKEMLOneSHAKE$                               &            11,902                &           12,197                   &               12,219       &               24,416             \\
$\FrodoKEMLThreeSHAKE$                             &            26,334                &           26,618                   &               26,713       &               53,331             \\
$\FrodoKEMLFiveSHAKE$                             &            47,506                &            48,169                   &                48,505       &                96,674             \\
\bottomrule
\end{tabular}
\end{table}


\subsection{Comparison with other algorithms}\label{sec:comparison}

In this section, we compare the performance profile of \FrodoKEM with two other quantum-safe KEMs
that are also expected to be deployed and adopted for real-world applications:
CRYSTALS-Kyber~\cite{EuroSP:Kyber}, which was recently standardized by NIST as ML-KEM~\cite{MLKEM}, and
Classic McEliece~\cite{CME}, which is a Round 4 candidate in the NIST Post-Quantum Cryptography standardization project~\cite{NIST17}. 
As stated before, these two KEMs, alongside \FrodoKEM, are currently undergoing standardization by ISO.


\begin{table}[t]
\caption{Performance comparison of \FrodoKEM with two other algorithms: CRYSTALS-Kyber and Classic McEliece.
The results for \FrodoKEM were obtained on a 3.2GHz Intel Core i7-8700 (Coffee Lake) processor,
while the results for CRYSTALS-Kyber and Classic McEliece are taken from~\cite{SUPERCOP} (version {\tt supercop-20241022}), corresponding to a
3.0GHz Intel Core i3-8109U (Coffee Lake) processor.
For \FrodoKEM we present results for the variant using $\AESOneTwoEight$, and for Classic McEliece we use the fast (f) variants. 
Cycle counts are rounded to the nearest $10^3$ cycles.}\label{tab:comparison}
\scriptsize
\medskip
\centering
\renewcommand{\tabcolsep}{0.25cm}
\renewcommand{\arraystretch}{1.1}
\begin{tabular}{l|r r|r r r|r}
\toprule
\multirow{2}{*}{\textbf{Scheme}}     & \multicolumn{2}{c|}{\textbf{Sizes (in bytes)}} &     \multicolumn{3}{c|}{\textbf{Cycles ($\times 10^3$)}}                     &    \multirow{2}{*}{\textbf{Total}}  \\ 
\cmidrule(lr){2-3} \cmidrule(lr){4-6}
                                     & \textbf{public key} & \textbf{ciphertext}      &      \textbf{KeyGen}       &    \textbf{Encaps}     &    \textbf{Decaps}     &                                     \\
\midrule
\multicolumn{7}{l}{\bf NIST security level 1} \\
\midrule
Kyber512                             &       800  &    768          &             23                &               36                   &                    28         &                   87             \\
$\FrodoKEMLOneAES$                   &     9,616  &  9,752          &            957                &            1,134                   &                 1,059         &                3,150             \\
{\tt mceliece348864f}                &   261,120  &     96          &         30,381                &               30                   &                   118         &               30,529             \\
\midrule
\multicolumn{7}{l}{\bf NIST security level 3} \\
\midrule
Kyber678                             &     1,184  &  1,088          &               39              &               53                   &                    42         &                  134             \\
$\FrodoKEMLThreeAES$                 &    15,632  & 15,792          &            2,043              &            2,125                   &                 2,042         &                6,210             \\
{\tt mceliece460896f}                &   524,160  &    156          &           97,899              &               64                   &                   245         &               98,208             \\
\midrule
\multicolumn{7}{l}{\bf NIST security level 5} \\
\midrule
Kyber1024                            &     1,568  &  1,568          &               55              &               75                   &                    62         &                  192             \\
$\FrodoKEMLFiveAES$                  &    21,520  & 21,696          &            3,432              &            3,609                   &                 3,471         &               10,512             \\
{\tt mceliece6960119f}               & 1,047,319  &    194          &          195,984              &              120                   &                   287         &              196,391             \\
\bottomrule
\end{tabular}
\end{table}
   

\autoref{tab:comparison} illustrates that CRYSTALS-Kyber offers the best performance in terms of both speed and bandwidth. 
\FrodoKEM's sizes and runtimes are, across the board, more than an order of magnitude larger and slower, respectively, compared to Kyber's.
These results were key in motivating NIST's selection of CRYSTALS-Kyber as the preferred drop-in replacement for most applications.

However, for security-sensitive applications, it can be argued that Classic McEliece provides a more relevant comparison to \FrodoKEM.
In this case, the significantly larger public key sizes of Classic McEliece, which exceed \FrodoKEM's by more than an order of magnitude,
may render it impractical for many use cases.
Similarly, the high computational cost of the full Classic McEliece protocol presents additional challenges.
For example, the runtime of Classic McEliece at NIST level 5 is approximately 65.5~ms on a server-class processor (compare to \FrodoKEM's 3.29~ms). 
Nonetheless, Classic McEliece offers an advantage in certain static key exchange scenarios in which its substantial key generation cost
can be amortized over multiple key encapsulation executions.


%%% Local Variables:
%%% mode: latex
%%% TeX-master: "Main"
%%% End:



%\section{Bibliography}
%Citing papers is done in the usual way using BibTeX or \texttt{biblatex}
%commands. For example: the RSA paper~\cite{RSA78}.

%It is highly encouraged to use CryptoBib from \url{https://cryptobib.di.ens.fr}

% This sample uses bibtex rather than biblatex.
\bibliography{biblio, refs, ../cryptobib/abbrev1, ../cryptobib/crypto}

% NOTES
% - Download abbrev3.bib and crypto.bib from https://cryptobib.di.ens.fr/
% - Use bilbio.bib for additional references not in the cryptobib database.
%   If possible, take them from DBLP.

%%%%%%%%%%%%%%%%%%%%%%%%%%%%%%%%%%%%%%%%%%%%%%%%%%%%%%%%%%%%%%%%%%%%%%%%%%%%%%%%%%%%%%%%%%%%%%%%%%%%%%%%%%%%%%%%%%%%%%%%%%%%%%%%
\appendix

\section{Security reductions}%
\label{sec:Security_appendix}

$\FrodoKEM$ depends on the hardness of plain learning with errors. A summary of the reductions supporting the security of $\FrodoKEM$ is
as follows:

\begin{enumerate}
\item $\FrodoKEM$, using the concrete error distributions
  $\chi_\Frodo$ specified in \autoref{tab:distribution}, is an
  \INDCCA-secure KEM against classical attacks in the classical random
  oracle model, under the assumption that $\FrodoPKE$ using a rounded
  Gaussian error distribution is an \INDCPA-secure public-key
  encryption scheme against classical attacks.  This is
  \autoref{thm:cca-kem-to-cpa-pke-rom-parameterized}, and the
  reduction is tight.  The argument combines results
  of~\cite{TCC:HofHovKil17} on the modular FO transform with results
  of~\cite{EC:LanSteSte14} on \renyi divergence. We also note that the
  same conclusion follows from the assumption that $\FrodoPKE$ using
  the distributions~$\chi_{\Frodo}$ is \INDCPA secure, using the same
  proof but without any analysis of \renyi divergence. 

\item $\FrodoKEM$, using any error distribution, is an \INDCCA-secure
  KEM against quantum attackers in the quantum random oracle model,
  under the assumption that $\FrodoPKE$ using the same error
  distribution is an \OWCPA-secure public-key encryption scheme
  against quantum attackers.  This is
  \autoref{thm:cca-kem-to-cpa-pke-qrom}, and the reduction is
  non-tight.  We view this theorem as giving support for the security
  of general constructions of LWE-based KEMs in the style of FrodoKEM
  against quantum adversaries, but it does not concretely support the
  bit-security of the six \FrodoKEM instantiations in this document,
  which is why we omit the corresponding column from
  \autoref{tab:security}.
\item $\FrodoKEM$ is an \MINDCCA secure key encapsulation mechanism in the ROM and the QROM for $n_j$ as large as $2^{64}$ for any user $j$, under the assumption that $\FrodoPKE$ is a \MINDCPA secure public-key encryption scheme. The argument is found in \cite{GlabushThesis} and \cite{Multi-challenge}.
\item Changing the distribution of matrix $\bfA$ from a truly uniform
  distribution to one generated from a public random seed in a
  pseudorandom fashion does not affect the security of $\FrodoKEM$ or
  $\FrodoPKE$, provided that the pseudorandom generator is modeled
  either as an ideal cipher (when using $\AESOneTwoEight$) or a random
  oracle (when using $\SHAKE128$). This is shown in
  \autoref{sec:strength:pseudorandom-A}.

\item $\FrodoPKE$, using any error distribution and a uniformly random
  $\bfA$, is an \MINDCPA-secure public-key encryption scheme under the
  assumption that the uniform-secret learning with errors decision
  problem is hard for the same parameters (except for a small additive
  loss in the number of samples), for either classical or quantum
  adversaries.  This is a consequence of
  \autoref{thm:cpa-pke-to-nf-dlwe} and \autoref{thm:nf-dlwe-to-dlwe},
  and the result is tight.

\item The uniform-secret learning with errors decision problem, using
  a rounded Gaussian distribution with parameter $\sigma$ from
  \autoref{tab:distribution} and an appropriate bound on the number of
  samples, is hard under the assumption that the \emph{worst-case}
  bounded-distance decoding with discrete Gaussian samples problem
  (\BDDwDGS, \autoref{def:BDDwDGS}) is hard for related parameters.
  \autoref{thm:bddwdgs-to-dlwe} gives a non-tight classical reduction
  against classical or quantum adversaries (in the standard model).
\end{enumerate}

\subsection{\MINDCCA security reduction}%
\label{sec:rom-mindcca}


Here, we give a detailed description of the proof steps for \autoref{thm:cca-kem-to-cpa-pke-rom-parameterized}.

\paragraph{Step 1: \MINDCPA $\PKE$ to \MOWPCA deterministic $\PKEOne$.}

For completeness, we recall the definition of \MOWPCA, following the
presentation of Hofheinz et al.~\cite{TCC:HofHovKil17}, and extended in \cite{GlabushThesis}.

\begin{definition}[\MOWPCA for PKE~\cite{RSA:OkaPoi01}]%
  \label{def:OW-CPA}
  Let \PKE be a public-key encryption scheme with message space
  $\MsgSp$ and let~$\Adversary$ be an algorithm.  The \MOWPCA security
  experiment for $\Adversary$ attacking \PKE is
  $\Exp{\MOWPCA}{\PKE}(\Adversary)$ from \autoref{fig:owpca}.  The
  advantage of~$\Adversary$ in the experiment is
  \[ \Adv{\MOWPCA}{\PKE}(\Adversary) :=  \Pr\left[ \Exp{\MOWPCA}{\PKE}(\Adversary) \Rightarrow 1 \right]  . \]
\end{definition}

\begin{figure}[h]
	\centering
	\fbox{
		\begin{minipage}[t]{0.4\textwidth}
			\underline{Experiment $\Exp{\MOWPCA}{\PKE}(\Adversary)$:}
			\vspace{-1em}
			\begin{algorithmic}[1]
                    \FOR {$ j = 1,..., u$} \STATE{ $(\pk_j, \sk_j) \gets \PKE.\KeyGen()$} \ENDFOR
                    \STATE $\Vec{\pk} = (\pk_1,...,\pk_u)$          
				\STATE $m' \gets
                                \Adversary^{\OPco(\cdot, \cdot), \challoracle(\cdot)}(\pk, c^*)$
				\RETURN $\OPco(m', c^*)$
			\end{algorithmic}
		\end{minipage}
		~
		\begin{minipage}[t]{0.4\textwidth}
			\underline{Oracle $\OPco(m, c)$:}
			\vspace{-1em}
			\begin{algorithmic}[1]
				\IF {$\PKE.\Dec(\sk, c)=m$}
				\RETURN 1
				\ELSE
				\RETURN 0
				\ENDIF
			\end{algorithmic}
                \underline{Oracle $\challoracle(j):$}
                \vspace{-1em}
                \begin{algorithmic}
                \STATE $m\getsr \MsgSp$
			\STATE $c^* \gets \PKE.\Enc(m, \pk)$
                \end{algorithmic}
		\end{minipage}
%		~

	}
	\caption{Security experiment for \OWPCA.}
	\label{fig:owpca}
\end{figure}

\begin{figure}[h]
\centering
\fbox{
\begin{minipage}[t]{0.35\textwidth}
\underline{$\PKEOne.\KeyGen()$:}
\vspace{-1em}
\begin{algorithmic}[1]
\RETURN $\PKE.\KeyGen()$
\end{algorithmic}

\medskip

\underline{$\PKEOne.\Enc(\mu, \pk)$:}
\vspace{-1em}
\begin{algorithmic}[1]
\STATE $\salt \getsr \{0,1\}^{\lengthsalt}$
\STATE $r \gets G_2(\mu\|\salt)$
\STATE $c \gets \PKE.\Enc(\mu, \pk; r)$
\RETURN $c\|\salt$
\end{algorithmic}
\end{minipage}
~
\begin{minipage}[t]{0.6\textwidth}
\underline{$\PKEOne.\Dec(c\|\salt, \sk)$:}
%\vspace{-1em}
\begin{algorithmic}[1]
\STATE $\mu' \gets \PKE.\Dec(c, \sk)$
\IF {$\mu' = \bot$ or $c \ne \PKE.\Enc(\mu', \pk; G_2(\mu'\|\salt))$}
  \RETURN $\bot$
\ELSE
  \RETURN $\mu'$
\ENDIF
\end{algorithmic}
\end{minipage}
}
\caption{Construction of deterministic public-key encryption scheme $\PKEOne=\ST[\PKE, G_2]$ from a public-key encryption scheme $\PKE$ and hash function $G_2$.}
\label{fig:ST}
\end{figure}
The $\ST$ transform converts a public-key encryption scheme
$\PKE$ to a deterministic public-key encryption scheme $\PKEOne$; see
\autoref{fig:ST}. Glabush's Theorem 4.2.3 tightly establishes the
$\OWPCVA$-security of $\PKEOne$ under, among others, the assumption
that $\PKE$ is \INDCPA secure and $\gamma$-spread. (In the \OWPCVA
security game, the attacker additionally has a ciphertext-validity
oracle, which checks whether a queried ciphertext has a valid
decryption.)  These results were adapted to the multi-challenge setting in \cite{GlabushThesis}. However, they note that $\MOWPCA$ security follows
(tightly) \emph{without} the $\gamma$-spread assumption, because in
the security bounds $\gamma$-spreadness is relevant only to
ciphertext-validity queries.  We state that adapted version here.

\begin{lemma}[\cite{TCC:HofHovKil17, GlabushThesis}, Theorem~4.2.3, $\MOWPCA$ version]
  \label{lem:ST}
  Let $\PKE$ be a $\delta$-correct public-key encryption scheme with
  message space $\MsgSp$. For any $\OWPCA$ adversary $\Adversary$ that
  issues at most~$q_G$ queries to the random oracle~$G_2$ and~$q_P$
  queries to the plaintext-checking oracle, there exists an $\INDCPA$
  adversary $\Bdversary$ such that
  \[
  \Adv{\MOWPCA}{\PKEOne}(\Adversary) \leq q_G \cdot \delta(u) + \frac{2 q_G
    + 1}{|\MsgSp||\saltlist|} + 3 \cdot \Adv{\MINDCPA}{\PKE}(\Bdversary) \enspace
  ,
  \]
  and the running time of $\Bdversary$ is about that of $\Adversary$
  plus the time needed to simulate the random oracle.
\end{lemma}

\noindent It is straightforward to verify from the proof that
$\Bdversary$ uses $\Adversary$ solely as a ``black box'' subroutine.

\paragraph{Step 2: Approximating the error distribution.}

The rounded Gaussian distribution (\autoref{def:rounded-gaussian}),
which is important to the worst-case-to-average-case reduction, is
difficult to sample on a finite computer (and impossible to sample in
constant time). Following Langlois et al.~\cite{EC:LanSteSte14}, we
replace this infinite-precision distribution with a finite
approximation, and quantify the \MOWPCA security loss using their
\renyi divergence.



\begin{definition}[\renyi divergence]
  \label{def:renyi}
  The \renyi divergence of positive order $\alpha \neq 1$ of a
  discrete distribution~$P$ from a distribution~$Q$ is defined as
  \[
    \RD_\alpha(P\|Q)=\frac1{\alpha-1}\ln \parens*{
    \sum_{x\in\mathop{\mathrm{supp}} P} P(x)\left(
      \frac{P(x)}{Q(x)}\right)^{\alpha-1}} \enspace .
  \]
\end{definition}

Note that our definition differs from that of~\cite{EC:LanSteSte14} in
that we take the logarithm of the sum, and that \renyi divergence is
not symmetric.  The following result relates probabilities of a
certain event occurring under two distributions as a function of their
\renyi divergence.

\cpnote{There is some odd spacing after the parenthesis in these lemmas that cite other works; see if we can figure out how to fix it.}

\begin{lemma}[{{\cite[Lemma 4.1]{EC:LanSteSte14}}}]%
  \label{lem:renyi}
  Let $S$ be an event defined in a probabilistic experiment $G_Q$ in
  which~$s$ samples are drawn from distribution $Q$. Then the
  probability that $S$ occurs in the same experiment but with~$Q$
  replaced by~$P$ is bounded as follows:
  \begin{equation}
    \label{eqn:renyi_loss}
    \Pr[G_P(S)] \leq \left( \Pr[G_Q(S)] \cdot \exp(s \cdot
      \RD_\alpha(P\|Q)) \right)^{1-1/\alpha}.
  \end{equation}
\end{lemma}

It immediately follows that reductions from any \emph{search} problem,
such as the one represented by the $\MOWPCA$ game, are preserved up to
the relaxation in \eqref{eqn:renyi_loss}. For any given security
relationship, and any concrete choice of the two distributions~$P$
and~$Q$, the loss can be minimized by choosing an optimal value of the
order~$\alpha$.

\begin{corollary}[Distribution substitution for \MOWPCA]
  \label{cor:sbst-owpca}
  Let $\PKE_X$ be a public-key encryption scheme that is parameterized
  by a distribution~$X$, and let~$s$ be an upper bound on the total
  number of samples drawn from~$X$ by $\PKE_X.\Enc$ and
  $\PKE_X.\KeyGen$ combined. Let $\Adversary$ be an \OWPCA adversary
  against $\PKE_X$, and let~$P$ and~$Q$ be discrete
  distributions. Then for any $\alpha > 1$,
  \[
    \Adv{\MOWPCA}{\PKE_P}(\Adversary) \leq
    \left( \Adv{\MOWPCA}{\PKE_Q}(\Adversary)\cdot
      \exp(s \cdot \RD_\alpha(P \| Q)) \right)^{1-1/\alpha} \enspace .
  \]
\end{corollary}

\begin{proof}
  This follows immediately from \autoref{lem:renyi}, with $S$ being
  the event that~$\Adversary$ ``wins'' the \OWPCA experiment from
  \autoref{fig:owpca}, i.e., causes it to output~$1$.
\end{proof}

We use \autoref{cor:sbst-owpca} to relate the \OWPCA security of
$\T[\FrodoPKE_{P},G_{2}]$ to the \OWPCA security of
$\T[\FrodoPKERGauss,G_{2}]$ where $\FrodoPKERGauss$ is the same as
$\FrodoPKE$ but with the error distribution $P=\chi_{\Frodo}$ replaced
by a rounded Gaussian distribution~$Q=\Psi$
(see \autoref{def:rounded-gaussian}).

\paragraph{Step 3: \MOWPCA deterministic $\PKEOne$ to \MINDCCA KEM.}
Hofheinz et al.~\cite{TCC:HofHovKil17} define the $\Unotperp$ transform from a deterministic public-key
encryption scheme $\PKEOne$ to a key encapsulation mechanism
$\KEMnotperp$; see \autoref{fig:Unotperp}. Hofheinz et al.'s Theorem~3.4 shows
the \INDCCA security of $\KEMnotperp = \Unotperp[\PKEOne,F]$ assuming
the \OWPCA security of the underlying $\PKEOne$. This result was adapted to the multi-challenge setting in Theorem 4.2.4 from~\cite{GlabushThesis}.
This result is stated below in \autoref{lem:Unotperp}. 

\begin{figure}[h]
\centering
\fbox{
\begin{minipage}[t]{0.4\textwidth}
\underline{$\KEMnotperp.\KeyGen()$:}
\vspace{-1em}
\begin{algorithmic}[1]
\STATE $(\pk, \sk) \getsr \PKEOne.\KeyGen()$
\STATE $s \getsr \MsgSp$
\STATE $\sk' \gets (\sk, s)$
\RETURN $(\pk, \sk')$
\end{algorithmic}

\medskip

\underline{$\KEMnotperp.\Encaps(\pk)$:}
\vspace{-1em}
\begin{algorithmic}[1]
\STATE $\salt \getsr \{0,1\}^{\lengthsalt}$
\STATE $c\|\salt \gets \PKEOne.\Enc(\mu, \pk)$
\STATE $ss \gets F(\mu, c\|\salt)$
\RETURN $(c\|\salt, ss)$
\end{algorithmic}
\end{minipage}
~
\begin{minipage}[t]{0.5\textwidth}
\underline{$\KEMnotperp.\Decaps(c\|\salt, (\sk, s))$:}
%\vspace{-1em}
\begin{algorithmic}[1]
\STATE $\mu' \gets \PKE.\Dec(c\|\salt, \sk)$
\IF {$\mu' \ne \bot$}
  \RETURN $ss' \gets F(\mu', c\|\salt)$
\ELSE
  \RETURN $ss' \gets F(s, c\|\salt)$
\ENDIF
\end{algorithmic}
\end{minipage}
}
\caption{Construction of key encapsulation mechanism $\KEMnotperp=\Unotperp[\PKEOne, F]$ from a deterministic public-key encryption scheme $\PKEOne$ and hash function $F$.}
\label{fig:Unotperp}
\end{figure}

\cpnote{Here's another.}

\begin{lemma}[{{\cite[Theorem~4.2.4]{GlabushThesis}}}]%
  \label{lem:Unotperp}
  Model~$F$ as a random oracle.  Then if $\PKEOne$ is
  $\delta_1$-correct, so is $\KEMnotperp$.  For any $\INDCCA$
  adversary $\Adversary$ against $\KEMnotperp$ issuing at
  most $q_F$ queries to~$F$, there exists an $\OWPCA$ adversary
  $\Bdversary$ against $\PKEOne$ that makes at most $q_F$ queries to
  its plaintext-checking oracle, and for which
  \[
    \Adv{\MINDCCA}{\KEMnotperp}(\Adversary) \leq \frac{n^2}{|\M||\saltlist|} + \frac{q_F}{|\MsgSp|} +
    \Adv{\MOWPCA}{\PKEOne}(\Bdversary) \enspace ,
  \]
  where the running time of $\Bdversary$ is about that of
  $\Adversary$, plus the time to simulate the random oracle and
  decapsulation queries.
\end{lemma}

It is straightforward to verify from the proof that $\Bdversary$ uses
$\Adversary$ solely as a ``black box'' subroutine.  Together,
\autoref{lem:ST}, \autoref{cor:sbst-owpca}, and \autoref{lem:Unotperp}
establish \autoref{thm:cca-kem-to-cpa-pke-rom-parameterized}.

\paragraph{Applying \autoref{thm:cca-kem-to-cpa-pke-rom-parameterized}.}
For an application of
\autoref{thm:cca-kem-to-cpa-pke-rom-parameterized} to our schemes,
consider the relation between the $\INDCCA$ security of \FrodoKEMLOne
and the \MINDCPA security of $\FrodoPKELOne_\Psi$, where the error
distribution of the latter is taken to be the rounded Gaussian
$\Psi_{2.8\cdot\sqrt{2\pi}}$ as defined in \autoref{sec:gaussians}.

To extract exact bounds on the \MINDCCA security (in the classical ROM)
of \FrodoKEMLOne, we make a number of assumptions about the underlying
cost model. Specifically,
\begin{itemize}
\item We ignore the overhead of running the reduction
  of \autoref{thm:cca-kem-to-cpa-pke-rom-parameterized}, including the
  cost of simulating random oracles.
\item The cost to the adversary of \emph{making} an oracle query is
  $2^{18}$ classical gates. This bound is based on the NIST Call for
  Proposals, Section 4.A.5, which estimates the cost of finding collisions
  in SHA-3 at all security levels. (Here we
  ignore the small performance differences between $\SHAKE128$,
  $\SHAKE256$, and SHA3-256.)
\item We interpret ``$b$ bits of
  classical security'' as a statement that the advantage in the
  corresponding game of a uniform $t$-gate classical adversary is
  bounded by $t/2^b$. For some tasks, such as collision finding,
  this upper bound can be quite loose for smaller values
  of~$t$ (and thus beneficial to the adversary).
\item The \MINDCPA security of $\FrodoPKELOne_\Psi$ is given by the
  smaller of the costs of the primal and dual attacks on the LWE
  problem (\autoref{tab:attacks}), discounted by the reduction factor
  of $\nbar + \mbar=16$ (\autoref{thm:cpa-pke-to-nf-dlwe}),
  yielding~$2^{-145.6}$.
\end{itemize}

Under these assumptions, if an adversary $\Bdversary$ has uniform gate
complexity $t$, then it has advantage
$\Adv{\indcpa}{\FrodoPKELOne_\Psi}(\Bdversary)$ bounded
by~$t\cdot 2^{-145.6}$.

The \renyi divergence of $\chi_\FrodoLOne$ from the rounded Gaussian
is $\RD_\alpha(\chi_\FrodoLOne\| \Psi_{2.8\cdot \sqrt{2\pi}}) \leq
0.0000324$ for $\alpha=200$ (\autoref{tab:distribution}). The number of
samples drawn from the error distribution by $\FrodoPKE.\KeyGen$ is
$2n\nbar$, and by $\FrodoPKE.\Enc$ is $2\mbar n+\mbar \nbar$, which
for $n=640$ and $\mbar=\nbar=8$ totals
$s=2\times(8 + 8)\times 640+64=20544$.

Substituting $\qro < t\cdot 2^{-18}$, $|\MsgSp|=2^{128}$ and
$\delta<2^{-138.7}$ into \autoref{eq:cca-to-cpa-renyi-bound}, we can
bound the advantage in $\Exp{\indcca}{\FrodoKEMLOne}$ for an adversary
$\Adversary$ with gate count~$t \geq 1$ as follows:
\begin{align*}
  \Adv{\indcca}{\FrodoKEMLOne}(\Adversary)
  &< 2^{-146} \cdot t + \left((2.01 \cdot 2^{-146} + 3 \cdot
    2^{-145.6}) \cdot t \cdot \exp(20544\cdot 0.0000324)\right)^{0.995} \\
  &< 2^{-141.6} \cdot t \enspace .
\end{align*}
Similarly, computed bounds on the advantage of a classical \INDCCA
adversary for other parameter settings appear
in \autoref{tab:security}.

\subsubsection{Deterministic generation of $\bfA$}\label{sec:strength:pseudorandom-A}

\newcommand{\idx}{\ensuremath{\mathsf{idx}}}
\newcommand{\Idx}{\ensuremath{\mathrm{Idx}}}
\newcommand{\IC}{\ensuremath{\mathrm{IC}}}

The matrix $\bfA$ in $\FrodoKEM$ and $\FrodoPKE$ is  deterministically expanded from a short random seed in the function $\Frodo.\gen$ either using $\AESOneTwoEight$ or $\SHAKE128$. In order to
relate $\FrodoKEM$ and $\FrodoPKE$'s security to the hardness of the learning with errors
problem, we argue that we can replace a uniformly sampled $\bfA\in
\bbZ_q^{n\times n}$ with matrices sampled according to
$\Frodo.\gen$. Although the matrix appears pseudorandom under standard security assumptions to an adversary without access to the seed, we argue security of this step against a stronger (and more realistic) adversary via the indifferentiability framework
\cite{TCC:MauRenHol04, C:CDMP05}.

Informally, a construction $\calC$ with
access to an ideal primitive $\calG$ is said to be $\eps$-indifferentiable
from an ideal primitive $\calF$ if there exists a simulator $\calS$ such that
for any polynomial time distinguisher $\calD$ it holds that $\left| \Pr\left[
\calD^{\calC, \calG} = 1  \right] - \Pr\left[\calD^{\calF, \calS} = 1
\right] \right|< \eps$. An
indifferentiability argument implies that any cryptosystem secure in the
$\calF$-model remains secure (in a tight sense) in the $\calG$-model with
$\calF$ instantiated as $\calC^\calG$ \cite{TCC:MauRenHol04}. In what follows, we consider the ideal
primitive $\calF$ to be an ideal ``domain expansion'' function expanding a small
seed to a matrix $\bfA$. Critically, the security of the step depends on the properties of $\calG$ rather than randomness of the seed. The construction $\calC$ and primitive
$\calG$ depend on whether we use $\AESOneTwoEight$ or $\SHAKE128$, modeled below as an ideal cipher and an ideal extendable-output function (XOF) respectively.

\paragraph{Using AES128 to generate $\bfA$.}
\autoref{alg:genA_AES} generates the entries of $\bfA$ as $16$-bit values and then reduces each one modulo $q$. For simplicity, we assume that $\bfA$ consists of  $N=16n^2$ bits 
and we set $M = N/128$. This means that $\bfA$ consists of $M$ $128$-bit
$\AESOneTwoEight$ blocks. The pseudorandom bits in the $i$th block are
generated by encrypting a fixed index $\idx_i$ with a uniformly random $\seedA\in \{0,1\}^{128}$ as the key. Throughout, we refer to the set $\Idx := \left\{\idx_1,\ldots, \idx_M\right\}$ as the set of indices used in the pseudorandom generation of $\bfA$.

The ideal domain expansion primitive $\calF$ expands a uniformly random seed
$\seedA \in \{0,1\}^{128}$ to a larger bit string $s_1 \| s_2 \| \cdots \|
s_M \in \{0, 1\}^{128M}$ subject to the condition that $s_i \neq s_j$ for any
distinct pair of $i$, $j$. Observe that a uniformly sampled $\bfA$ satisfies
this condition with probability at least $1-M^2/2^{128}$. In our security
reductions, the matrix $\bfA$ is constructed through $m = n$ calls to the LWE
oracle (\autoref{def:dlweproblem}). By increasing the number of calls to this
oracle marginally, by setting $m = 1.01n > n \cdot (1-M^2/2^{128})^{-1}$, we
can construct an LWE matrix $\bfA$ sampled from the same distribution as the
output of $\calF$ with overwhelming probability without affecting its
underlying security.
\medskip

When $\Frodo.\gen$ uses $\AESOneTwoEight$, we consider a construction
$\calC^\calG$
in the Ideal Cipher model implementing $\calF$ as $\AESOneTwoEight_{\seedA}(\idx_1) \| \cdots \| \AESOneTwoEight_{\seedA}(\idx_M)$. We show that
$\calC^\calG$ is indifferentiable from $\calF$ as follows. Consider the two
worlds with which $\calD$ interacts to make queries on the construction
$\calC$ and $\calG$:
\begin{itemize}
	\item \textbf{REAL.} In the real world, upon query $\calC(k)$, $\calD$
	receives $\AESOneTwoEight_k(\idx_1) \| \cdots \| \AESOneTwoEight_k(\idx_M)$. 
      Queries to $\calG$ are answered naturally with
	$\AESOneTwoEight_{(\cdot)}(\cdot)$ or $\AESOneTwoEight^{-1}_{(\cdot)}(\cdot)$ as
	required.
	\item \textbf{IDEAL.} In the ideal world, upon query $\calC(k)$, the
	simulator $\calS$ simulates $\calF$ as follows. $\calS$ samples $M$
	uniformly random strings $s_1, \ldots, s_M$ subject to no collisions and outputs
	$\calF(k) = s_1 \| \cdots \| s_M$. It additionally stores a mapping $M_k$
	from $\{\idx_1, \ldots, \idx_M\}$ to $S = \{s_1, \ldots, s_M\}$. These
	will be used to answer $\calG$ queries. Without loss of generality, we
	assume that whenever $\calG$ is queried on a key $k$, $\calS$ pretends
	that $\calC(k)$ has been queried and sets up $M_k$.
	
	$\calD$ can now effectively simulate an ideal cipher $\calG$ as follows.
	For forward queries with an input in $\Idx$ or backward queries with
	an input in $S_k$, $\calS$ uses the mapping $M_k$ to answer the query in a
	manner consistent with $\calC(\cdot)$ simulation. For all other queries, the simulator maintains an
	on-the-fly table to simulate an ideal cipher. It samples independent
	uniformly random responses for each input query (forward or backward) subject to
	the fact that the resulting table of input/output pairs $(x, y)$ combined
	with $(\idx_i, s_i)$ pairs remains a permutation
	over $\{0, 1\}^{128}$ for every key $k$.
\end{itemize}

It is easy to see that the simulator is efficient. Indifferentiability of the
two worlds follows by construction as $\AESOneTwoEight(\cdot, \cdot)$ is
modeled as an ideal cipher. Thus, in generating $\bfA$ starting with a seed
$\seedA$ using $\AESOneTwoEight$, we can effectively replace the ideal
domain extension primitive $\calF$ with our construction in the ideal cipher
model.

\paragraph{Using $\SHAKE128$ to generate $\bfA$.} An argument in using
$\SHAKE128$ to expand $\seedA$ to the matrix~$\bfA$ is significantly
simpler. In the random oracle model, $\SHAKE128$ is an ideal XOF
\cite{dworkin2015sha}. In fact, for every distinct prefix $str$, we can
model $\SHAKE128(str \| \cdot, \ell)$ as an independent hash function mapping
$\{0, 1\}^{128}$ to $\{0, 1\}^\ell$. 

The domain expansion step is constructed
by computing $\SHAKE128(\inner{i} \| \seedA, 16n)$ for $1 \leq i \leq n$
where $\inner{i} \in \bit^{16}$; each step fills up the $i$th row of the matrix $\bfA$. As each row is
independently constructed via an ideal hash function, this construction maps
a uniformly random seed $\seedA$ to a much larger uniformly random matrix $\bfA$
thereby implementing the ideal functionality $\calF$ perfectly.

\paragraph{Reusing $\bfA$.}

Finally, we point out that generating~$\bfA$ from
$\seedA$ can be a significant computational burden, but this
cost can be amortized by relaxing the requirement that a fresh
$\seedA$ be used for every instance of key encapsulation, e.g., by
caching and reusing~$\bfA$ for a small period of time. In this case,
we observe that the cost of generating~$\bfA$ represents roughly 40\% of the cost
of encapsulation and decapsulation on the targeted x64 Intel machine used in 
\autoref{sec:performance}. 
A straightforward argument shows that the amortization above is compatible with all the security
reductions in this section.  But importantly, it now allows for an
all-for-the-price-of-one attack against those key encapsulations that share
the same~$\bfA$. This can be mitigated by making sure that we cache and
reuse~$\bfA$ only for a small number of uses, but we need to do this
in a very careful manner.

\paragraph{Generating $\bfA$ from joint randomness.}
It is also possible to generate $\bfA$ from joint randomness or using protocol
random nonces.  For example, when integrating \FrodoKEM into the TLS protocol,
$\bfA$ could be generated from a seed consisting of the random nonces \texttt{client\_random}
and \texttt{server\_random} sent by the client and server in their
\texttt{ClientHello} and \texttt{ServerHello} messages in the TLS handshake protocol.  
This functionality does not match the standard
description of a KEM and the API provided by NIST, but is possible in general.
A design with both parties contributing entropy to the seed might better
protect against all-for-the-price-of-one attacks by being more robust to
faulty
random number generation at one of the parties.

\subsubsection{\INDCPA security}\label{sec:strength:cpa-pke}

In this section we show that $\FrodoPKE$, using any error distribution
$\chi$ and uniformly random $\bfA$, is an \INDCPA-secure public-key
encryption scheme based on the hardness of the learning with errors
decision problem with the same error distribution. We first tightly
relate the \INDCPA security of $\FrodoPKE$ to the normal-form DLWE
problem, where the secret coordinates have the same distribution as
the errors.

\begin{theorem}[normal-form DLWE $\implies$ $\INDCPA$ security of
  $\FrodoPKE$]
  \label{thm:cpa-pke-to-nf-dlwe}
  Let $n, q, \mbar, \nbar$ be positive integers, and $\chi$ be a
  probability distribution on $\bbZ$. There exist classical algorithms
  $\Bdversary_{1}, \Bdversary_{2}$ that use as a ``black box''
  subroutine any (quantum or classical) algorithm $\Adversary$ against
  the $\INDCPA$ security of $\FrodoPKE$ (with a uniformly random
  $\bfA$), for which
  \[ \Adv{\indcpa}{\FrodoKEM}(\Adversary) \leq \nbar \cdot
    \Adv{\nfdlwe}{n,n,q,\chi}(\Bdversary_1) + \mbar \cdot
    \Adv{\nfdlwe}{n,n+\nbar,q,\chi}(\Bdversary_2) \enspace . \] The
  running times of $\Bdversary_1$ and $\Bdversary_2$ are approximately
  that of $\Adversary$.
\end{theorem}
The proof of \autoref{thm:cpa-pke-to-nf-dlwe} is the same as that of
\cite[Theorem~3.2]{RSA:LinPei11} or \cite[Theorem~5.1]{CCS:BCDMNN16}.

The following theorem relates the LWE decision problem in its normal
form to one where the secret is \emph{uniformly random} over
$\bbZ_{q}$. We need this only for connecting the latter variant, which
arises in the reduction from worst-case lattice problems described in
the next subsection, to the normal form as used in $\FrodoPKE$. (In
particular, our cryptanalysis and concrete security bounds are for the
normal form.) The theorem is specialized to power-of-two modulus~$q$
(our case of interest), and the stated bounds in the advantage and
number of LWE samples are more precise than those given in the
original work. These bounds follow from the fact that, by a
straightforward calculation, a uniformly random $n$-by-$(n+k)$ matrix
over~$\bbZ_{q}$ has an invertible $n$-by-$n$ submatrix except with
probability at most~$2^{-k}$.

\begin{theorem}[uniform-secret DLWE $\implies$ normal-form DLWE; \cite{C:ACPS09}, Lemma~2]
  \label{thm:nf-dlwe-to-dlwe}
  Let $n, m, k, q$ be positive integers with $q \geq 2$ a power of
  two, and let $\chi$ be a probability distribution on $\bbZ$. There
  exists a classical algorithm~$\Bdversary$ that uses as a ``black
  box'' subroutine any (quantum or classical) algorithm $\Adversary$
  against the normal-form LWE decision problem, for which
  \[ \Adv{\nfdlwe}{n,m,q,\chi}(\Adversary) \leq \Adv{\dlwe}{n,m + n +
      k,q,\chi}(\Bdversary) + 2^{-k} \enspace . \] The running time of
  $\Bdversary$ is approximately that of $\Adversary$.
\end{theorem}

\subsubsection{Reductions from worst-case lattice problems}
\label{sec:strength:lattice}

When choosing parameters for LWE, one needs to choose an error
distribution, and in particular its ``width.''  Certain choices (e.g.,
sufficiently wide Gaussians) are supported by \emph{reductions} from
worst-case lattice problems to LWE; see,
e.g.,~\cite{Reg09,STOC:Peikert09,STOC:BLPRS13,STOC:PeiRegSte17}.  At a
high level, such a reduction transforms any algorithm that solves LWE
\emph{on the average}---i.e., for random instances sampled according
to the prescribed distribution---into an algorithm of related
efficiency that solves \emph{any instance} of certain lattice problems
(not just random instances).

% In this section we recall and slightly improve the most recent
% reduction, from~\cite{STOC:PeiRegSte17}.

%  For the (quantum) reduction from the worst-case approximate
%   $\SIVP$ and $\GapSVP$ problems, we improve the constant factor in
%   the lower bound~$c \sqrt{n}$ on the width of the LWE error, from~$2$
%   to~\cpnote{new constant.}.

The original work of~\cite{Reg09} and a follow-up
work~\cite{STOC:PeiRegSte17} gave quantum polynomial-time reductions,
from the worst-case $\GapSVP_\gamma$ (\autoref{def:GapSVP}),
$\SIVP_\gamma$ (\autoref{def:SIVP}), and $\DGS_{\varphi}$
(\autoref{def:DGS}) problems on $n$-dimensional lattices, to
$n$-dimensional LWE (for an unbounded polynomial $m=\text{poly}(n)$
number of samples) with Gaussian error of standard deviation
$\sigma \geq c\sqrt{n}$.  The constant factor~$c$ was originally
stated as $c=\sqrt{2/\pi}$, but can easily be improved to any
$c > 1/(2\pi)$ via a tighter analysis of essentially the same
proof.\footnote{The approximation factor~$\gamma$ for $\GapSVP$ and
  $\SIVP$ is $\tilde{O}(qn/\sigma) = (qn/\sigma) \log^{O(1)} n$, and
  the parameter~$\varphi$ for $\DGS$ is $\Theta(q\sqrt{n}/\sigma)$
  times the ``smoothing parameter'' of the lattice.} However, for
efficiency reasons our choices of~$\sigma$ (see
\autoref{tab:distribution}) are somewhat smaller than required by these
reductions.

Instead, following~\cite[Section~1.1]{Reg09}, below we obtain an
alternative \emph{classical} (i.e., non-quantum) reduction from a
variant of the worst-case bounded-distance decoding~($\BDD$) problem
to our LWE parameterizations. In contrast to the quantum reductions
described above, which requires Gaussian error of standard deviation
$\sigma \geq c \sqrt{n}$, the alternative reduction supports a
smaller error width---as small as the ``smoothing
parameter''~\cite{DBLP:journals/siamcomp/MicciancioR07} of the lattice
of integers~$\ZZ$.  For the $\BDD$ variant we consider, which we call
``BDD with Discrete Gaussian Samples'' ($\BDDwDGS$), the input
additionally includes discrete Gaussian samples over the dual lattice,
but having a larger width than known algorithms are able to
exploit~\cite{DBLP:conf/approx/LiuLM06,DBLP:conf/coco/DadushRS14}. Details
follow.

% \cpnote{New theorem stating quantum reduction with best known
%   constant?}

\paragraph{Bounded-distance decoding with discrete Gaussian samples.}
%\label{sec:strength:lattice:bdd}

% \cpnote{Should this be a paragraph header, rather than a new
%   subsubsection (which separates it from the previous one, on
%   worst-case reductions?)}

We first define a variant of the bounded-distance decoding problem,
which is implicit in prior works that consider ``$\BDD$ with
preprocessing,''~\cite{DBLP:journals/jacm/AharonovR05,DBLP:conf/approx/LiuLM06,DBLP:conf/coco/DadushRS14}
and recall the relevant aspects of known algorithms for the problem.

\begin{definition}[Bounded-distance decoding with discrete Gaussian samples]
  \label{def:BDDwDGS}
  For a lattice~$\calL \subset \RR^{n}$ and positive reals
  $d < \lambda_{1}(\calL)/2$ and $r > 0$, an instance of the
  \emph{bounded-distance decoding with discrete Gaussian samples}
  problem $\BDDwDGS_{\calL,d,r}$ is a point $\bft \in \RR^{n}$ such that
  $\dist(\bft,\calL) \leq d$, and access to an oracle that samples from
  $D_{\calL^{*},s}$ for any (adaptively) queried $s \geq r$. The goal is
  to output the (unique) lattice point $\bfv \in \calL$ closest to~$\bft$.
\end{definition}

\begin{remark}
  \label{rem:BDDwDGS-param}
  For a given distance bound~$d$, known $\BDDwDGS$ algorithms use
  discrete Gaussian samples that all have the same width
  parameter~$s$. However, the reduction to LWE will use the ability to
  vary~$s$. Alternatively, we mention that when
  $r \geq \eta_{\eps}(\calL^{*})$ for some very
  small~$\eps > 0$ (which will always be the case in our setting),
  we can replace the variable-width DGS oracle from
  \autoref{def:BDDwDGS} with a fixed-width one that samples from
  $D_{\bfw+\calL^{*},r}$ for any queried coset $\bfw+\calL^{*}$,
  always for the same width~$r$.  This is because we can use the
  latter oracle to implement the former one (up to statistical
  distance $8\eps$), by sampling~$\bfe$ from the continuous
  Gaussian of parameter $\sqrt{s^{2}-r^{2}}$ and then adding a sample
  from $D_{\calL^{*}-\bfe,r}$. See~\cite[Theorem~3.1]{C:Peikert10} for
  further details.
\end{remark}

The state-of-the-art algorithms for solving
$\BDDwDGS$~\cite{DBLP:journals/jacm/AharonovR05,DBLP:conf/approx/LiuLM06,DBLP:conf/coco/DadushRS14}
employ a certain $\calL$-periodic function
$f_{\calL,1/r} \colon \RR^{n} \to [0,1]$, defined as
\begin{equation}
  \label{eq:f_L}
  f_{\calL,1/r}(\bfx) := \frac{\rho_{1/r}(\bfx+\calL)}{\rho_{1/r}(\calL)} =
  \E_{\bfw \sim D_{\calL^{*},r}}[\cos(2\pi \inner{\bfw, \bfx})] \enspace ,
\end{equation}
where the equality on the right follows from the Fourier series of
$f_{\calL,1/r}$ (see~\cite{DBLP:journals/jacm/AharonovR05}).  To solve
$\BDDwDGS$ for a target point~$\bft$, the algorithms use several
discrete Gaussian samples $\bfw_{i} \sim D_{\calL^{*},r}$ to estimate
the value of~$f_{\calL,1/r}$ at~$\bft$ and nearby points via
\autoref{eq:f_L}, to ``hill climb'' from~$\bft$ to the nearest
lattice point.  For the relevant points~$\bft$ we have the (very
sharp) approximation
\[ f_{\calL,1/r}(\bft) \approx \exp(-\pi r^{2} \cdot
  \dist(\bft,\calL)^{2}) \enspace , \] so by the Chernoff-Hoeffding
bound, approximating $f_{\calL,1/r}(\bft)$ to within (say) a factor of
two uses at least
\[ \frac{1}{f_{\calL,1/r}(\bft)^{2}} \approx \exp(2\pi r^{2} \cdot
  \dist(\bft,\calL)^{2}) \] samples.\footnote{In fact, the algorithms
  need approximation factors much better than two, so the required
  number of samples is even larger by a sizable constant
  factor. However, the above crude bound will be sufficient for our
  purposes.}  Note that without enough samples, the ``signal'' of
$f_{\calL,1/r}(\bft)$ is overwhelmed by measurement ``noise,'' which
prevents the hill-climbing from making progress toward the answer.

In summary, when limited to~$N$ discrete Gaussian samples, the known
approaches to solving $\BDDwDGS$ are limited to distance
\begin{equation}
  \label{eq:dist-given-samples}
  \dist(\bft, \calL) \leq r^{-1} \sqrt{\ln(N)/(2\pi)} \enspace .
\end{equation}
Having such samples does not appear to provide any speedup in
decoding at distances that are larger than this bound by some constant factor
greater than one.  In particular, if
$d \cdot r \geq \omega(\sqrt{\log n})$ (which is the smoothing
parameter of the integer lattice~$\bbZ$ for negligible error~$\eps$),
then having $N=\text{poly}(n)$ samples does not seem to provide any
help in solving $\BDDwDGS_{\calL,d,r}$ (versus having no samples at
all).

% CJP: don't need this because maximizing the decoding *distance* is
% not relevant to the reduction to LWE (the BDD distance we use in the
% reducing is much smaller than what can be decoded in principle).
% Instead, the *sample complexity* of the BDD algorithms (vs the
% reduction) is our main concern.

% The state of the art for solving $\BDDwDGS$ is~\cite{DRS}, which
% obtains a very sharp (non-asymptotic) bound on the decoding distance,
% along with a sufficiently tight lower bound on the number of discrete
% Gaussian samples used.

% \begin{theorem}[{{\cite[Theorem~3.1, generalized]{DRS}}}]
%   \label{thm:DRS-BDDwDGS}
%   Fix a lattice $\calL \subset \RR^{n}$ and some positive
%   $\eps < 1/200$, and let
%   $d_{\eps} = \sqrt{\ln(2(1+1/\eps))/\pi}$ and
%   $\delta_{\max} = \tfrac12 - \tfrac{2}{\pi d_{\eps}^{2}}$.  Then
%   for any $r \geq \eta_{\eps}(\calL^{*})$, there is an algorithm that
%   solves $\BDDwDGS_{\calL,d,r}$ for distance bound
%   $d = \delta_{\max} d_{\eps} / r$, using
%   $N > \ln (1/\eps)/\sqrt{\eps}$ samples from $D_{\calL^{*},r}$.
% \end{theorem}

% \begin{remark}
%   \label{rem:DRS-BDDwDGS}
%   The above statement generalizes the equality
%   $r = \eta_{\eps}(\calL^{*})$ in~\cite{DRS} to a lower bound, which
%   simply replaces $\eta_{\eps}(\calL^{*})$ with~$r$ in the denominator
%   of the distance bound~$d$. \cpnote{More about why this change is
%     justified.  The decoding distance comes from the analysis of
%     gradient ascent on $f_{\calL}$ itself, ignoring its approximation via
%     samples.  The number of samples comes from the approximation of
%     $f_{\calL}$ and its gradient: DGS shows that
%     $f_{\calL}(\bft) \approx \eps^{1/4}$ for $\bft$ within the
%     distance bound, hence $1/\sqrt{\eps}$ samples are needed for
%     the approximation, by Chernoff.}
% \end{remark}

% For example, if we let $\eps = 2^{-256}$ so that the algorithm
% needs $N > 2^{128}$ discrete Gaussian samples and hence steps of
% computation (even ignoring the second-order terms), then the decoding
% radius $d < 3.681/r$.  This suggests the following:

% \begin{conjecture}
%   \label{con:BDDwDGS-hard}
%   $\BDDwDGS_{\calL,d,r}$ is concretely infeasible in the worst case for
%   any $d \geq 4/r$ and any not-too-large
%   $r \geq \eta_{2^{-256}}(\calL^{*})$.\footnote{The ``not too large''
%     qualifier is needed here because $\BDD$ is easy for the very small
%     distance bound $2^{-n} \cdot \lambda_{1}(\calL)$, using LLL.}
% \end{conjecture}



\paragraph{Reduction from \BDDwDGS to LWE.}

We now recall the following result from~\cite{STOC:PeiRegSte17},
which generalizes a key theorem from~\cite{Reg09} to give a reduction
from $\BDDwDGS$ to the LWE decision problem.

\begin{theorem}[{{$\BDDwDGS$ hard $\implies$ decision-LWE hard
      \cite[Lemma~5.4]{STOC:PeiRegSte17}}}]
  \label{thm:bddwdgs-to-dlwe}
  Let $\eps = \eps(n)$ be a negligible function and let
  $m=\text{poly}(n)$ and $C=C(n) > 1$ be arbitrary.  There is a
  probabilistic polynomial-time (classical) algorithm that, given
  access to an oracle that solves $\DLWE_{n,m,q,\alpha}$ with
  non-negligible advantage and input a number $\alpha \in (0,1)$, an
  integer $q \geq 2$, a lattice $\calL \subset \RR^{n}$, and a
  parameter $r \geq C q \cdot \eta_{\eps}(\calL^{*})$, solves
  $\BDDwDGS_{\calL,d,r}$ using $N=m \cdot \text{poly}(n)$ samples,
  where $d = \sqrt{1-1/C^{2}} \cdot \alpha q/r$.
\end{theorem}

\begin{remark}
  \label{rem:PRS-generalized}
  The above statement generalizes the fixed choice of $C=\sqrt{2}$ in
  the original statement (inherited from~\cite[Section~3.2.1]{Reg09}),
  using~\cite[Corollary~3.10]{Reg09}. In particular, for any constant
  $\delta > 0$ there is a constant $C > 1$ such that
  $d = (1-\delta) \cdot \alpha q / r$.
\end{remark}
% justification: the inner product of the discrete Gaussian with the
% BDD error has param \sqrt{1-1/C^2} alpha q, so we need to add
% (alpha q/C) "smoothing" error. This corresponds to adding a
% continuous Gaussian with param
% alpha q/(C d) = r / (C sqrt(1-1/C^2)) = r / sqrt(C^2-1)
% to the discrete Gaussian of param r \geq C q eta.
% This works out with the harmonic mean requirement on the
% discrete+continuous lemma (Claim 3.9 of Regev).


In particular, by \autoref{eq:dist-given-samples}, if the
Gaussian parameter $\alpha q$ of the LWE error sufficiently exceeds
$\sqrt{\ln(N)/(2\pi)}$ (e.g., by a constant factor greater than one),
then the $\BDDwDGS_{\calL,d,r}$ problem is plausibly hard (in the
worst case), hence so is the corresponding LWE problem from
\autoref{thm:bddwdgs-to-dlwe} (on the average).  An interesting
direction is to obtain a more precise bound on, and improve, the
``sample overhead'' of the reduction, i.e., the $\text{poly}(n)$
factor connecting the number of LWE samples~$m$ and the number of DGS
samples~$N$.

%%% Local Variables:
%%% mode: latex
%%% TeX-master: "Main"
%%% End:


\section{Renewed Cryptanalysis}

In this section, we provide a renewed cryptanalysis of the LWE instances underlying the \INDCPA-secure $\FrodoPKE$ public-key encryption scheme. 
Originally proposed as part of the $\FrodoKEM$ submission to the NIST PQC standardization process~\cite{NISTPQC-R3:FrodoKEM20}, these LWE instances received a thorough cryptanalysis in the style of ML-KEM~\cite{MLKEM,NISTPQC-R3:CRYSTALS-Kyber20}.
Such analysis proposes two computational models, ``core-SVP'' and ``beyond core-SVP'' to estimate the cost of lattice reduction attacks on LWE. The specific estimates are then generated using the ``leaky-LWE-estimator'' script introduced in \cite{dachman2020lwe}.\footnote{\url{https://github.com/lducas/leaky-LWE-Estimator/tree/NIST-round3}}

In order to revalidate our security claims using different code and to extend them to account for results published since \cite{NISTPQC-R3:FrodoKEM20}, we proceed to perform an analysis of the cost of lattice attacks, using the ``lattice-estimator'' script.\footnote{\url{https://github.com/malb/lattice-estimator}}
We will make our estimates reproducible by publishing our code. Overall, we keep the internals of the estimator untouched, only customizing the cost models in order to capture different hypothetical scenarios.

\subsection{Relevant Lattice Reduction Attacks}

We briefly recall the cryptanalytic attacks on LWE relevant to the $\Frodo$ parametrization. These attacks have been analyzed in extensive prior work. Hence, we will just give a brief description and references here.

\paragraph{Primal attack.}
Let $(\bfA, \bfb) \in \ZZ_q^{m \times n} \times \ZZ_q^m$ be a collection of $m$ LWE samples. The primal attack attempts to solve the search variant of LWE by finding the vector closest to $\bfb$ in the ``primal lattice''
\[
\Lambda_q(\bfA) = \left\{\bfA \bfx \bmod q \mid \bfx \in \ZZ^n \right\}.
\]
Since $\bfb = \bfA \bfs + \bfe \bmod q$ with $\bfe$ coming from a narrow distribution such that \mbox{$\norm{\bfe} \ll \lambda_1(\Lambda_q(\bfA))$}, this leads to recovery of $\bfA \bfs$ as the closest vector to $\bfb$. From this, $\bfe$ can be recovered, and consequently $\bfs$ if $\bfA$ has rank $n$.

We estimate the cost of two common variants of the primal attack, that we call ``uSVP'' and ``BDD''.
In the uSVP case, we build a basis $\bfB = \left( \bfA \mid q\bfI_m \mid -\bfb \right)^T$ and reduce it using block reduction, as to obtain a basis close to BKZ-$\beta$-reduced for a block size $\beta$ satisfying the \cite{USENIX:ADPS16} success condition.\footnote{We note that while smaller block size $\beta-\Delta$ may suffice, the success probability drops sharply as $\Delta$ grows~\cite{PKC:PosVir21}.}\todo{We could say more if we really wanted, but I'm not sure it's necessary}
This results in recovering the shortest vector in the integer span of $\bfB$, that is $(\bfs^T \mid \mathbf{\star}^T \mid 1) \cdot \bfB = - \bfe^T$, where $\mathbf{\star} = ( \bfb - \bfA \bfs - \bfe )/q$.
In the BDD case, we first build a basis $\bfB_0 = \left( \bfA \mid q\bfI_m \right)^T$, and BKZ-$\beta_\text{red}$ reduce it. Then construct a new basis $\bfB_1 = \left(\bfB_0 \mid -\bfb^T \right)$, LLL reduce it, and perform one call to a shortest vector problem (SVP) solver in rank $\beta_\text{svp}$ on the last projective block of the basis, followed by a final call to LLL, similarly to \cite{RSA:LiuNgu13}. This similarly recovers $-\bfe$, but allows the flexibility of using \mbox{$\beta_\text{red} < \beta_\text{svp}$}, balancing the cost of the reduction versus recovery-of-$\bfe$ phases.

\paragraph{Dual attack.}
Let $(\bfA, \bfb) \in \ZZ_q^{m \times n} \times \ZZ_q^m$. The dual attack attempts to solve the decision variant of LWE by distinguishing whether $\bfb = \bfA \bfs + \bfe \bmod q$, or $\bfb$ was sampled uniformly at random from $\ZZ_q^m$. In general terms, this is done by finding a short vector $\bfx$ in the ``dual lattice''
\[
\Lambda_q^\perp(\bfA) = \left\{\bfx \in \ZZ^m \mid \bfx^T \bfA = 0 \bmod q \right\}.
\]
Once such a vector is found, one computes $\bfx^T \bfb$. If $\bfb$ was sampled from the LWE distribution, then \mbox{$\bfx^T \bfb = \bfx^T \bfe \ll q$}, while if $\bfb$ was uniformly sampled, then $\bfx^T \bfb$ will be similarly uniformly distributed $\bmod~q$~\cite{PQCBook:MicReg09}.\footnote{A variant where short vectors $(\bfx, \bfy)$ such that $\bfx^T \bfA = \bfy \bmod q$ are sought can be similarly used if $\bfs$ is sampled from a narrow distribution, such that \mbox{$\bfx^T \bfb = \bfy^T \bfs + \bfx^T \bfe \ll q$}.}
The shorter $\bfx$, the more accurate this test is, but also the costlier it is to recover $\bfx$. Therefore, there is a natural accuracy-runtime trade-off.
Let $b = 0$ if $\bfb = \bfA \bfs + \bfe \mod q$, and $b = 1$ otherwise. Let $T_i \in \{0, 1\}$ be the outcome of the $i$-th run of the test above,\footnote{Probability over the coins used by lattice reduction and used for re-randomizing the basis of $\Lambda_q^\perp(\bfA)$ before lattice reduction.} and let $\varepsilon$ be the distinguishing advantage of such test, meaning that $\Pr[T_i = b] \ge \frac{1}{2} + \varepsilon$. To amplify this advantage, we run the test $N$ times, and perform majority voting, meaning the overall attack returns a guess $\lfloor \sum T_i / N \rceil$ for $b$. Using Hoeffding's inequality, or other similar tail bounds over $\sum T_i$, one can see that choosing $N$ such that $N \varepsilon^2 \in O(1)$ suffices to correctly guess $b$ with high probability. Using the Chernoff bound over $\sum T_i$, one can see that choosing $N = 1/(2 \varepsilon^2)$ upper bounds the probability of guessing incorrectly by $(2/e) \cdot 2^{-N}$ as $N \rightarrow \infty$.

Both the primal and dual attack can be optimized to exploit specifics of the LWE parametrization being attacked. Narrow secret distributions can be targeted by mounting attacks where variants of the above lattices are reduced, achieving overall smaller attack costs~\cite{ACISP:BaiGal14,EC:Albrecht17,_INDOCRYPT:EJK20,AC:GuoJoh21,MATZOV22}.

The specific attacks can also be reduced in cost if the block-reduction algorithm being used presents some specific feature. For example, sieving-based SVP solvers can output more than one short vector at the same cost as a single one~\cite{USENIX:ADPS16}. These can be used to perform fewer calls to the SVP solver as part of lattice reduction~\cite{EC:ADHKPS19} in both attacks, and to reduce the number of times that a basis for $\Lambda_q^\perp(\bfA)$ has to be reduced to perform $N$ overall tests during the dual attack~\cite{RSA:LaaWal21}.

Given LWE distribution parameters and an attack strategy, identifying optimal attack parameters and their implied cost requires a model for the cost and quality of lattice reduction and SVP solving (and, in the case lattice sieving is used, a model on the number of short vectors returned by the SVP solver). We proceed to describe two possible methodologies for carrying out such estimates.

\subsection{Estimating the Cost of Lattice Reduction}

The primal and dual atacks require strong lattice reduction, capable of producing lattice bases of significantly better quality than the one given in input.
State-of-the-art solutions use ``block-reduction'', where only adjacent subsets of basis vectors, or blocks, are considered during each step. Oracles for solving the SVP or approximate-SVP~\cite{_JoC:LiNgu24} problems are called on orthogonal projections of these blocks, and the short vectors output are then integrated into the original basis.

The general approach just described is behind most strong lattice reduction algorithms, such as BKZ~\cite{FCT:SE91,schnorr1994lattice}, Progressive-BKZ~\cite{EC:AWHT16}, Self-dual BKZ~\cite{EC:MicWal16}, and G6K~\cite{EC:ADHKPS19}. The leading term in the cost (in terms of runtime, memory or energy) in all of these algorithms is the cost of the (approximate) SVP solver. Assessing the cost of lattice reduction hence reduces to determining the cost of SVP solving, and the number of times that the SVP solver must be called before good reduction is achieved. The notion of reduction being targeted is BKZ-$\beta$-reducedeness~\cite{FCT:SE91,schnorr1994lattice}, requiring that every first vector of a reduced lattice basis' block of rank $\beta$ be the shortest vector in the block. 

\subsection{Core-SVP Methodology}

In the ``core-SVP'' methodology~\cite{USENIX:ADPS16} one observes that in order to achieve BKZ-$\beta$-reducedness, the shortest vector problem must have been solved on blocks of rank $\beta$. Being the cost of SVP solving the leading term in the cost of lattice reduction, one can approximate the cost of lattice reduction by just considering the smaller cost of a single SVP solver call. This choice is a conservative one, but also adds a margin against possible improvements in block-reduction reducing the number of SVP calls required to achieve BKZ-$\beta$-reducedness.

The utility of core-SVP estimates is in the security margin they provide. For this reason, we make various simplifying assumptions to obtain a likely lower bound for the cost of lattice reduction.
\begin{itemize}
	\item We exclusively consider algorithms in the RAM model, where read/write access to memory is essentially free.
	\item We consider asymptotic quantum speedups to classical algorithms, where quantum computation can be run arbitrarily long times, and doesn't incur any quantum memory costs.
	\item We consider that operations on rank-$\beta$ vectors cost $\beta$ arithmetic operations. While cost models targeting ring-based lattices somewhat ignore this cost factor to hedge against any use of rotation symmetry in the vectors derived from the ring elements, we consider this to be an inherent overhead when working with unstructured lattices.
\end{itemize}

These concessions to the adversary imply the use of the asymptotically fastest available SVP solvers, lattice sieves. In particular, we consider \cite{SODA:BDGL16} sieves using locality-sensitive filtering, and proposed quantum speedups using Grover's algorithm~\cite{AC:AGPS20} and quantum random walks~\cite{cryptoeprint:2024/1692}.

Due to our focus on lattice sieving, we make two further considerations when estimating the cost of the primal and dual attacks.\todo{I don't think d4f and more-than-one sv are immediately compatible. \cite{AC:GuoJoh21} argues how to enable both by using two-step reduction.}
\begin{itemize}
	\item We assume that ``dimensions-for-free'' speedups~\cite{EC:Ducas18} hold at all times, meaning that in order to solve SVP in rank $\beta$, we only need to perform lattice sieving in rank $\beta \cdot \dff_\beta$ where $\dff_\beta = \left(1 - \frac{\log{(4/3)}}{\log{(\beta / (2 \pi e))}}\right)$.\todo{recheck this was the right factor}
	\item We assume that lattice sieves in rank $\beta$ return $2^{0.2075 \beta}$ short vectors in the lattice,\todo{how short?} which is very useful at the moment of performing dual attacks.\todo{re-check wording given d4f. is the exponent right? and especially, `in rank $\beta$' is ambiguous}
\end{itemize}

With these simplifications, we obtain three cost models for lattice reduction outputting BKZ-$\beta$-reduced bases, corresponding directly to three models for the cost of solving SVP in rank $\beta$:
\begin{itemize}
	\item \clsfsieve, where SVP in rank $\beta$ costs $2^{0.292 \cdot \beta \cdot \dff_\beta + \log_2(\beta \cdot \dff_\beta)}$.
	\item \qgroversieve, where SVP in rank $\beta$ costs $2^{0.265 \cdot \beta \cdot \dff_\beta + \log_2(\beta \cdot \dff_\beta)}$.
	\item \qrandwalksieve, where SVP in rank $\beta$ costs $2^{0.257 \cdot \beta \cdot \dff_\beta + \log_2(\beta \cdot \dff_\beta)}$.
\end{itemize}

With these cost models available, we proceed to estimate the cost of the ``uSVP'' and ``BDD'' primal attacks, as well as the dual attack described above and the controversial~\cite{C:DucPul23,EC:PouShe24,cryptoeprint:2023/1850} ``dual-sieve-FFT''~\cite{AC:GuoJoh21,MATZOV22} variant. In Table~\ref{tab:new-core} we report the resulting cost estimates.

\begin{table}
	\centering
	\resizebox{\textwidth}{!}{
\begin{tabular}{lllrrrrrrrrrrrrrrrrrrr}
	\hline
	Params     & SVP model         & Reduction   &   uSVP cost &   uSVP beta &   uSVP dim &   BDD cost &   BDD red beta &   BDD svp beta &   BDD dim &   Dual cost &   Dual vecs (?) &   Dual beta &   Dual dim &   Dual hyb cost &   Dual guess (?) &   Dual N (?) &   Dual d/s indices (?) &   Dual fft indices (?) &   Dual red beta &   Dual sieve beta &   Dual LWE samples \\
	\hline
	Frodo-640  & q-rand-walk-sieve & core        &     122.966 &         486 &       1285 &    123.987 &            485 &            487 &      1288 &     127.857 &         113.979 &         506 &       1288 &         126.88  &          116.913 &      103.575 &                      0 &                      0 &             502 &               502 &                640 \\
	Frodo-640  & q-grover-sieve    & core        &     126.52  &         486 &       1285 &    127.542 &            485 &            487 &      1288 &     131.561 &         113.979 &         506 &       1288 &         130.553 &          116.913 &      103.575 &                      0 &                      0 &             502 &               502 &                640 \\
	Frodo-640  & c-lsf-sieve       & core        &     138.514 &         486 &       1285 &    139.541 &            485 &            487 &      1288 &     144.064 &         113.979 &         506 &       1288 &         142.954 &          116.913 &      103.575 &                      0 &                      0 &             502 &               502 &                640 \\
	\hline
	Frodo-967  & q-rand-walk-sieve & core        &     175.052 &         699 &       1934 &    176.052 &            699 &            699 &      1938 &     180.433 &         159.103 &         721 &       1942 &         179.699 &          161.083 &      148.29  &                      0 &                      0 &             718 &               718 &                967 \\
	Frodo-967  & q-grover-sieve    & core        &     180.211 &         699 &       1934 &    181.211 &            699 &            699 &      1938 &     185.758 &         159.103 &         721 &       1942 &         185.001 &          161.083 &      148.29  &                      0 &                      0 &             718 &               718 &                967 \\
	Frodo-967  & c-lsf-sieve       & core        &     197.621 &         699 &       1934 &    198.621 &            699 &            699 &      1938 &     203.728 &         159.103 &         721 &       1942 &         201.249 &          195.015 &      146.948 &                     10 &                      0 &             712 &               712 &                967 \\
	\hline
	Frodo-1344 & q-rand-walk-sieve & core        &     231.562 &         930 &       2577 &    232.583 &            929 &            931 &      2588 &     237.68  &         208.063 &         955 &       2656 &         225.934 &          206.56  &      188.035 &                      0 &                     80 &             907 &               907 &               1344 \\
	Frodo-1344 & q-grover-sieve    & core        &     238.467 &         930 &       2577 &    239.489 &            929 &            931 &      2588 &     244.774 &         208.063 &         955 &       2656 &         231.521 &          227.855 &      186.476 &                      0 &                     80 &             902 &               902 &               1344 \\
	Frodo-1344 & c-lsf-sieve       & core        &     261.769 &         930 &       2577 &    262.796 &            929 &            931 &      2588 &     268.715 &         208.063 &         955 &       2656 &         253.157 &          234.665 &      186.524 &                     10 &                     80 &             899 &               899 &               1344 \\
	\hline
\end{tabular}
}
\caption{\textcolor{red}{TODO: resolve discrepancy wrt number of returned vecs per sieve, style the table to make it readable.}\label{tab:new-core}}
\end{table}

\todo[inline]{Talk a little about the results, disclaiming quantum/classical etc.}
\todo[inline]{May want to add a note about attacking the public-key versus the ciphertext.}

\subsection{Beyond Core-SVP Methodology}

While the core-SVP methodology can give us confidence on the baseline hardness of lattice reduction attacks, it is quite pessimistic about the effective security of lattice-based schemes. Having a good security buffer is important, since it is not possible to exclude future improvements in cryptanalysis. Relying only on core-SVP estimates may lead to an unnecessary over-specification of parameters to achieve such a buffer.

In the third round of the NIST PQC standardization process, the designers of ML-KEM introduced a ``beyond core-SVP'' methodology~\cite{NISTPQC-R3:CRYSTALS-Kyber20}, that was also adopted by the $\Frodo$ team~\cite{NISTPQC-R3:FrodoKEM20}. Therein, they make various further assumptions on the practical hardness of lattice reduction, such as the difference between the costs of BKZ and progressive BKZ, the number of gates required to impement arithmetic operations, the possibility that lower-than-expected block sizes still lead to successful attacks.

In this section we propose assumptions along similar lines. These are not a strict subset or superset of the assumptions made in~\cite{NISTPQC-R3:CRYSTALS-Kyber20}. Our intention is simply to propose a complementary look at possible attack overheads. In short, we consider the following.
\begin{itemize}
	\item Ignoring the number of SVP solver calls leads to a significant underestimate of the cost of lattice reduction, when most algorithms will need to call the solver in the order of thousands of times to reduce the primal and dual lattice bases. We instead consider that BKZ requires $8 \cdot (d - \beta)$ calls to an SVP solver to reduce a $d$-dimensional lattice. Progressive BKZ instead will require $\sum_{\beta' = 60 }^{\beta} d - \beta'$ calls to the SVP solver, where we consider SVP calls in rank $< 60$ to be of negligible cost.
	\item Lattice sieving is a very memory intensive algorithm. Accounting for the extra memory cost requires more than a na\"ive area-times multiplication, since \cite{SODA:BDGL16} optimizes its filters for the RAM model. We adopt the recent analysis of the area-times cost of lattice sieving using two-dimensional memory architectures by Jaques~\cite{CiC:Jaques24}, which re-calibrates locality-sensitive filtering and recursive sieving to optimize costs. This result in the \cmemsieve cost model, where SVP in rank $\beta$ costs $2^{0.3113 \cdot \beta \cdot \dff_\beta + \log_2(\beta \cdot \dff_\beta)}$.
	\item Due to the significant cost of memory in sieving, for sanity check we also estimate the cost of SVP solving via parallel pruned enumeration. To make a fair comparison with \cmemsieve, we compute the amount of memory required by \cmemsieve, and use Assumption~1 in~\cite{CiC:Jaques24} together with their case-study of the $\textsc{Cores/Memory}$ ratio for a nVidia GeForce RTX 4090 graphics card, to estimate the number of cores available for enumeration on a computer able to alternatively run lattice sieving as the SVP solver. We then take the asymptotically cheapest approximate-SVP solver known based on pruned enumeration~\cite{C:ABLR21}, and obtain the \cparaenum cost model, where SVP solving in rank $\beta$ costs $2^{0.1250 \beta \log_2 \beta - 0.654 \beta + 25.84 + \log_2(64)} / 2^{0.2075 \beta - \log_2(\textsc{Cores/Memory})}$. \todo{for enum, should we count the number of lattice vectors using d4f or not when comparing to sieving?}
	\item We exclude from cost estimates the dual attack, as many of the proposed speedups have raised controversy~\cite{C:DucPul23,EC:PouShe24,cryptoeprint:2023/1850}, and there is a general lack of positive experimental results implementing it.\todo{make sure! I guess only \cite{EC:Albrecht17} does?}
	\item It is currently unclear how feasible long-running quantum computations in the style of Grover's search or quantum random walks will be. Limiting circuit depth is believed to incur in a sharp loss of the quantum computing advantage~\cite{EC:JNRV20}, severely limiting the resulting speedups\cite{NISTPQC-R1:NTRU-HRSS-KEM17,AC:AonNguShe18} on pruned enumeration~\cite{C:BBTV24}. Similar loss of quantum advantage has also been predicted in the case of Grover and quantum-random-walk-based lattice sieving~\cite{AC:AGPS20,cryptoeprint:2024/1692}, due to memory access and correction costs, as well as limitations on circuit depth. We therefore ignore conjectured quantum speedups for lattice sieving.
	\item \todo[inline]{if time suffices, do variance over winning $\beta$}
	\item \todo[inline]{it we figure it out, do two-step paper estimates}
\end{itemize}

We then proceed to re-estimate the primal uSVP and BDD attacks, using the \clsfsieve, \cmemsieve and \cparaenum SVP cost models, assuming BKZ and Progressive BKZ lattice reduction. The results can be found in Table~\ref{tab:new-beyond}.

\begin{table}
	\centering
	\resizebox{\textwidth}{!}{
		\begin{tabular}{lllrrrrrrr}
			\hline
			Params     & SVP model            & Reduction   &   uSVP cost &   uSVP beta &   uSVP dim &   BDD cost &   BDD red beta &   BDD svp beta &   BDD dim \\
			\hline
			Frodo-640  & c-lsf-sieve          & bkz         &     154.745 &         499 &       1288 &    149.846 &            478 &            512 &      1287 \\
			Frodo-640  & 2d-ram-sieve         & bkz         &     163.555 &         499 &       1288 &    158.58  &            479 &            511 &      1286 \\
			Frodo-640  & parallel-approx-enum & bkz         &     192.973 &         499 &       1288 &    187.26  &            484 &            505 &      1289 \\
			Frodo-640  & c-lsf-sieve          & pbkz        &     154.269 &         499 &       1288 &    150.502 &            482 &            508 &      1285 \\
			Frodo-640  & 2d-ram-sieve         & pbkz        &     162.995 &         499 &       1288 &    159.094 &            484 &            505 &      1289 \\
			Frodo-640  & parallel-approx-enum & pbkz        &     191.92  &         499 &       1288 &    187     &            485 &            504 &      1289 \\
			\hline
			Frodo-967  & c-lsf-sieve          & bkz         &     216.411 &         719 &       1927 &    210.379 &            694 &            730 &      1941 \\
			Frodo-967  & 2d-ram-sieve         & bkz         &     229.221 &         719 &       1927 &    223.056 &            695 &            729 &      1938 \\
			Frodo-967  & parallel-approx-enum & bkz         &     297.37  &         719 &       1927 &    289.397 &            702 &            720 &      1942 \\
			Frodo-967  & c-lsf-sieve          & pbkz        &     215.931 &         719 &       1927 &    210.915 &            698 &            725 &      1942 \\
			Frodo-967  & 2d-ram-sieve         & pbkz        &     228.657 &         719 &       1927 &    223.437 &            698 &            725 &      1942 \\
			Frodo-967  & parallel-approx-enum & pbkz        &     296.139 &         719 &       1927 &    288.754 &            702 &            720 &      1942 \\
			\hline
			Frodo-1344 & c-lsf-sieve          & bkz         &     282.932 &         957 &       2576 &    275.441 &            928 &            962 &      2628 \\
			Frodo-1344 & 2d-ram-sieve         & bkz         &     300.082 &         957 &       2576 &    292.272 &            928 &            962 &      2628 \\
			Frodo-1344 & parallel-approx-enum & bkz         &     424.015 &         957 &       2576 &    412.795 &            935 &            954 &      2606 \\
			Frodo-1344 & c-lsf-sieve          & pbkz        &     282.449 &         957 &       2576 &    275.951 &            930 &            960 &      2604 \\
			Frodo-1344 & 2d-ram-sieve         & pbkz        &     299.516 &         957 &       2576 &    292.753 &            931 &            959 &      2598 \\
			Frodo-1344 & parallel-approx-enum & pbkz        &     422.664 &         957 &       2576 &    412.089 &            936 &            953 &      2600 \\
			\hline
		\end{tabular}
	}
	\caption{\textcolor{red}{TODO: add two-step attack, style the table to make it readable.}\label{tab:new-beyond}}
\end{table}

\FloatBarrier

\section{Cryptanalytic attacks}%
\label{sec:attack:cryptanalytic}

\begin{table}
\begin{center}
\caption{\textbf{Primal and dual attacks on a single instance of an LWE problem.} Attack costs are given as the base-$2$ logarithm.}\label{tab:attacks}
\medskip
\centering
\renewcommand{\tabcolsep}{0.3cm}
\renewcommand{\arraystretch}{1.1}
\begin{tabular}{l|c|ccc}
\toprule
Scheme & Attack Mode & Classical & Quantum & Plausible \\
\midrule
\multirow{2}{*}{\FrodoLOne} & Primal & 150.8 & 137.6 & 109.6 \\
& Dual & 149.6 & 136.5 & 108.7 \\
\midrule
\multirow{2}{*}{\FrodoLThree} & Primal & 216.0 & 196.7 & 156.0 \\
& Dual & 214.5 & 195.4 & 154.9 \\
\midrule
\multirow{2}{*}{\FrodoLFive} & Primal & 281.6 & 256.3 & 202.6 \\
& Dual & 279.8 & 254.7 & 201.4 \\
\bottomrule
\end{tabular}
\end{center}
\end{table}
\NISTdescription{The submission package shall include a statement that summarizes the known cryptanalytic attacks on the scheme, and provides estimates of the complexity of these attacks.}

\NISTdescription{The submitter shall provide a list of references to any published materials describing or analyzing the security of the submitted algorithm or cryptosystem. When possible, the submission of copies of these materials (accompanied by a waiver of copyright or permission from the copyright holder for public evaluation purposes) is encouraged.}

In this section, we explain our methodology to estimate the security
level of our proposed parameters. The methodology is similar to the
one proposed in~\cite{USENIX:ADPS16}, with slight modifications taking
into account the fact that some quasi-linear
accelerations~\cite{AFRICACRYPT:Schneider13,BNP_IJAC16} over sieving
algorithms~\cite{SODA:BDGL16,LaarhovenThesis} are not available
without the ring structure.

We also remark that this methodology is significantly more
conservative than what is usually used in the
literature~\cite{albrecht15:_concrete_lwe}, at least since
recently. Indeed, we must acknowledge that lattice cryptanalysis is
far less mature than that for factoring and computing discrete
logarithms, for which the best-known attacks can more safely be
considered best-possible attacks.


\paragraph{Concrete parameters.}

\cpnote{See if we need to retain this paragraph, and what it would best belong.}

Concretely, for the extremely large bound $N = 2^{256}$ on the
number of discrete Gaussian samples, the threshold for Gaussian
parameters~$\alpha q$ that conform to \autoref{thm:bddwdgs-to-dlwe} is
$\sqrt{\ln(N)/(2\pi)} \approx 5.314$, which corresponds to a standard
deviation threshold of $\sqrt{\ln(N)}/(2\pi) \approx 2.120$.  Our
$\FrodoPKE$ parameters for security Levels~1 and 3, which use standard
deviation $\sigma \geq 2.3$ (see \autoref{tab:distribution}), exceed
this threshold by a comfortable margin.  (Indeed, $\sigma = 2.3$
corresponds to $N \approx 2^{300}$.) For efficiency reasons, our
parameters for security Level~5 use a somewhat smaller standard
deviation of $\sigma = 1.4$; this corresponds to the very large bound
$N \approx 2^{111}$. While this~$N$ is smaller than the running time
for the Level 5 brute-force security level, we stress that these two
quantities are not comparable; $N$ is merely a bound on the
\emph{number of samples} provided in a $\BDDwDGS$ input, and it
controls the decoding distance for known efficient algorithms. 


\subsubsection{Methodology: the core-SVP hardness}
\label{subsec:coreSVPharness}

In this section, let $\numsamp$ denote the number of LWE samples
available to the attacker. Due to the small number of samples (i.e., $\numsamp \approx n$
in our schemes) we are not concerned with either BKW types of
attacks~\cite{C:KirFou15} or linearization attacks~\cite{ICALP:AroGe11}. This essentially leaves us with two BKZ~\cite{AC:CheNgu11} attacks, usually referred to as primal and dual attacks that we will briefly recall below.


%%%%%%%%%%%%%%%%%%%%%%%%% TO CHECK
%\patrick{TODO: discuss this issue later. Lewis's email: As far as I know, BKW and linearization attack estimates are included in Martin's lattice estimator, as well as the Matzov optimizations. If we make sense of the estimator results and incorporate them/replace the estimates in the paper, we could clear both remaining issues. Otherwise, we can just remove or clarify the sentences in the screenshot I've attached.} 

%\lewis{This section is the same as 5.2.1 in the specifications document. I don't think it holds up if we care about multi-challenge security, as we can no longer assume a small number of samples.}


Formally, BKZ with block-size~$b$ requires up to polynomially many
calls to an SVP oracle in dimension~$b$, but some heuristics allow to
decrease the number of calls to be essentially
linear~\cite{ChenThesis}. To account for further improvement, we shall
count only the cost of one such call to the SVP oracle: the core-SVP
hardness. Such precaution is motivated by the fact that there are ways
to amortize the cost of SVP calls inside BKZ, especially when sieving
is to be used as the SVP oracle. Such a strategy was suggested in a
talk, but has so far not been experimentally tested, as more
implementation effort is required to integrate sieving within BKZ.

Even evaluating the concrete cost of one SVP oracle call in
dimension~$b$ is difficult, because the numerically optimized pruned
enumeration strategy does not yield a closed
formula~\cite{EC:GamNguReg10,AC:CheNgu11}. Yet, asymptotically,
enumeration is super-exponential (even with pruning), while sieving
algorithms are exponential $2^{cb + o(b)}$ with a well understood
constant $c$ in the exponent. A sound and simple strategy is therefore
to give a lower bound for the cost of an attack by $2^{cb}$ vector
operations (i.e., about $b 2^{cb}$ CPU cycles\footnote{Because of the
  additional ring structure, \cite{USENIX:ADPS16} chooses to ignore
  this factor $b$ to the advantage of the adversary, assuming the
  techniques of~\cite{AFRICACRYPT:Schneider13,BNP_IJAC16} can be
  adapted to more advanced sieve algorithms~\cite{USENIX:ADPS16}. But
  for plain LWE, we can safely include this factor.}), and to make
sure that the block-size~$b$ is in a range where enumeration costs
more than~$2^{cb}$. From the estimates of~\cite{AC:CheNgu11}, it is
argued in~\cite{USENIX:ADPS16} that this is the case both classically
and quantumly whenever $b \geq 200$.

The best known constant in the exponent for classical algorithms is
$c_{\text{C}} = \log_2 \sqrt{3/2} \approx 0.292$, as provided by the
sieve algorithm of~\cite{SODA:BDGL16}.  For quantum algorithms it is
$c_{\text{Q}} = \log_2 \sqrt{13/9} \approx
0.265$~\cite[Sec. 14.2.10]{LaarhovenThesis}. Because all variants of
the sieve algorithm require building a list of $\sqrt{4/3}^b$ many
vectors, the constant
$c_{\text{P}} = \log_2 \sqrt{4/3} \approx {.2075}$ can plausibly serve
as a ``worst-possible'' lower bound for sieving algorithms.

\paragraph{Conservatism: lower bounds vs. experiments.}

These estimates are very conservative compared to the state of the art
implementation of~\cite{mariano2017parallel}, which has practical
complexity of about $2^{0.405 b + 11}$ cycles in the range
$b=60 \dots 80$. The classical lower bound of $2^{0.292b}$ corresponds
to a margin factor of $2^{20}$ at blocksize $b=80$, and this margin
should continue increasing with the blocksize (abusing the linear fit
suggests a margin of $2^{45}$ at blocksize $b=300$).

\paragraph{Conservatism: future improvements.}

Of course, one could assume further improvements on known techniques.
At least asymptotically, it may be reasonable to assume that
$2^{0.292 b + o(b)}$ is optimal for SVP considering that the
underlying technique of~\cite{SODA:BDGL16} has been shown to reach
lower bounds for the generic nearest-neighbor search
problem~\cite{SODA:ALRW17}.  As for concrete improvements, we note
that this algorithm has already been subject to some fine-tuning
in~\cite{mariano2017parallel}, so we may conclude that there is not
much more to be gained without introducing new ideas. We therefore
consider our margin sufficient to absorb such future improvements.

\paragraph{Conservatism: cost models.}

The NIST call for proposals~\cite{NIST17} suggested a particular cost model,
inspired by the estimates of a Grover search attack on AES,
essentially accounting for the quantum gate count. In comparison, the
literature on sieving algorithms mostly focuses on analysis in the RAM
model and quantumly accessible RAM models, and considers the amount of
memory they use. Their cost in the area-time model should be higher by
polynomial, if not exponential, factors.

Firstly, our model accounts for arithmetic operations rather than
gates (used to compute inner products and evaluate norms of
vectors). The conversion to gate count may not be trivial as it is
unclear how many bits of precision are required.

Secondly, even in the classical setting, the cost of sieving in large
dimensions may not be accurately captured by the count of elementary
operations in the RAM model, as those algorithms use an exponential
amount of memory. Admittedly, the most basic sieve algorithm (with
theoretical complexity $2^{0.415b + o(b)}$) has sequential memory
access, and can therefore be efficiently implemented by a large
circuit without memory access delays. But more advanced
ones~\cite{SODA:BDGL16} have much less predictable memory access
patterns, and memory complexities as large as time complexities
($2^{0.292 b + o(b)}$). It is unclear if they can be adapted to reach
a complexity $2^{0.292 b + o(b)}$ in the area-time model; one might
expect extra polynomial factors to appear. (Following an idea
of~\cite{EPRINT:BecGamJou15}, Becker et al.~\cite{SODA:BDGL16} also
claim a version that only requires $2^{0.2015b + o(b)}$ memory, but
we suspect this would come at some hidden cost on the running time.)

Moreover, the quantum versions of all sieving algorithms work in the
quantumly accessible RAM model~\cite{LMP15}. Again, the conversion to
an efficient quantum circuit will induce extra costs---at least
polynomial ones.

\subsubsection{Primal attack}

The primal attack consists of constructing a unique-SVP instance from
the LWE problem and solving it using BKZ. We examine how large the
block dimension $b$ is required to be for BKZ to find the unique
solution. Given the matrix LWE instance
$( \bfA,\bfb = \bfA \bfs + \bfe)$ one builds the lattice
$\Lambda = \{ \bfx \in \mathbb Z^{m+n+1} : ( \bfA | \bfI_m | - \bfb)
\bfx = 0 \bmod q \}$ of dimension $d = m + n + 1$, volume $q^m$, and
with a unique-SVP solution $\bfv = (\bfs, \bfe, 1 )$ of norm
$\lambda \approx \sigma \sqrt{n+m}$. The number of used samples $m$
may be chosen between $0$ and $\numsamp$, and we numerically optimize
this choice.

Using the typical models of BKZ (geometric series assumption, Gaussian
heuristic~\cite{ChenThesis,albrecht15:_concrete_lwe}) one concludes
that the primal attack is successful if and only if
\begin{equation}
  \label{eqn:primal_attack_cond}
  \sigma \sqrt b \leq \delta^{2b-d -1} \cdot q^{m/d} \quad \text{ where } \delta = ((\pi b)^{1/b} \cdot b/(2\pi e))^{1/(2b-2)} \enspace .
\end{equation}

We note that this condition, introduced in~\cite{USENIX:ADPS16}, is
substantially different from the one suggested in~\cite{EC:GamNgu08}
and used in many previous security analyses, such
as~\cite{albrecht15:_concrete_lwe}. The recent study~\cite{AC:AGVW17}
showed that this new condition predicts significantly smaller security
levels than the older, and is corroborated by extensive experiments.

\subsubsection{Dual attack}
\label{sec:dual-attack}

The dual attack searches for a short nonzero vector in the lattice
$\hat \Lambda = \{ (\bfv, \bfw) \in \mathbb{Z}^{n + m} : \bfv^{t} =
\bfw^{t} \bfA \pmod q\}$ of dimension $d=n+m$ and volume~$q^{n}$,
which is generated by the rows of the basis matrix
\[ \bfB =
  \begin{pmatrix}
    q \bfI_{n} & \\ \bar \bfA & \bfI_{m}
  \end{pmatrix}  \in \mathbb{Z}^{(n+m) \times (n+m)} ,
\]
where each entry of~$\bar \bfA$ is an arbitrary mod-$q$ integer
representative of the corresponding entry of~$\bfA$.  As above, the
BKZ algorithm with block size~$b$ will output such a vector of length
$\ell \approx \delta^{d-1} q^{n/d}$. The dual attack then uses this
vector as a distinguisher for LWE, as described next.

For convenience we actually analyze the attack against a
\emph{continuous} form of LWE, with Gaussian~$\bfs, \bfe$ over the
reals~$\mathbb{R}$. An instance of this problem has the form
$(\bfA, \bfb)$, where $\bfb \in (\mathbb{R}/q\mathbb{Z})^{n + m}$
either is uniformly random and independent of~$\bfA$, or has the form
\[
  \bfb =
  \begin{pmatrix}
    \bfb_{1} \\ \bfb_{2}
  \end{pmatrix}
  = \bfB \cdot
  \begin{pmatrix}
    \bfs \\ \bfe
  \end{pmatrix} =
  \begin{pmatrix}
    q \cdot \bfs \\ \bar \bfA \bfs + \bfe
  \end{pmatrix}
  \bmod q \mathbb{Z}^{n + m} , \] where the entries of
$(\bfs, \bfe) \in \mathbb{R}^{n + m}$ are independent continuous
Gaussians of standard deviation~$\sigma$. Because there is a trivial
reduction from this LWE decision problem to the discrete one of
interest that has \emph{rounded Gaussian}~$\bfs, \bfe$
(over~$\mathbb{Z}$), the latter problem is no easier than the
former.\footnote{The reduction just replaces $\bfb$ with
  $\round{\bfb_{2} - \bar \bfA (\bfb_{1}/q \bmod \mathbb{Z}^{n})} \in
  \mathbb{Z}_{q}^{m}$, where the $\bmod$ operation returns the unique
  representative (i.e., fractional part) in $[-1/2, 1/2)^{n}$, and
  $\round{\cdot}$ rounds to the nearest integer by subtracting the
  fractional part. This reduction maps the (continuous) uniform
  distribution to the (discrete) uniform distribution, and in the LWE
  case, subtracts $\bar \bfA$ times the fractional part of~$\bfs$ and
  rounds away the fractional part of~$\bfe$, yielding
  $\bar \bfA \round{\bfs} + \round{\bfe} \bmod q\bbZ^{m}$.}

Having found some $(\bfv,\bfw) \in \hat\Lambda$ of length $\ell$, the
attacker computes
\[ z = (\bfv^{t}, \bfw^{t}) \cdot \bfB^{-1} \cdot \bfb = q^{-1}
  (\bfv^{t} - \bfw^{t} \bar \bfA) \cdot \bfb_{1} + \bfw^{t} \cdot
  \bfb_{2} \bmod q , \] and attempts to distinguish it from uniformly
random over $\mathbb{R}/q\mathbb{Z}$. Its advantage in doing so can be
bounded as follows. First, suppose that $\bfb = \bfB \cdot \left(
  \begin{smallmatrix}
    \bfs \\ \bfe
  \end{smallmatrix}
\right)$ is from a continuous LWE instance as defined above. Then
\[ z = (\bfv^{t}, \bfw^{t}) \cdot
  \begin{pmatrix}
    \bfs \\ \bfe
  \end{pmatrix} \bmod q \] is distributed as a Gaussian of standard
deviation $\ell \sigma$, modulo~$q$. On the other hand, if~$\bfb$ is
uniformly random, then~$z$ is uniformly random
in~$\mathbb{R}/q\mathbb{Z}$. By the results
of~\cite{DBLP:journals/siamcomp/MicciancioR07}, these two
distributions have statistical distance at most $\eps = 2\delta$ for
$\delta = \exp(-2\pi^2 \tau^2) < 1/8$ where $\tau = \ell \sigma / q$,
so this~$\eps$ bounds the distinguishing advantage.

Because the value~$\mu$ encrypted by the underlying $\FrodoPKE$ (using
LWE) actually serves as a seed to pseudorandomly generate the
$\FrodoKEM$ KEM key $\ssk$ (see \autoref{alg:KEM:Encaps}), a small
advantage~$\eps$---say, below $1/2$---in distinguishing~$\mu$ from a
uniform random string does not significantly decrease the brute-force
search space.
% Note that small advantages~$\eps$ are not meaningful to attack a KEM:
% \cpnote{I'm not sure this argument holds much water, since with
%   FrodoKEM one would just directly use the KEM key it produces. Maybe
%   we should make the argument that for distinguishing problems, we
%   need a $1/\eps^{2}$ factor to measure bit security?}
% \dsnote{This paragraph is also a bit confusing to me.  The security notion we have to meet for the KEM is IND, and distinguishing the LWE instance with advantage $\epsilon$ basically means distinguishing KEM keys with advantage $\epsilon$.  This paragraph more seems to be arguing IND isn't the right notion, since we're going to compose the KEM key with a symmetric encryption scheme, and so you really need session key recovery (one way ness, OW) in order to break the symmetric encryption scheme.  There is something to that argument, but OW is not what we were asked to provide, IND is.  I don't really understand the last part about the $2^{.2075b}$ vectors and how that fits in, so I don't have a comment on that.}
% because the
% agreed-upon key is to be used as a symmetric cipher key, any advantage
% below $1/2$ does not significantly decrease the brute-force search
% space. (To make this more formal, one may simply hash the agreed-upon
% key with a random oracle before using it for any other purpose.)
%
We therefore require the attacker to amplify its distinguishing
advantage by obtaining about $1/\eps^2$ short lattice vectors, which
we can model (most favorably to the attacker) as being Gaussian
distributed and independent.
%
% \cpnote{This amplification argument is not satisfactory
%   either, because we lack \emph{independence} due to the fixed choice
%   of $\bfe$. If we model the sieve algorithms as producing independent
%   Gaussian-distributed vectors then the inner products are
%   independent. But I don't know if this is a legitimate model, or if
%   we even would want to use it. Just using a $1/\eps^{2}$ factor for
%   bit security in distinguishing problems seems like the better way to
%   go.}
Because the lattice-sieve algorithms provide about $2^{.2075 b}$
vectors, the dual attack must be repeated at least
$R = \max(1, 1/(2^{.2075 b} \eps^2))$ times. (This view is also
favorable to the attacker, as the other vectors output by the sieving
algorithm are a bit larger than the shortest one.)

Primal and dual attacks for our suggested parameters are given in
\autoref{tab:attacks}.  The costs are listed for a single instance of
the LWE problem. (Our ultimate security claims, such as those listed
in \autoref{tab:security}, result from a series of reductions and thus
are weaker.)




\subsubsection{Beyond core-SVP hardness}
\label{sec:beyond-core}

At the time the core-SVP hardness measure was introduced by~\cite{USENIX:ADPS16}, the best implementations of sieving~\cite{SODA:BDGL16,mariano2017parallel} had performance significantly worse than the $2^{.292b}$ CPU cycles proposed as a conservative estimate by this methodology. This was due to substantial polynomial, or even sub-exponential, overheads hidden in the $2^{.292b + o(b)}$ complexity given in the analysis of~\cite{SODA:BDGL16}. Before~\cite{USENIX:ADPS16}, security estimates of lattice schemes were typically based on the cost of solving SVP via enumeration as given in~\cite{AC:CheNgu11,albrecht15:_concrete_lwe}, leading to much more aggressive parameters. Beyond affecting the cost of SVP-calls, this methodology also introduced a different prediction of when BKZ solves LWE which was later confirmed by~\cite{AC:AGVW17} and refined in~\cite{dachman2020lwe}.

While doubts were expressed to whether enumeration~\cite{AC:CheNgu11} with its super-exponential, yet practically smaller, costs would ever be outperformed by sieving for relevant cryptographic dimensions, significant progress on sieving algorithms~\cite{ducas2018shortest,albrecht2019general} has brought the cross-over point down to dimension about $b = 80$. In fact, the current SVP records are now held by algorithms that employ sieving.\footnote{\url{https://www.latticechallenge.org/svp-challenge/}} These developments mandate a revision and refinement of security estimates for the \FrodoKEM parameters, especially regarding classical attacks. In particular, while experiments indicated that, before those improvements were made, the costs hidden in the $o(b)$ were positive both in practice and asymptotically, the dimensions-for-free technique of~\cite{ducas2018shortest} offers a sub-exponential speed-up, making it a priori unclear whether the total $o(b)$ term is positive or negative, both asymptotically and concretely.

We follow the same methodology as the one detailed in the Round-3 specification document of the Kyber key encapsulation mechanism \cite{EuroSP:Kyber} to count the number of gates required to solve the LWE problem. This analysis refines the core-SVP methodology in the following ways:
\begin{itemize}
	\item It uses the probabilistic simulation of~\cite{dachman2020lwe} rather than the GSA-intersect model of~\cite{USENIX:ADPS16,AC:AGVW17} to determine the BKZ blocksize $b$ for a successful attack. The required blocksize is somewhat larger (by an added term between $10$ and $25$ for our parameters), because of a ``tail'' phenomenon~\cite{yu2017second}.
	\item It accounts for the ``few dimensions for free'' introduced in~\cite{ducas2018shortest}, which permits to solve SVP in dimension $b$ while running sieving in a somewhat smaller dimension $b' = b-o(b)$.
	\item It relies on the concrete estimation for the gate cost of sieving from~\cite{albrecht2019estimating}.
	\item It accounts for the number of calls to the SVP oracle.
	\item It dismisses the dual attack as realistically more expensive than the primal one, noting that the analysis of~\cite{USENIX:ADPS16} (also used here) assumes that the sieve provides $2^{.208b}$ many vectors as short as the shortest one. But first, most of those vectors are larger by a factor of $\sqrt{4/3}$, and second, the trick of exploiting all of those vectors is not compatible with the ``dimension-for-free'' trick of~\cite{ducas2018shortest}.
\end{itemize}
The scripts for these refined estimates are provided in a git branch of the leaky-LWE-estimator of~\cite{dachman2020lwe},\footnote{\url{https://github.com/lducas/leaky-LWE-Estimator/tree/NIST-round3}} and lead to the estimates given in Table~\ref{tab:refined_LWE}. We refer to the Kyber Round-3 specification document for the details of this analysis. We point out that it is paired with a detailed discussion of the ``known unknowns'', providing a plausible confidence interval for these estimates. For the classical hardness of the LWE problem at Levels 1 and 2, it is estimated in the Kyber Round-3 document that the true cost is no more than $16$ bits away from this estimate, in either direction. 

We also note that a similarly refined count of quantum gates seems to be essentially irrelevant: the work of \cite{albrecht2019estimating} concluded that obtaining a quantum speed-up for sieving is rather tenuous, while the quantum security target for each level is significantly lower than the classical target. 


\begin{table}[H]
\begin{center}
\caption{Refined estimates for the LWE hardness, where $n$ is the optimal lattice dimension for the attack,  
$b$ the BKZ blocksize and $b'$ the sieving dimension accounting for ``dimensions for free''}\label{tab:refined_LWE}
\medskip
\centering
\renewcommand{\tabcolsep}{0.3cm}
\renewcommand{\arraystretch}{1.1}
\begin{tabular}{l|c|c|c|c|c}
\toprule
             & $n$    &  $b$  & $b'$  & $\log_2(\text{gates})$   & $\log_2(\text{memory in bits})$ \\ 
\midrule
\FrodoLOne 	& 1297	& 496	& 453	& 175.1 	& 110.4 \\
\FrodoLThree 	& 1969	& 724	& 668	& 240.0 	& 155.8 \\
\FrodoLFive 	& 2634	& 957	& 888	& 305.4 	& 202.1 \\
\bottomrule
\end{tabular}
\end{center}
\end{table}

The refined analysis above covers recent improvements for solving the SVP and models the current state of the art. The resulting gate counts show that the \FrodoKEM parameter sets comfortably match their respective target security levels with a large margin. While these numbers might encourage parameter modifications to more closely match security targets and improve performance and bandwidth, we prefer to leave parameter sets unchanged. This aligns with \FrodoKEM's conservative design approach and hedges against improvements for cryptanalytic algorithms solving general lattice problems. 


\subsubsection{Decryption failures}\label{sec:failures}

The concrete $\FrodoPKE$ parameters induce a tiny probability of
incorrect decryption (see \autoref{tab:security}), for honestly
generated keys and ciphertexts. This is because a ciphertext may
decrypt to a different message than the encrypted one, if the
combination of the short error matrices in the key and the ciphertext
is too large (see \autoref{sec:cpa-pke-correctness}).  This
aspect of the scheme carries over to the transformed, CCA-targeting
$\FrodoKEM$, where incorrect decryption in the underlying PKE
typically causes a decryption failure.

It has long been well understood that the ability to induce incorrect
decryption or decryption failure in LWE-based schemes can leak
information about the secret key, up to and including full key
recovery (with sufficiently many failures). In brief, this is because
such failures indicate some correlation between the secret key and the
encryption randomness.

In the context of chosen-ciphertext attacks on $\FrodoKEM$ and
similarly transformed schemes, the attacker can attempt to create
ciphertexts whose underlying error matrices---which are derived
pseudorandomly using an attacker-chosen seed---are atypically
large. Such ``weak'' ciphertexts have an increased probability of
inducing decryption failures when submitted to a decryption
oracle. The process of searching for such ciphertexts, which can be
done offline (without using a decryption oracle), is known as
``failure boosting.''

In 2019, D'Anvers et al.~\cite{PKC:DGJNVV19} performed a detailed study of
the complexity of failure-boosting attacks (in both the classical and
quantum setting) against a variety of NIST candidates, including
$\FrodoKEM$. In summary, they found that the Level 3
parameterization $\FrodoLThree$ suffered \emph{no loss} in its
claimed security (either classical or quantum) under such
attacks. This is essentially because the cost of finding weak
ciphertexts exceeds the benefit obtained from the corresponding
increase in decryption failure probability.

We ran the scripts from~\cite{PKC:DGJNVV19} on the
parameters for $\FrodoLOne$, $\FrodoLThree$, and $\FrodoLFive$ described in this work, and
confirmed that applying the failure-boosting attack does not violate security
Levels 1, 3, and 5, respectively. (Note that for $\FrodoLFive$,
failure boosting did not provide any improvement over the intrinsic
failure probability of $2^{-252.5}$. We consider this to be consistent
with the Level 5 requirement of 256 bits of brute-force security, because the overhead in using
decryption failures to win the CCA security game exceeds 3.5 bits.)


%\lewis{I deleted the section on multi-target and multi-ciphertext security. I rolled these two into a single notion of multi-challenge security and included it into the main text. I may be biased, since multi-challenge security is something I have worked on, and may have dedicated too much of the paper to explaining why it is significant. I do, however, think it is an important selling point for FrodoKEM, and important to get across why we are submitting this paper now.}
%
%
%
%\lewis{I deleted the subsection "Proofs for the salted FO transform. We don't need to explain how to adapt theorem 3.1 and theorem 3.2 for our purposes, because the multi-challenge security theorems give us single target security bounds as well. One only have to set $n$ and $u$ to be 1. Instead I included the multi-challenge security bounds in the main text.}



%%% Local Variables:
%%% mode: latex
%%% TeX-master: "Main"
%%% End:


\end{document}

%%% Local Variables:
%%% mode: latex
%%% TeX-master: t
%%% End:
