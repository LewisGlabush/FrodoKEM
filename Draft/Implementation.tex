\section{Implementation and performance analysis}%
\label{sec:performance}

\ifshoworiginal
The submission package includes:
\begin{itemize}
\item a reference implementation written exclusively in Python,
\item a reference implementation written exclusively in portable C,
\item an optimized implementation written exclusively in portable C that includes efficient algorithms to generate the matrix $\bfA$ and to compute the matrix operations $\bfA \bfS + \bfE$ and $\bfS' \bfA + \bfE'$, and
\item an additional, optimized implementation for x64 platforms that exploits Advanced Vector Extensions~2 (AVX2) intrinsic instructions.
\end{itemize}

The implementations in the submission package support all three security levels and both variants of matrix generation:
$\FrodoKEMLOneAES$, $\FrodoKEMLOneSHAKE$, $\FrodoKEMLThreeAES$, $\FrodoKEMLThreeSHAKE$, $\FrodoKEMLFiveAES$, and $\FrodoKEMLFiveSHAKE$. 
The only difference between the reference and the optimized implementation is that the latter includes two efficient functions to generate the public matrix $\bfA$ and to compute the matrix operations $\bfA \bfS + \bfE$ and $\bfS' \bfA + \bfE'$. Similarly, the only difference between the optimized and the additional implementation is that the latter uses AVX2 intrinsic instructions to speed up the implementation of the aforementioned functions. 
Hence, the different implementations share most of their codebase: this illustrates the simplicity of software based on $\FrodoKEM$. 

\else

An important feature of \FrodoKEM is that it is easy to implement and naturally facilitates
writing implementations that are compact and run in constant-time.
This latter feature aids to avoid common cryptographic implementation mistakes which can lead
to key-extraction based on, for instance, timing differences when executing the code.\footnote{Nonetheless,
care must be taken to avoid timing leaks. In 2020, Guo, Johansson, and Nilsson~\cite{C:GuoJohNil20}
demonstrated a key-recovery attack on the reference implementation in the Round 2 submission of \FrodoKEM
by exploiting branching in the computation of $\ssk$ in \FrodoKEM.\Decaps. This attack can be avoided
by ensuring the implementation reads both $\bfk'$ and $\bfs$, compares $\bfB' \| \bfC$ and $\bfB'' \| \bfC'$
in a constant-time way that avoids early termination, and sets $\overline{\bfk}$ using data-independent evaluation.}

Our compact implementation of the \FrodoKEM scheme consists of slightly more than 250 lines of plain C code.\footnote{Our reference and optimized implementations in C are available as part of this submission here: \url{https://doi.org/10.5281/zenodo.14633189}.}
This same code is used for all three security levels to implement \FrodoKEMLOne, \FrodoKEMLThree, and \FrodoKEMLFive,
with parameters changed by a small number of macros at compile-time.
Moreover, most of the code is either shared or reused for our implementation of \eFrodoKEM.
We remark that the separation in two implementations, one for \FrodoKEM and one for \eFrodoKEM, is only
done to provide a simpler and cleaner codebase supporting each API. 
In particular, the API for \eFrodoKEM has been customized to perform a \emph{single} key generation
per encapsulation execution.    

Computing on matrices---the basic operation in \FrodoKEM---allows for easy scaling to different dimensions $n$.
In addition, \FrodoKEM uses a modulus $q$ that is always equal or less than $2^{16}$. These two combined
aspects allow for the full reuse of the matrix functions for the different security levels by instantiating them
with the right parameters at build time. Since the modulus $q$ used is always a power of two, implementing
arithmetic modulo $q$ is simple, efficient and easy to do in constant-time in modern computer architectures:
for instance, computing modulo $2^{16}$ comes for free when using 16-bit data-types. Moreover, the dimension
values were chosen to be divisible by 16 in order to facilitate vectorization optimizations and to simplify the
use of $\AESOneTwoEight$ for the generation of the matrix $\bfA$.

Also the error sampling is designed to be simple and facilitates code reuse: for any security level, \FrodoKEM
requires 16 bits per sample, and the tables $T_\chi$ corresponding to the discrete cumulative density functions
always consist of values that are less than $2^{15}$. Hence, a simple function applying inversion sampling
(see~\autoref{alg:samplechi}) can be instantiated using precomputed tables $T_\chi$. Moreover, due to
the small sizes of these precomputed tables constant-time table lookups, needed to protect against attacks
based on timing differences, can be implemented almost for free in terms of effort and performance impact.
\fi

All our implementations avoid the use of secret address accesses and secret branches and, hence,
are protected against timing and cache attacks at the software level.


\subsubsection{Performance analysis on x64 Intel}\label{sec:results_x64}

\ifshoworiginal
In this section, we summarize the results of our performance evaluation using a machine equipped with a 3.4GHz Intel Core i7-6700 (Skylake) processor and running Ubuntu 16.04.3 LTS. As standard practice, TurboBoost was disabled during the tests. For compilation we used GNU GCC version 7.2.0 with the command {\tt gcc -O3 -march=native}. 
The generation of the matrix $\bfA$ is the most expensive part of the computation. As described in \autoref{sec:spec:genA}, we support two ways of generating $\bfA$: one using $\AESOneTwoEight$ and one using $\SHAKE128$.


\subsubsection{Performance using AES128}\label{sec:results_aes}

\autoref{tab:results_x64_aes} details the performance of the optimized implementations and the additional x64 implementations when using $\AESOneTwoEight$ for the generation of the matrix $\bfA$. The top two sets of results correspond to performance when using OpenSSL's AES implementation\footnote{Note that in order to enable AES-NI instructions in OpenSSL, we use the \texttt{EVP\_aes\_128\_ecb} interface in OpenSSL.} and the bottom set presents the results when using a standalone AES implementation using Intel's Advanced Encryption Standard New Instructions (AES-NI).

As can be observed, the different implementation variants have similar performance, even when using hand-optimized AVX2 intrinsic instructions. This illustrates that $\FrodoKEM$'s algorithms, which are mainly based on matrix operations, facilitate automatic parallelization using vector instructions. Hence, the compiler is able to achieve close to ``optimal'' performance with little intervention from the programmer.
The best results for $\FrodoKEMLOneAES$, $\FrodoKEMLThreeAES$ and $\FrodoKEMLFiveAES$ (i.e., 0.93~ms, 1.75~ms and 2.91~ms, respectively, obtained by adding the times for encapsulation and decapsulation) are achieved by the additional implementation using AVX2 intrinsics. However, the difference in performance between the different implementations reported in \autoref{tab:results_x64_aes} is, in all the cases, less than~0.5\%.

We note that the performance of $\FrodoKEM$ using AES on Intel platforms greatly depends on AES-NI instructions. For example, when turning off the use of these instructions the computing cost of the optimized implementation of $\FrodoKEMLOneAES$ (resp.,~$\FrodoKEMLThreeAES$) is 26.4~ms (resp.,~60.9~ms), which is roughly a 28-fold (resp.,~35-fold) degradation in performance.


\begin{table}[t]
\caption{\textbf{Performance (in thousands of cycles) of $\FrodoKEM$ on a 3.4GHz Intel Core i7-6700 (Skylake) processor with matrix $\bfA$ generated using $\AESOneTwoEight$.} Results are reported using OpenSSL's AES implementation and using a standalone AES implementation, all of which exploit AES-NI instructions. Cycle counts are rounded to the nearest $10^3$ cycles.
\patrick{POSSIBLY, UPDATE RESULTS.}}\label{tab:results_x64_aes}
\medskip
\centering
\renewcommand{\tabcolsep}{0.4cm}
\renewcommand{\arraystretch}{1.1}
\begin{tabular}{l|c c c|c}
\toprule
\multirow{2}{*}{\textbf{Scheme}}     &     \multirow{2}{*}{\textbf{KeyGen}}      &    \multirow{2}{*}{\textbf{Encaps}}   &    \multirow{2}{*}{\textbf{Decaps}}   &    \textbf{Total}        \\ 
                                                       &                                                            &                                                         &                                                       &    \textbf{(Encaps + Decaps)}   \\
\midrule
\multicolumn{5}{l}{\bf Optimized Implementation (AES from OpenSSL)} \\
\midrule
$\FrodoKEMLOneAES$                               &            1,389                &            1,637                   &                 1,534         &                3,171             \\
$\FrodoKEMLThreeAES$                             &            2,831                &            3,047                   &                 2,904       &                5,951             \\
$\FrodoKEMLFiveAES$                             &            4,775                &            5,063                   &                 4,840       &                 9,903             \\
\midrule
\multicolumn{5}{l}{\bf Additional implementation using AVX2 intrinsic instructions (AES from OpenSSL)} \\
\midrule
$\FrodoKEMLOneAES$                               &            1,387                &            1,634                   &                 1,531       &                3,165             \\
$\FrodoKEMLThreeAES$                             &            2,846                &            3,047                   &                2,894       &                5,941             \\
$\FrodoKEMLFiveAES$                             &            4,779                &            5,051                   &                 4,849       &                 9,900             \\
\midrule
\multicolumn{5}{l}{\bf Additional implementation using AVX2 intrinsic instructions (standalone AES)} \\
\midrule
$\FrodoKEMLOneAES$                               &            1,398                &            1,644                   &                 1,540       &                3,184             \\
$\FrodoKEMLThreeAES$                             &            2,874                &            3,054                   &                 2,908       &                5,962             \\
$\FrodoKEMLFiveAES$                             &            4,765                &            5,069                   &                 4,867       &                9,936             \\
\bottomrule
\end{tabular}
\end{table}


\subsubsection{Performance using SHAKE128}\label{sec:results_shake}

\autoref{tab:results_x64_shake} outlines the performance figures of the optimized implementations and the additional x64 implementations when using $\SHAKE128$ for the generation of the matrix $\bfA$. 
The top set of results shows the performance of the optimized implementation written in C only, while the bottom set presents the results when using a 4-way implementation of SHAKE using AVX2 instructions (``SHAKE4x using AVX2'').
Note that the use of such a vectorized implementation of SHAKE is necessary to boost the practical performance. In our use-case, it results in a two-fold speedup when compared to the version using a SHAKE implementation written in plain C.

Comparing \autoref{tab:results_x64_aes} and \autoref{tab:results_x64_shake}, $\FrodoKEM$ using AES, when implemented with AES-NI instructions, is around 2.6--3.1$\times$ faster than the vectorized SHAKE implementation. Nevertheless, this comparative result could change drastically if hardware-accelerated instructions such as AES-NI are not available on the targeted platform, or if support for hardware-accelerated instructions for SHA-3 is added in the future. 


\begin{table}[t]
\caption{\textbf{Performance (in thousands of cycles) of $\FrodoKEM$ on a 3.4GHz Intel Core i7-6700 (Skylake) processor with matrix $\bfA$ generated using $\SHAKE128$.} Results are reported for two test cases: (i) using a SHAKE implementation written in plain C and, (ii) using a 4-way implementation of SHAKE using AVX2 instructions. Cycle counts are rounded to the nearest $10^3$ cycles.}\label{tab:results_x64_shake}
\medskip
\centering
\renewcommand{\tabcolsep}{0.4cm}
\renewcommand{\arraystretch}{1.1}
\begin{tabular}{l|c c c|c}
\toprule
\multirow{2}{*}{\textbf{Scheme}} & \multirow{2}{*}{\textbf{KeyGen}} & \multirow{2}{*}{\textbf{Encaps}} & \multirow{2}{*}{\textbf{Decaps}} & \textbf{Total}             \\ 
                                 &                                  &                                  &                                  & \textbf{(Encaps + Decaps)} \\ 
\midrule
\multicolumn{5}{l}{\bf Optimized Implementation (plain C SHAKE)} \\
\midrule
$\FrodoKEMLOneSHAKE$            &           7,649                  &           7,847                  &                7,743             &              15,590        \\
$\FrodoKEMLThreeSHAKE$          &          16,874                  &          17,101                  &               16,971           &              34,072        \\
$\FrodoKEMLFiveSHAKE$          &          30,465                  &          30,626                  &               30,451             &              61,077        \\
\midrule
\multicolumn{5}{l}{\bf Additional implementation using AVX2 intrinsics (SHAKE4x using AVX2)} \\
\midrule
$\FrodoKEMLOneSHAKE$            &           4,031                  &           4,218                  &                4,116             &               8,334        \\
$\FrodoKEMLThreeSHAKE$          &           8,599                  &           8,799                  &                8,659             &              17,458        \\
$\FrodoKEMLFiveSHAKE$          &          15,067                  &          15,338                  &               15,170             &              30,508        \\
\bottomrule
\end{tabular}
\end{table}
\fi

\autoref{tab:results_x64} summarizes the results of our performance evaluation using a machine equipped
with a 3.2GHz Intel Core i7-8700 (Coffee Lake) processor and running Ubuntu 22.04.2 LTS. Following standard practice,
TurboBoost was disabled during the tests. For compilation we used GNU GCC version 15.0.1 with the command
{\tt gcc -O3 -march=native}. 
%The generation of the matrix $\bfA$ is the most expensive part of the computation. As described in
%\autoref{sec:spec:genA}, we support two ways of generating $\bfA$: one using $\AESOneTwoEight$ and
%one using $\SHAKE128$.

For the case of generating the matrix $\bfA$ using $\AES128$, we present the results when using
an AES implementation that exploits the cryptographic extension set AES-NI.
The corresponding running times for $\FrodoKEMLOneAES$, $\FrodoKEMLThreeAES$ and $\FrodoKEMLFiveAES$ are
0.67~ms, 1.28~ms and 2.17~ms, respectively, obtained by adding the times for encapsulation and decapsulation.
This performance is expected to be typical in static key exchange applications where the cost of key generation
is amortized across many key encapsulation executions. 
For the full KEM, the running times are 0.97~ms, 1.91~ms and 3.22~ms, respectively. 
These timings roughly match the cost of \eFrodoKEM in an ephemeral setting (the overhead is about 1\% or less).

Our implementation also includes the optional use of AVX2 intrinsic instructions. In our experiments, we observed that
this optimization offers a very small performance improvement compared to the plain C implementation. 
This illustrates that $\FrodoKEM$'s algorithms, which are mainly based on matrix operations, facilitate
automatic parallelization using vector instructions. Hence, the compiler is able to achieve close to ``optimal''
performance with little intervention from the programmer.

We note that the performance of $\FrodoKEM$ using AES on Intel platforms greatly depends on AES-NI instructions.
For example, when turning off the use of these instructions the computing cost of the optimized implementations
suffers a more than 20-fold increase.

\autoref{tab:results_x64} also outlines the performance figures of our implementation when using $\SHAKE128$
for the generation of $\bfA$. 
In this case, we use a 4-way implementation of SHAKE that exploits AVX2 instructions.
In our tests, we observed that this approach results in a two-fold speedup when compared to a version using a SHAKE implementation
written in plain C.  
%The middle set of results shows the performance of the implementation written in C only, while the bottom
%set presents the results when using a 4-way implementation of SHAKE using AVX2 instructions.
%Note that the use of such a vectorized implementation of SHAKE is necessary to boost the practical performance.
%In our use-case, it results in a two-fold speedup when compared to the version using a SHAKE implementation
%written in plain C.

Comparing the use of $\AESOneTwoEight$ and $\SHAKE128$, $\FrodoKEM$ using AES, when implemented with AES-NI instructions,
is around 3$\times$ faster than $\FrodoKEM$ using SHAKE with a vectorized implementation. Nevertheless, this comparative
result may change drastically if hardware-accelerated instructions such as AES-NI are not available on the
targeted platform, or if support for hardware-accelerated instructions for SHA-3 is added in the future. 

\begin{table}[t]
\caption{Performance (in thousands of cycles) of $\FrodoKEM$ on a 3.2GHz Intel Core i7-8700 (Coffee Lake) processor.
For the variants using $\AESOneTwoEight$, results are reported using an AES implementation that exploits AES-NI instructions. 
For the variants using $\SHAKE128$, results are reported using a 4-way vectorized implementation of SHAKE using AVX2 instructions.
Cycle counts are rounded to the nearest $10^3$ cycles.}\label{tab:results_x64}
\cpnote{see about using siunitx to get nicer display and alignment of numbers here and in other tables}
\medskip
\centering
\renewcommand{\tabcolsep}{0.25cm}
\renewcommand{\arraystretch}{1.1}
\begin{tabular}{l|c c c|c}
\toprule
\multirow{2}{*}{\textbf{Scheme}}     &     \multirow{2}{*}{\textbf{KeyGen}}      &    \multirow{2}{*}{\textbf{Encaps}}   &    \multirow{2}{*}{\textbf{Decaps}}   &    \textbf{Total}        \\ 
                                   &                                             &                                       &
                                   &    \textbf{(Encaps + Decaps)}   \\
\midrule
\multicolumn{5}{l}{\bf AES using AES-NI} \\
\midrule
$\FrodoKEMLOneAES$                               &            938                &            1,105                   &                 1,044         &                2,149             \\
$\FrodoKEMLThreeAES$                             &            2,017                &            2,105                   &                 1,983       &                4,088             \\
$\FrodoKEMLFiveAES$                             &            3,353                &            3,597                   &                 3,326       &                 6,923             \\
%\midrule
%\multicolumn{5}{l}{\bf Plain C SHAKE} \\
%\midrule
%$\FrodoKEMLOneSHAKE$            &           7,649                  &           7,847                  &                7,743             &              15,590        \\
%$\FrodoKEMLThreeSHAKE$          &          16,874                  &          17,101                  &               16,971           &              34,072        \\
%$\FrodoKEMLFiveSHAKE$          &          30,465                  &          30,626                  &               30,451             &              61,077        \\
\midrule
\multicolumn{5}{l}{\bf Vectorized SHAKE using AVX2} \\
\midrule
$\FrodoKEMLOneSHAKE$            &           2,806                  &           2,941                  &                2,877             &               5,818        \\
$\FrodoKEMLThreeSHAKE$          &           6,026                  &           6,096                  &                5,994             &              12,090        \\
$\FrodoKEMLFiveSHAKE$          &          10,725                  &          10,643                &               10,497             &              21,140        \\
\bottomrule
\end{tabular}
\end{table}


\subsubsection{Performance analysis on ARM}\label{sec:results_arm}

\ifshoworiginal
In this section, we summarize the results of our performance evaluation using a device powered by a 1.992GHz 64-bit ARM Cortex-A72 (ARMv8) processor and running Ubuntu 16.04.2 LTS. For compilation we used GNU GCC version 5.4.0 with the command {\tt gcc -O3 -march=native}. 

\autoref{tab:results_arm} details the performance of the optimized implementations when using $\AESOneTwoEight$ and $\SHAKE128$. Similar to the case of the x64 Intel platform, the overall performance of $\FrodoKEM$ is highly dependent on the performance of the primitive that is used for the generation of the matrix $\bfA$. Hence, the best performance in this case is achieved when using OpenSSL's AES implementation, which exploits the efficient NEON engine. On the other hand, $\SHAKE$ performs significantly better when there is no support for specialized instructions in the targeted platform: using a plain C version of $\SHAKE$ is more than 3 times faster than using a plain C version of AES.

\else

\autoref{tab:results_arm} details the performance of our implementations on a device powered by a 1.992GHz 64-bit ARM Cortex-A72 (ARMv8)
processor and running Ubuntu 24.04.2 LTS.
We provide three options which, again, are determined by the way we generate matrix $\bfA$. The first option uses OpenSSL's AES engine for implementing and accelerating $\AES128$ with the AES cryptographic extensions available on the targeted ARMv8 processor. This implementation uses OpenSSL version 3.0.13 and was compiled with GNU GCC version 14.2.0.
The other two options use plain C implementations of AES and SHAKE for generating $\bfA$ with $\AES128$ and $\SHAKE128$, respectively. 
For these cases, we use OpenSSL version 3.0.2 and compiled the implementations with GNU GCC version 13.1.0.
For all the options we used the command {\tt gcc -O3 -march=native}. 
Similar to the case of the x64 Intel platform, the overall performance of $\FrodoKEM$ is highly
dependent on the performance of the primitive that is used for the generation of the matrix $\bfA$.
Hence, the best performance in this case is achieved when using an AES implementation that
exploits the hardware acceleration provided by the ARMv8 cryptographic extensions.
The respective running times for the full KEM are 4.98~ms, 9.99~ms and 17.24~ms for $\FrodoKEMLOneAES$, $\FrodoKEMLThreeAES$ and $\FrodoKEMLFiveAES$, respectively
(as above, these timings only have a negligible overhead in comparison to the cost of \eFrodoKEM in an ephemeral setting).
On the other hand, $\SHAKE$ performs significantly better when
there is no support for specialized instructions in the targeted platform: using a plain C version
of $\SHAKE$ is more than 3$\times$ faster than using a plain C version of AES.
\fi

\begin{table}[t]
\caption{Performance (in thousands of cycles) of $\FrodoKEM$ on a 1.992GHz 64-bit ARM Cortex-A72 (ARMv8) processor. Results are reported for three test cases: (i) using an AES implementation exploiting cryptographic extensions, (ii) using an AES implementation written in plain C, and (iii) using a $\SHAKE$ implementation written in plain C. Results have been scaled to cycles using the nominal processor frequency. Cycle counts are rounded to the nearest $10^3$ cycles.}\label{tab:results_arm}
\medskip
\centering
\renewcommand{\tabcolsep}{0.25cm}
\renewcommand{\arraystretch}{1.1}
\begin{tabular}{l|c c c|c}
\toprule
\multirow{2}{*}{\textbf{Scheme}}     &     \multirow{2}{*}{\textbf{KeyGen}}      &    \multirow{2}{*}{\textbf{Encaps}}   &    \multirow{2}{*}{\textbf{Decaps}}   &    \textbf{Total}        \\ 
                                                       &                                                            &                                                         &                                                       &    (Encaps + Decaps)   \\
\midrule
\multicolumn{5}{l}{\bf AES using cryptographic extensions} \\
\midrule
$\FrodoKEMLOneAES$                               &            3,109                &            3,402                   &                 3,414         &                6,816             \\
$\FrodoKEMLThreeAES$                             &            6,515                &            6,702                   &                6,678       &                13,380             \\
$\FrodoKEMLFiveAES$                             &            11,010                &            11,796                   &                11,527       &                23,323             \\
\midrule
\multicolumn{5}{l}{\bf Plain C AES} \\
\midrule
$\FrodoKEMLOneAES$                               &            47,776                &           48,007                   &               47,958       &              95,965             \\
$\FrodoKEMLThreeAES$                             &         109,922                &         110,695                   &             110,387       &               221,082             \\
$\FrodoKEMLFiveAES$                             &            207,752                &           209,724                   &                209,164       &               418,888             \\
\midrule
\multicolumn{5}{l}{\bf Plain C SHAKE} \\
\midrule
$\FrodoKEMLOneSHAKE$                               &            11,902                &           12,197                   &               12,219       &               24,416             \\
$\FrodoKEMLThreeSHAKE$                             &            26,334                &           26,618                   &               26,713       &               53,331             \\
$\FrodoKEMLFiveSHAKE$                             &            47,506                &            48,169                   &                48,505       &                96,674             \\
\bottomrule
\end{tabular}
\end{table}


\subsection{Comparison with other algorithms}\label{sec:comparison}

In this section, we compare the performance profile of \FrodoKEM with two other quantum-safe KEMs
that are also expected to be deployed and adopted for real-world applications:
CRYSTALS-Kyber~\cite{EuroSP:Kyber}, which was recently standardized by NIST as ML-KEM~\cite{MLKEM}, and
Classic McEliece~\cite{CME}, which is a Round 4 candidate in the NIST Post-Quantum Cryptography standardization project~\cite{NIST17}. 
As stated before, these two KEMs, alongside \FrodoKEM, are currently undergoing standardization by ISO.


\begin{table}[t]
\caption{Performance comparison of \FrodoKEM with two other algorithms: CRYSTALS-Kyber and Classic McEliece.
The results for \FrodoKEM were obtained on a 3.2GHz Intel Core i7-8700 (Coffee Lake) processor,
while the results for CRYSTALS-Kyber and Classic McEliece are taken from~\cite{SUPERCOP} (version {\tt supercop-20241022}), corresponding to a
3.0GHz Intel Core i3-8109U (Coffee Lake) processor.
For \FrodoKEM we present results for the variant using $\AESOneTwoEight$, and for Classic McEliece we use the fast (f) variants. 
Cycle counts are rounded to the nearest $10^3$ cycles.}\label{tab:comparison}
\scriptsize
\medskip
\centering
\renewcommand{\tabcolsep}{0.25cm}
\renewcommand{\arraystretch}{1.1}
\begin{tabular}{l|r r|r r r|r}
\toprule
\multirow{2}{*}{\textbf{Scheme}}     & \multicolumn{2}{c|}{\textbf{Sizes (in bytes)}} &     \multicolumn{3}{c|}{\textbf{Cycles ($\times 10^3$)}}                     &    \multirow{2}{*}{\textbf{Total}}  \\ 
\cmidrule(lr){2-3} \cmidrule(lr){4-6}
                                     & \textbf{public key} & \textbf{ciphertext}      &      \textbf{KeyGen}       &    \textbf{Encaps}     &    \textbf{Decaps}     &                                     \\
\midrule
\multicolumn{7}{l}{\bf NIST security level 1} \\
\midrule
Kyber512                             &       800  &    768          &             23                &               36                   &                    28         &                   87             \\
$\FrodoKEMLOneAES$                   &     9,616  &  9,752          &            957                &            1,134                   &                 1,059         &                3,150             \\
{\tt mceliece348864f}                &   261,120  &     96          &         30,381                &               30                   &                   118         &               30,529             \\
\midrule
\multicolumn{7}{l}{\bf NIST security level 3} \\
\midrule
Kyber678                             &     1,184  &  1,088          &               39              &               53                   &                    42         &                  134             \\
$\FrodoKEMLThreeAES$                 &    15,632  & 15,792          &            2,043              &            2,125                   &                 2,042         &                6,210             \\
{\tt mceliece460896f}                &   524,160  &    156          &           97,899              &               64                   &                   245         &               98,208             \\
\midrule
\multicolumn{7}{l}{\bf NIST security level 5} \\
\midrule
Kyber1024                            &     1,568  &  1,568          &               55              &               75                   &                    62         &                  192             \\
$\FrodoKEMLFiveAES$                  &    21,520  & 21,696          &            3,432              &            3,609                   &                 3,471         &               10,512             \\
{\tt mceliece6960119f}               & 1,047,319  &    194          &          195,984              &              120                   &                   287         &              196,391             \\
\bottomrule
\end{tabular}
\end{table}
   

\autoref{tab:comparison} illustrates that CRYSTALS-Kyber offers the best performance in terms of both speed and bandwidth. 
\FrodoKEM's sizes and runtimes are, across the board, more than an order of magnitude larger and slower, respectively, compared to Kyber's.
These results were key in motivating NIST's selection of CRYSTALS-Kyber as the preferred drop-in replacement for most applications.

However, for security-sensitive applications, it can be argued that Classic McEliece provides a more relevant comparison to \FrodoKEM.
In this case, the significantly larger public key sizes of Classic McEliece, which exceed \FrodoKEM's by more than an order of magnitude,
may render it impractical for many use cases.
Similarly, the high computational cost of the full Classic McEliece protocol presents additional challenges.
For example, the runtime of Classic McEliece at NIST level 5 is approximately 65.5~ms on a server-class processor (compare to \FrodoKEM's 3.29~ms). 
Nonetheless, Classic McEliece offers an advantage in certain static key exchange scenarios in which its substantial key generation cost
can be amortized over multiple key encapsulation executions.


%%% Local Variables:
%%% mode: latex
%%% TeX-master: "Main"
%%% End:
