\section{Auxiliary algorithms}%
\label{sec:algs}

This section describes the auxiliary algorithms used in $\FrodoPKE$ and $\FrodoKEM$.

%\lewis{Renaming this section to "Auxiliary algorithms" from "Algorithm description." Previously the beginning of this section read "This section specifies the algorithms comprising the $\FrodoKEM$ key encapsulation mechanism.  $\FrodoKEM$ is built from a public-key encryption scheme, $\FrodoPKE$, as well as several other components." }

\paragraph{Notation.}
In this work, the algorithms are described in terms of the following parameters:
\begin{itemize}
\item $\chi$, a probability distribution on $\bbZ$, and $T_\chi$, the corresponding distribution table for sampling;
\item $q=2^D$, a power-of-two integer modulus with exponent $D\leq 16$;
\item $n,\mbar,\nbar$, integer matrix dimensions with $n \equiv 0 \pmod 8$;
\item $B\leq D$, the number of bits encoded in each matrix entry;
\item $\ell=B\cdot \mbar\cdot\nbar$, the length of bit strings that are encoded as $\mbar$-by-$\nbar$ matrices;
\item $\lengthseedA$, the bit length of seeds used for pseudorandom matrix generation;
\item $\lengthseedSE$, the bit length of seeds used for pseudorandom bit generation for error sampling.
\end{itemize}

%Additional parameters for specific algorithms accompany the algorithm description.

\subsection{Matrix encoding of bit strings}
\label{sec:matrix-encoding}

This subsection describes how bit strings are encoded as mod-$q$ integer matrices.
Recall that $2^B \leq q$. The encoding function $\encode(\cdot)$ encodes an integer $0 \leq k < 2^B$ as an element in $\bbZ_q$ by multiplying it by $q/2^B=2^{D-B}$:
\[ \encode(k) := k\cdot  q/2^B \enspace . \]
This encoding function can be found in early works on LWE-based encryption, for example \cite{PKC:KawTanXag07,STOC:PeiWat08,C:PeiVaiWat08}.
Using this function, the function $\Frodo.\Encode$ encodes bit strings of
length $\ell=B\cdot \mbar\cdot\nbar$ as  $\mbar$-by-$\nbar$-matrices with
entries in $\bbZ_q$ by applying $\encode(\cdot)$ to $B$-bit sub-strings sequentially and filling the matrix row by row entry-wise. The function $\Frodo.\Encode$ is shown in \autoref{alg:encode}.
Each $B$-bit sub-string is interpreted as an integer $0 \leq k < 2^B$ and then encoded by $\encode(k)$, which means that $B$-bit values are placed into the $B$ most significant bits of the corresponding entry modulo $q$.  

The corresponding decoding function $\Frodo.\Decode$ is defined as shown in
\autoref{alg:decode}. It decodes the $\mbar$-by-$\nbar$ matrix $\bfK$ into a
bit string of length $\ell=B\cdot\mbar\cdot\nbar$. It extracts $B$ bits from
each entry by applying the function $\decode(\cdot)$:
\[ \decode(c) = \round{c \cdot 2^B/q} \bmod 2^B \enspace . \]
That is, the $\bbZ_{q}$-entry is interpreted as an integer, then
divided by $q/2^B$ and rounded. This amounts to rounding to the $B$
most significant bits of each entry. With these definitions, it is the case that $\decode(\encode(k)) = k$ for all $0\leq k < 2^B$.

\begin{figure}[h!]
\centering
\begin{minipage}[t]{0.48\textwidth}
\begin{algorithm}[H]
\caption{\label{alg:encode} $\Frodo.\Encode$}
% Ananth: I use phantom to make sure that the two Algorithm margins align in
% the final multicol
{\bf Input:} Bit string $\bfk\in \{0,1\}^\ell$, $\ell = B\cdot\mbar\cdot\nbar$.\phantom{$\bbZ_q^{\mbar\times\nbar}$}\\
{\bf Output:} Matrix $\bfK \in \bbZ_q^{\mbar \times \nbar}$.\\[-1.5ex]
\rule{\linewidth}{.5pt}
\vspace{-0.5cm}
\begin{algorithmic}[1]
    \FOR{($i = 0$; $i < \mbar$;  $i\gets i + 1$)}   
    \FOR{($j = 0$; $j < \nbar$;  $j\gets j+1$)}
    \STATE $k \gets \sum_{l=0}^{B-1}\bfk_{(i\cdot \nbar + j)B+l}\cdot 2^l$
    \STATE $\bfK_{i,j} \gets \encode(k)= k\cdot q/2^B$
    \ENDFOR
    \ENDFOR
    \RETURN$\bfK = (\bfK_{i,j})_{0\leq i < \mbar, 0 \leq j < \nbar}$
\end{algorithmic}
\end{algorithm}
\end{minipage}
~
\begin{minipage}[t]{0.49\textwidth}
\begin{algorithm}[H]
\caption{\label{alg:decode} $\Frodo.\Decode$}
{\bf Input:} Matrix $\bfK \in \bbZ_q^{\mbar\times\nbar}$.\\
{\bf Output:} Bit string $\bfk \in \{0,1\}^\ell$, $\ell = B\cdot\mbar\cdot\nbar$.\\[-1.5ex]
\rule{\linewidth}{.5pt}
\vspace{-0.5cm}
\begin{algorithmic}[1]
    \FOR{($i = 0$; $i < \mbar$;  $i\gets i+1$)}   
    \FOR{($j = 0$; $j < \nbar$;  $j\gets j+1$)}
    \STATE $k \gets \decode(\bfK_{i,j})= \round{\bfK_{i,j}\cdot 2^B/q} \bmod 2^B$
    \STATE $k = \sum_{l=0}^{B-1}k_l \cdot 2^l$ \text{ where } $k_l\in \{0,1\}$
    \FOR{($l = 0$; $l < B$; $l\gets l+1$)}
    \STATE $\bfk_{(i\cdot \nbar + j)B+l} \gets k_l$ 
    \ENDFOR
    \ENDFOR
    \ENDFOR
    \RETURN$\bfk$
\end{algorithmic}
\end{algorithm}
\vspace{-0.5em}
\textit{Note to implementers: recall $\round{x} = \lfloor x + \frac{1}{2} \rfloor$.}
\end{minipage}
\end{figure}

\subsection{Packing matrices modulo $q$}
\label{sec:pack}

This section specifies packing and unpacking algorithms to transform
matrices with entries in $\bbZ_q$ to bit strings and vice versa. The
algorithm $\Frodo.\Pack$ packs a matrix into a bit string by simply
concatenating the $D$-bit matrix coefficients, as shown in
\autoref{alg:frodopack}. Note that in the software implementation, the
resulting bit string is stored as a byte array, padding with zero bits
to make the length a multiple of~$8$. The reverse operation
$\Frodo.\Unpack$ is shown in \autoref{alg:frodounpack}.

\begin{figure}[h!]
\centering
\begin{minipage}[t]{0.45\textwidth}
\begin{algorithm}[H]
\caption{\label{alg:frodopack} $\Frodo.\Pack$}
{\bf Input:} Matrix $\bfC \in \bbZ_q^{n_1\times n_2}$.\\
{\bf Output:} Bit string $\bfb \in \{0,1\}^{D\cdot n_1\cdot n_2}$.\phantom{$\bbZ_q^{\mbar\times\nbar}$}\\[-1.5ex]
\rule{\linewidth}{.5pt}
\vspace{-0.5cm}
\begin{algorithmic}[1]
    \FOR{($i = 0$; $i < n_1$;  $i\gets i+1$)}   
    \FOR{($j = 0$; $j < n_2$;  $j\gets j+1$)}
    \STATE $\bfC_{i,j} = \sum_{l=0}^{D-1}c_l \cdot 2^l$ \text{ where } $c_l\in \{0,1\}$
    \FOR{($l = 0$; $l < D$; $l\gets l+1$)}
    \STATE $\bfb_{(i\cdot n_2  + j)D+l} \gets c_{D-1-l}$ 
    \ENDFOR
    \ENDFOR
    \ENDFOR
    %\FOR{($i = 0$; $i < D\cdot n_1\cdot n_2/8$;  $i\gets i+1$)}
    %\STATE $b_\bfC[i] \gets \sum_{l=0}^{7} \bfb_{8i + l} \cdot 2^{7-l}$
    %\ENDFOR
    \RETURN $\bfb$
\end{algorithmic}
\end{algorithm}
\end{minipage}
~
\begin{minipage}[t]{0.5\textwidth}
\begin{algorithm}[H]
\caption{\label{alg:frodounpack} $\Frodo.\Unpack$}
{\bf Input:}  Bit string $\bfb \in \{0,1\}^{D\cdot n_1\cdot n_2}$, $n_1$, $n_2$.\phantom{$\bbZ_q^{\mbar\times\nbar}$}\\
{\bf Output:} Matrix $\bfC \in \bbZ_q^{n_1\times n_2}$.\\[-1.5ex]
\rule{\linewidth}{.5pt}
\vspace{-0.5cm}
\begin{algorithmic}[1]
    %\FOR{($i = 0$; $i < D\cdot n_1\cdot n_2/8$;  $i\gets i+1$)}
    %\STATE $b_\bfC[i] = \sum_{l=0}^{7} c_l 2^l$, $c_l\in \{0,1\}$
    %\FOR{($l = 0$; $l < 8$; $l\gets l+1$)}
    %\STATE $\bfb_{8i+l} \gets c_{7-l}$
    %\ENDFOR
    %\ENDFOR
    \FOR{($i = 0$; $i < n_1$;  $i\gets i+1$)}   
    \FOR{($j = 0$; $j < n_2$;  $j\gets j+1$)}
    \STATE $\bfC_{i,j} \gets \sum_{l=0}^{D-1}\bfb_{(i\cdot n_2  + j)D+l}\cdot 2^{D-1-l}$
    \ENDFOR
    \ENDFOR
    \RETURN $\bfC$
\end{algorithmic}
\end{algorithm}
\end{minipage}
\end{figure}

\subsection{Deterministic random bit generation}\label{sec:rbg}

$\FrodoKEM$ requires the deterministic generation of random bit
sequences from a random seed value. This is done using the
SHA-3-derived extendable output function
$\SHAKE$~\cite{dworkin2015sha}. The function $\SHAKE$ is taken as
either $\SHAKE128$ or $\SHAKE256$ (indicated below for each
parameter set of $\FrodoKEM$), and takes as input a bit string $X$ and a
requested output bit length $L$.

\subsection{Sampling from the error distribution}\label{sec:sampling}

The error distribution~$\chi$ used in $\FrodoKEM$ is a discrete,
symmetric distribution on $\bbZ$, centered at zero and with small
support, which approximates a rounded continuous Gaussian
distribution.

The support of~$\chi$ is
$S_{\chi}=\set{-s, -s+1, \dots, -1, 0, 1, \dots, s-1, s}$ for a
positive integer $s$. The probabilities $\chi(z) = \chi(-z)$ for
$z \in S_{\chi}$ are given by a discrete probability density function,
which is described by a table
\[ T_{\chi} = (T_{\chi}(0), T_{\chi}(1), \dots, T_{\chi}(s)) \] of
$s+1$ positive integers related to the cumulative distribution function.  For a
certain positive integer $\lengthchi$, the table entries satisfy the
following conditions:
\[ T_{\chi}(0) = 2^{\lengthchi-1}\cdot \chi(0)-1 \quad\quad \textrm{and} \quad\quad
  T_{\chi}(z) = T_{\chi}(0) + 2^{\lengthchi}\sum_{i=1}^z\chi(i) \enspace \textrm{ for
  } 1 \leq z \leq s. \]

Since the distribution $\chi$ is symmetric and centered at zero, it is easy to verify that
$T_{\chi}(s) = 2^{\lengthchi-1}-1$.

Sampling from~$\chi$ via inversion sampling is done as shown in
\autoref{alg:samplechi}.  Given a string of~$\lengthchi$ uniformly
random bits~$\bfr \in \bit^{\lengthchi}$ and a distribution table
$T_{\chi}$, the algorithm $\Frodo.\Samp$ returns a sample~$e$ from the
distribution $\chi$. (Note that $T_\chi(s)$ is never accessed.)
We emphasize that it is important to perform
this sampling in constant time to avoid exposing timing side-channels,
which is why Step~\ref{alg:sample:loop} of the algorithm does a complete loop through the
entire table~$T_{\chi}$. The comparison in Step~\ref{alg:sample:cmp} needs
to be implemented in a constant-time manner.

\begin{algorithm}[H]
\caption{\label{alg:samplechi} $\Frodo.\Samp$}
{\bf Input:} A (random) bit string $\bfr = (\bfr_0, \bfr_1, \dots,
\bfr_{\lengthchi-1}) \in \bit^{\lengthchi}$, the table $T_{\chi} = (T_{\chi}(0), T_{\chi}(1), \dots, T_{\chi}(s))$.\\
{\bf Output:} A sample $e \in \bbZ$.\\[-1.5ex]
\rule{\linewidth}{.5pt}
\vspace{-0.5cm}
\begin{algorithmic}[1]
    \STATE $t \gets \sum_{i=1}^{\lengthchi-1} \bfr_i\cdot 2^{i-1}$
    \STATE $e\gets 0$
    \FOR{($z = 0$; $z < s$;  $z\gets z+1$)}\label{alg:sample:loop}
    \IF{$t > T_\chi(z)$}\label{alg:sample:cmp}
    \STATE $e\gets e+1$
    \ENDIF
    \ENDFOR
    \STATE $e\gets (-1)^{\bfr_0}\cdot e$
    \RETURN$e$
\end{algorithmic}
\end{algorithm}

An $n_1$-by-$n_2$ matrix of~$n_1n_2$ samples from the error
distribution is sampled on input of a 
$(n_1n_2 \cdot \lengthchi)$-bit string, here written as a sequence 
$(\bfr^{(0)}, \bfr^{(1)}, \dots, \bfr^{(n_1n_2-1)})$ of
$n_1n_2$ bit vectors of length
$\lengthchi$ each,
by sampling~$n_1n_2$ error terms through calls to $\Frodo.\Samp$ on a
corresponding $\lengthchi$-bit substring~$\bfr^{(i\cdot n_2 + j)}$ and
the distribution table $T_\chi$ to sample the matrix entry
$\bfE_{i,j}$. The algorithm $\Frodo.\SampV$ is shown in
\autoref{alg:samplevecchi}.

\begin{algorithm}[H]
\caption{\label{alg:samplevecchi} $\Frodo.\SampV$}
{\bf Input:} A (random) bit string $(\bfr^{(0)}, \bfr^{(1)}, \dots, \bfr^{(n_1n_2-1)}) \in \{0,1\}^{n_1n_2 \cdot \lengthchi}$ (here, each $\bfr^{(i)}$ is a vector of $\lengthchi$ bits), the
  table $T_{\chi}$.\\
  % = (T_{\chi}(0), T_{\chi}(1), \dots, T_{\chi}(s))$
{\bf Output:} A sample $\bfE \in \bbZ^{n_1\times n_2}$.\\[-1.5ex]
\rule{\linewidth}{.5pt}
\vspace{-0.5cm}
\begin{algorithmic}[1]
    \FOR{($i = 0$; $i < n_1$;  $i\gets i+1$)}
    \FOR{($j = 0$; $j < n_2$;  $j\gets j+1$)}
    \STATE $\bfE_{i,j} \gets \Frodo.\Samp(\bfr^{(i\cdot n_2 + j)}, T_\chi)$
    \ENDFOR
    \ENDFOR
    \RETURN$\bfE$
\end{algorithmic}
\end{algorithm}

\subsection{Pseudorandom matrix generation}
\label{sec:genA}

The algorithm $\Frodo.\gen$ takes as input a seed
$\seedA\in\{0,1\}^\lengthseedA$ and an implicit dimension $n\in\bbZ$, and outputs
a pseudorandom matrix $\bfA\in \bbZ_q^{n\times n}$. There are two
options for instantiating $\Frodo.\gen$. The first one uses
$\AESOneTwoEight$ and is shown in \autoref{alg:genA_AES}; the second
uses $\SHAKE128$ and is shown in \autoref{alg:genA_SHAKE}.

\paragraph{Using $\AESOneTwoEight$.}

\autoref{alg:genA_AES} generates a matrix
$\bfA \in \bbZ_{q}^{n\times n}$ as follows.  For each row
index $i=0,1,\ldots,n-1$ and column index $j=0,8,\ldots,n-8$, the
algorithm generates a 128-bit block, which it uses to set the matrix
entries $\bfA_{i,j}, \bfA_{i,j+1}, \ldots, \bfA_{i,j+7}$ as follows.  It
applies $\AESOneTwoEight$ with key $\seedA$ to the input block
$\inner{i} \| \inner{j} \| 0 \cdots 0 \in \bit^{128}$, where $i,j$ are
encoded as 16-bit strings, represented in little-endian byte order. It then splits the 128-bit AES output block
into eight 16-bit strings, which it interprets as nonnegative
integers~$c_{i,j+k}$ for $k=0,1,\ldots,7$ in little-endian byte order.  Finally, it sets
$\bfA_{i,j+k}= c_{i,j+k} \bmod q$ for all~$k$.

\paragraph{Using $\SHAKE128$.}

\autoref{alg:genA_SHAKE} generates a matrix
$\bfA \in \bbZ_{q}^{n \times n}$ as follows. For each row index
$i=0,1,\ldots,n-1$, it calls $\SHAKE128$ with a main input
of~$\seedA$, prefixed with a counter (represented as a 16-bit integer in little-endian byte order), to produce a
$16n$-bit output string. It splits this output into 16-bit integers (in little-endian byte order)
$c_{i,j}$ for $j=0,1,\ldots, n-1$, and sets
$\bfA_{i,j} = c_{i,j} \bmod q$ for all~$j$.

\begin{algorithm}[]
\caption{\label{alg:genA_AES} $\Frodo.\gen$ using $\AESOneTwoEight$}
{\bf Input:} Seed $\seedA \in \{0,1\}^\lengthseedA$.\\
{\bf Output:} Matrix $\bfA \in \bbZ_q^{n\times n}$.\\[-1.5ex]
\rule{\linewidth}{.5pt}
\vspace{-0.5cm}
\begin{algorithmic}[1]
    \FOR{($i = 0$; $i < n$;  $i\gets i+1$)}
    \FOR{($j = 0$; $j < n$;  $j\gets j+8$)}
    \STATE $\bfb \gets \inner{i} \| \inner{j} \|0 \cdots 0 \in
    \bit^{128}$ where $\inner{i}, \inner{j} \in \bit^{16}$
    \STATE $\inner{c_{i,j}} \| \inner{c_{i,j+1}} \| \cdots \| \inner{c_{i,j+7}} \gets
    \AESOneTwoEight_{\seedA}(\bfb)$ where each $\inner{c_{i,k}} \in \bit^{16}$
    \FOR{($k=0$; $k<8$; $k\gets k+1$)}
    \STATE $\bfA_{i,j+k} \gets c_{i,j+k} \bmod q$
    \ENDFOR
    \ENDFOR
    \ENDFOR
    \RETURN $\bfA$
\end{algorithmic}
\end{algorithm}

\begin{algorithm}[]
\caption{\label{alg:genA_SHAKE} $\Frodo.\gen$ using $\SHAKE128$}
{\bf Input:} Seed $\seedA \in \{0,1\}^\lengthseedA$.\\
{\bf Output:} Pseudorandom matrix $\bfA \in \bbZ_q^{n\times n}$.\\[-1.5ex]
\rule{\linewidth}{.5pt}
\vspace{-0.5cm}
\begin{algorithmic}[1]
    \FOR{($i = 0$; $i < n$;  $i\gets i+1$)}
    \STATE $\bfb \gets \inner{i} \| \seedA \in
    \bit^{16+\lengthseedA}$ where $\inner{i} \in \bit^{16}$
    \STATE $\inner{c_{i,0}} \| \inner{c_{i,1}} \| \cdots \|
    \inner{c_{i,n-1}} \gets \SHAKE128(\bfb, 16n)$
    where each $\inner{c_{i,j}} \in \bit^{16}$
    \FOR{($j = 0$; $j < n$;  $j\gets j+1$)}
    \STATE $\bfA_{i,j} \gets c_{i,j} \bmod q$
    \ENDFOR
    \ENDFOR
    \RETURN$\bfA$
\end{algorithmic}
\end{algorithm}


%\paragraph{Using other functions.}
%In principle, other functions could be used to pseudorandomly generate the matrix $\bfA$, such as a lightweight stream cipher for platforms without the hardware instructions that make fast $\AES$ and $\SHAKE$ implementations possible.  As NIST currently does not standardize such a primitive, and the call for proposals indicated that submissions should use NIST primitives, we do not currently propose such an alternate instantiation.








%%% Local Variables:
%%% mode: latex
%%% TeX-master: "Main"
%%% End:
