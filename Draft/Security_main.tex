\section{Justification of security strength}%
\label{sec:strength:justification}

The security of $\FrodoKEM$ is supported both by security reductions
and by analysis of the best known cryptanalytic attacks.
A full overview of the reductions supporting \FrodoKEM's security is given in \autoref{sec:Security_appendix},
while the analysis of the best known cryptanalytic attacks can be found in \autoref{sec:attack:cryptanalytic}.
A summary of the bit-security estimates based on these two methodologies is shown in \autoref{tab:security}.

In this section, we provide the main theorems supporting the security of \FrodoKEM
in the ROM and QROM. Refer to \autoref{sec:Security_appendix} for complete details.

%\subsection{Security results}
\subsection{\MINDCCA security in the random oracle model}
\label{sec:strength:cca-kem} \label{sec:renyi_loss}

The following theorem says that the transformation $\SFOnotperpprime$, which we use to construct \FrodoKEM from \FrodoPKE, generically yields an \MINDCCA-secure KEM (in the classical random oracle model) from an \MINDCPA-secure public-key encryption scheme, even if the KEM and PKE are parameterized by different distributions, provided that those distributions are sufficiently close in terms of \renyi divergence.
We present multi-challenge security bounds, parameterized by the number of challenge ciphertexts~$n$, and the number of users~$u$.
A detailed description of the proof steps is given in \autoref{sec:rom-mindcca}.

To specialize this result to obtain an ordinary (single-key, single-ciphertext) $\INDCCA$ security bound, one merely sets $|\saltlist| = n = u = 1$.
In this case, the final additive term in the security bound from~\eqref{eq:cca-to-cpa-renyi-bound} is zero, and the overall bound matches what was obtained in~\cite{NISTPQC-R3:FrodoKEM20}.

% \begin{remark}
% In addition, the term $n^2/|\MsgSp|2^{\lengthsalt}$ term represents the probability of a collision in the challenge ciphertexts, and can be removed when we only have 1 challenge. Although, for $n = 1$ this term is negligible.
    
% \end{remark}

% \cpnote{I don't see such a term, but I do see one that also has a $\abs{\MsgSp}$ factor in the denominator.
%   I wonder if this term actually has a numerator of $n(n-1)$, in which case we do not have to say that it can be removed (because $n-1=0$).
%   It is kind of cheating to say that it can be removed, because that's not a specialization of this theorem, so it requires a justification (which we haven't provided).}

\iffalse (Lewis and Chris:) removing this rushed remark but keeping the text for future reference

\begin{remark}
  The results below are not sufficient to establish \MINDCCA security for \FrodoKEM, as we do not establish the \MINDCPA security of \FrodoPKE.
  \cpnote{The previous statement is not true, and sells us somewhat short.
    We do (trivially) have an \MINDCPA bound for \FrodoPKE; it is just looser than the one for (single-challenge) \INDCPA.
    But it does establish \MINDCCA security for \FrodoKEM.}
  What they do establish is that \MINDCCA security for \FrodoKEM reduces \emph{tightly} to the \MINDCPA security of \FrodoPKE, which is not true for \eFrodoKEM.
  Notably, the following results prove that the multi-ciphertext attack described in \autoref{sec:intro:multi-challenge} to \emph{un-salted} \FrodoKEM is not possible against \emph{salted} \FrodoKEM.
  \cpnote{I don't think the previous sentence is true.
    What if the \MINDCPA insecurity of \FrodoPKE grows linearly with $n$; then according to the theorem, the \MINDCCA security of \FrodoKEM might also grow linearly with $n$, which puts us in the same situation as with the multi-ciphertext attack, right?}
  Furthermore, they show that any multi-challenge attack on \FrodoKEM would be the result of a multi-sample attack on LWE.
  \cpnote{I think the previous sentence is basically true, but it's vague; LWE is already a multi-sample problem.
    I think we're talking about an attack on LWE with a large number of samples \emph{and secrets} (shared across the samples), right?}
\end{remark}

\fi

\begin{theorem}[\MINDCPA PKE $\implies$ \MINDCCA KEM in classical ROM,
  with distribution switch]\label{thm:cca-kem-to-cpa-pke-rom-parameterized}
  \ \newline
  Let $\PKE_X = (\KeyGen,\Enc,\Dec)$ be a $\delta(u)$-correct public-key
  encryption scheme with message space $\MsgSp$ that is parameterized
  by a distribution~$X$, and let $s$ be an upper bound on the total
  number of samples drawn from~$X$ by $\KeyGen$ and $\Enc$ combined.
  Let $G_1$, $G_2$ and $F$ be independent random oracles, and let
  $\KEMnotperpprime_X=\SFOnotperpprime[\PKE_X,G_1,G_2,F]$ be the KEM
  obtained by applying the $\SFOnotperpprime$ transform from
  \autoref{def:qfo} to $\PKE_{X}$. Let $P,Q$ be any discrete
  distributions. There exists a classical algorithm (a reduction)
  $\Bdversary$ against the \INDCPA security of $\PKE_Q$, which uses as
  a ``black box'' subroutine any $\Adversary$ against the \INDCCA
  security of $\KEMnotperpprime_P$ that makes at most $\qro$ oracle
  queries, for which
  \cpnote{Changed the $n^{2}$ numerator to $n(n-1)$; check that this is correct in the context of the other paper and update if possible.}
  \begin{align}
        \label{eq:cca-to-cpa-renyi-bound}
        %\begin{split}  
        \Adv{\MINDCCA}{\KEMnotperpprime_P}(\Adversary) \leq &
        \bigg( \left(
        \frac{n\cdot (2\qro+1)}{|\MsgSp||\saltlist|}
        + \qro \cdot \delta(u)
        + 3 \cdot\Adv{\MINDCPA}{\PKE_Q}(\Bdversary) \right) \nonumber \\
        & \cdot \exp(s \cdot \RD_\alpha(P \| Q)) \bigg)^{1-1/\alpha}+\frac{\qro}{|\MsgSp|}+ \frac{n(n-1)}{|\MsgSp| \cdot 2^{\lengthsalt}} 
        %\end{split}
  \end{align}
  for any $\alpha > 1$, where the \renyi divergence $\RD_\alpha$ is defined in \autoref{def:renyi}.
  The total running time of $\Bdversary$ is about that of $\Adversary$ plus the time needed to simulate the random oracles.
\end{theorem}

We point out that when $P=Q$ are the same distribution, we have
$\exp(s \cdot \RD_{\alpha}(P \| Q)) = 1$ for any $\alpha > 1$ and
hence can take~$\alpha$ to be arbitrarily large, making the exponent
$1-1/\alpha$ approach~$1$ in the limit. This special case is a main
theorem from~\cite{TCC:HofHovKil17}, and it relates the \INDCCA
security of $\FrodoKEM$ to the \INDCPA security of $\FrodoPKE$ when
they use the same error distribution, e.g., ~$\chi_{\Frodo}$.

The proof of \autoref{thm:cca-kem-to-cpa-pke-rom-parameterized}
combines components from three separate works: the modular analysis of
the Fujisaki--Okamoto transform by Hofheinz, H{\"o}velmanns and Kiltz~\cite{TCC:HofHovKil17}, the work on tight multi-target security for key encapsulation in \cite{GlabushThesis}, and the
work of Langlois, Stehl{\' e} and Steinfeld relating the security of
search problems when one distribution is substituted by another via
analysis of the \renyi divergence~\cite{EC:LanSteSte14}.  More
specifically, the proof of the theorem proceeds in the following
steps:
\begin{enumerate}
\item We apply Theorem 3.2 by Hofheinz et al.~\cite{TCC:HofHovKil17}, which shows that their $\T$
  transform converts an \INDCPA-secure public-key encryption scheme
  $\PKE_{Q}$ into an \OWPCA-secure public-key encryption scheme with
  deterministic encryption (in the random oracle model). Likewise, Theorem 3.2 in~\cite{TCC:HofHovKil17} is adapted in~\cite{GlabushThesis} to the $\ST$ transform used by $\FrodoKEM$; see~\cite[Section 4.2]{GlabushThesis}.
\item Next, we apply distribution substitution for the \OWPCA security
  experiment (which represents a search problem), to switch from
  distribution~$Q$ to~$P$.
\item Finally, we apply Glabush's adaptation of Theorem 3.4 from~\cite{TCC:HofHovKil17}, which shows that their
  $\Unotperp$ transform converts an \MOWPCA-secure public-key
  encryption scheme into an \MINDCCA-secure KEM (in the random oracle
  model).
\end{enumerate}

Hofheinz et al.~\cite{TCC:HofHovKil17} denote the composition of the $\T$ and $\Unotperp$ transforms as
the $\FOnotperp$ transform.  As described in
\autoref{sec:cca-transform}, we use a variant of this
transform called $\SFOnotperpprime$, which differs from
$\FOnotperp$ as follows:
\begin{itemize}
\item The $\T$ transform is replaced with the $\ST$ transform.
\item $\SFOnotperpprime$ uses a single hash function (with longer
  output) to compute $r$ and $K$, whereas $\FOnotperp$ uses two
  separate functions, but these are equivalent when the hash functions
  are modeled as independent random oracles and have appropriate
  output lengths.
\item The~$\SFOnotperpprime$ computation of $r$ and $K$ also takes the
  hash $G_1(\pk)$ of the public key $\pk$ as input,
  whereas~$\FOnotperp$ does not; this change preserves the relevant
  theorems (with trivial changes to the proofs), and has the potential
  to provide stronger multi-target security.
\end{itemize}

If we apply all the changes above, excepting the replacement of $\T$ by $\ST$, this results in the $\FOnotperpprime$ transform~\cite{EuroSP:Kyber,NISTPQC-R3:FrodoKEM20}.
The $\FOnotperpprime$ transform is applied to \FrodoPKE to build \eFrodoKEM.


\subsection{\INDCCA security in the quantum random oracle model}
\label{sec:strength:cca-kem-qrom}

Jiang et al.~\cite{C:JZCWM18} show that the $\FOnotperp$ transform yields an \INDCCA-secure KEM from an \OWCPA-secure public-key encryption scheme, in the \emph{quantum} random oracle model.
In~\cite{GlabushThesis}, this result was extended as to show that the $\SFOnotperp$ transform generically yields an $\MINDCCA$-secure KEM from a $\MINDCPA$ secure PKE.
This result extends to the $\SFOnotperpprime$ transform.

\begin{theorem}
  Let $\PKE = (\KeyGen,\Enc,\Dec)$ be a $\delta(u)$-correct public-key
  encryption scheme with message space $\MsgSp$. For any quantum adversary $\Bdversary$ against the $\MINDCCA$ security of $\KEMnotperp := \SFOnotperpprime[\PKE, G, H, \lengthsalt]$, there exists a quantum adversary $\Adversary$ against the $\MINDCPA$ security of $\PKE$, with roughly the same running time, such that
  \begin{equation}\label{thm:cca-kem-to-cpa-pke-qrom}
    \begin{split}  
        \Adv{\MINDCCA}{\KEMnotperp}(\Adversary) \leq &\dfrac{u^2}{|\M|2^{\lengthsalt}} + 4(\qro + 1)\sqrt{\dfrac{u\cdot n}{|\M\|\saltlist|}} + 16(\qro + 1)^2\cdot \delta(u)\\ & +
        4(\qro+1)\sqrt{\dfrac{u}{\M}} + 2\sqrt{(\qro + 1)\Adv{\MINDCPA}{\PKE}(\Bdversary)}
    \end{split}
  \end{equation}
\end{theorem}

%%% Local Variables:
%%% mode: latex
%%% TeX-master: "Main"
%%% End:
