\section{$\FrodoPKE$: \INDCPA-secure public-key encryption}%
\label{sec:cpa-pke}

This section describes $\FrodoPKE$, a public-key encryption scheme
with fixed-length message space, targeting \INDCPA security, that will
be used as a building block for $\FrodoKEM$. $\FrodoPKE$ is based on
the public-key encryption scheme by Lindner and Peikert~\cite{RSA:LinPei11}.

%with the following adaptations and
%specializations:
%\begin{itemize}
%\item The matrix $\bfA$ is generated from a seed using the function $\Frodo.\gen$ specified in \autoref{sec:genA}.
%\item Several ($\mbar$) ciphertexts are generated at once.
%\item The same Gaussian-derived error distribution is used for both
%  key generation and encryption.
%\end{itemize}

The PKE scheme is given by three algorithms
$(\FrodoPKE.\KeyGen, \FrodoPKE.\Enc,$ $\FrodoPKE.\Dec)$, defined
respectively in \autoref{alg:PKE:KeyGen}, \autoref{alg:PKE:Enc}, and
\autoref{alg:PKE:Dec}.  
$\FrodoPKE$ is parameterized by the parameters defined in \autoref{sec:algs}.
Additional parameters include the bit length of messages $\lengthm = \ell$, the message space $\MsgSp = \{0,1\}^\lengthm$,
and the matrix-generation algorithm $\Frodo.\gen$ (either \autoref{alg:genA_AES} or \autoref{alg:genA_SHAKE}).
%
%$\FrodoPKE$ is parameterized by the following parameters:
%\begin{itemize}
%\item $q=2^D$, a power-of-two integer modulus with exponent $D \leq 16$;
%\item $n, \mbar, \nbar$, integer matrix dimensions with $n \equiv 0 \pmod 8$;
%\item $B \leq D$, the number of bits encoded in each matrix entry;
%\item $\ell=B\cdot \mbar\cdot\nbar$, the length of bit strings that are encoded as $\mbar$-by-$\nbar$ matrices;
%\item $\lengthm = \ell$, the bit length of messages;
%\item $\MsgSp = \{0,1\}^\lengthm$, the message space;
%\item $\lengthseedA$, the bit length of seeds used for pseudorandom matrix generation;
%\item $\lengthseedSE$, the bit length of seeds used for pseudorandom bit generation for error sampling;
%\item $\Frodo.\gen$, the matrix-generation algorithm, either \autoref{alg:genA_AES} or \autoref{alg:genA_SHAKE};
%\item $T_\chi$, the distribution table for sampling.
%\end{itemize}
%
In the notation of~\cite{RSA:LinPei11}, their~$n_1$ and~$n_2$ both
equal~$n$ here, and their dimension~$\ell$ is~$\nbar$ here.


\begin{algorithm}[t]
\caption{\label{alg:PKE:KeyGen} $\FrodoPKE.\KeyGen$.}
{\bf Input:} None.\\
{\bf Output:} Key pair $(\pk, \sk) \in (\{0,1\}^\lengthseedA \times \mathbb{Z}_q^{n \times \nbar})\times \mathbb{Z}_q^{\nbar \times n}$.\\[-1.5ex]
\rule{\linewidth}{.5pt}
\vspace{-0.5cm}
\begin{algorithmic}[1]
    \STATE Choose a uniformly random seed $\seedA \getsr U(\{0,1\}^\lengthseedA)$
    \STATE Generate the matrix $\bfA \in \bbZ_q^{n \times n}$ via $\bfA\gets\Frodo.\gen(\seedA)$
    \STATE Choose a uniformly random seed $\seedSE \getsr U(\{0,1\}^\lengthseedSE)$
    \STATE Generate pseudorandom bit string $ (\bfr^{(0)}, \bfr^{(1)}, \dots, \bfr^{(2n\nbar-1)}) \gets \SHAKE(\mathtt{0x5F} \| \seedSE,$ $2n\nbar \cdot \lengthchi)$
    \STATE Sample error matrix $\bfS^\text{T} \gets \Frodo.\SampV((\bfr^{(0)}, \bfr^{(1)}, \dots, \bfr^{(n\nbar-1)}), \nbar, n, T_\chi)$
    \STATE Sample error matrix $\bfE \gets \Frodo.\SampV((\bfr^{(n\nbar)}, \bfr^{(n\nbar+1)}, \dots, \bfr^{(2n\nbar-1)}), n, \nbar, T_\chi)$
    \STATE Compute $\bfB = \bfA\bfS + \bfE$
    \RETURN public key $\pk \gets (\seedA, \bfB)$ and secret key $\sk \gets \bfS^\text{T}$
\end{algorithmic}
\end{algorithm}

\begin{algorithm}[]
\caption{\label{alg:PKE:Enc} $\FrodoPKE.\Enc$.}
{\bf Input:} Message $\mu \in \calM$ and public key $\pk = (\seedA, \bfB) \in \{0,1\}^\lengthseedA \times \bbZ_q^{n \times \nbar}$.\\
{\bf Output:} Ciphertext $c = (\bfC_1, \bfC_2) \in \bbZ_q^{\mbar\times n}\times \bbZ_q^{\mbar\times\nbar}$.\\[-1.5ex]
\rule{\linewidth}{.5pt}
\vspace{-0.5cm}
\begin{algorithmic}[1]
    \STATE Generate $\bfA\gets \Frodo.\gen(\seedA)$
    \STATE Choose a uniformly random seed $\seedSE \getsr U(\{0,1\}^\lengthseedSE)$
     \STATE Generate pseudorandom bit string $ (\bfr^{(0)}, \bfr^{(1)}, \dots, \bfr^{(2\mbar n+\mbar\nbar-1)}) \gets \SHAKE(\mathtt{0x96} \|$ $\seedSE, (2\mbar n+\mbar\nbar) \cdot \lengthchi)$
    \STATE Sample error matrix $\bfS' \gets \Frodo.\SampV((\bfr^{(0)}, \bfr^{(1)}, \dots, \bfr^{(\mbar n-1)}), \mbar, n, T_\chi)$
    \STATE Sample error matrix $\bfE' \gets \Frodo.\SampV((\bfr^{(\mbar n)}, \bfr^{(\mbar n+1)}, \dots, \bfr^{(2\mbar n-1)}), \mbar, n,$ $T_\chi)$
    \STATE Sample error matrix $\bfE'' \gets \Frodo.\SampV((\bfr^{(2\mbar n)}, \bfr^{(2\mbar n+1)}, \dots, \bfr^{(2\mbar n+\mbar\nbar-1)}),$ $\mbar, \nbar, T_\chi)$
    \STATE Compute $\bfB' = \bfS'\bfA + \bfE'$ and $\bfV = \bfS'\bfB + \bfE''$
    \RETURN ciphertext $c \gets (\bfC_1, \bfC_2)=(\bfB', \bfV + \Frodo.\Encode(\mu))$
\end{algorithmic}
\end{algorithm}

\begin{algorithm}[t]
\caption{\label{alg:PKE:Dec} $\FrodoPKE.\Dec$.}
{\bf Input:} Ciphertext $c=(\bfC_1, \bfC_2)  \in \bbZ_q^{\mbar\times n}\times \bbZ_q^{\mbar\times\nbar}$ and secret key $\sk = \bfS^\text{T} \in \mathbb{Z}_q^{\nbar \times n}$.\\
{\bf Output:} Decrypted message $\mu' \in \calM$.\\[-1.5ex]
\rule{\linewidth}{.5pt}
\vspace{-0.5cm}
\begin{algorithmic}[1]
    \STATE Compute $\bfM = \bfC_2 - \bfC_1\bfS$
    \RETURN message $\mu' \gets \Frodo.\Decode(\bfM)$
\end{algorithmic}
\end{algorithm}
   

\subsection{Correctness of \INDCPA PKE}\label{sec:cpa-pke-correctness}

The next lemma states bounds on the size of errors that can be handled
by the decoding algorithm.

\begin{lemma}\label{lem:CorrectnessDec}
  Let $q = 2^D$, $B \leq D$. Then $\decode(\encode(k)+e) = k$ for any
  $k,e\in \bbZ$ such that $0\leq k <2^B$ and
  $-q/2^{B+1} \leq e < q/2^{B+1}$.
\end{lemma}

\begin{proof}
  This follows directly from the fact that
  $\decode(\encode(k)+e) = \lfloor k + e2^B/q \rceil \bmod 2^B$.
\end{proof}

\paragraph{Correctness of decryption:}
The decryption algorithm $\FrodoPKE.\Dec$ computes 
\begin{align*}
\bfM 
&= \bfC_2 - \bfC_1\bfS \\
&= \bfV + \Frodo.\Encode(\mu) - (\bfS'\bfA + \bfE')\bfS \\
&= \Frodo.\Encode(\mu) + \bfS'\bfB + \bfE'' - \bfS'\bfA\bfS - \bfE'\bfS \\
&= \Frodo.\Encode(\mu) + \bfS'\bfA\bfS + \bfS'\bfE +  \bfE'' - \bfS'\bfA\bfS - \bfE'\bfS \\
&= \Frodo.\Encode(\mu) + \bfS'\bfE + \bfE''- \bfE'\bfS \\
&= \Frodo.\Encode(\mu) + \bfE'''
\end{align*}
for some error matrix $\bfE''' = \bfS' \bfE + \bfE'' - \bfE'
\bfS$. Therefore, any $B$-bit substring of the message $\mu$
corresponding to an entry of $\bfM$ will be decrypted correctly if the
condition in \autoref{lem:CorrectnessDec} is satisfied for the
corresponding entry of $\bfE'''$.

\paragraph{Failure probability.}

Each entry in the matrix $\bfE'''$ is the sum of~$2n$ products of two
independent samples from $\chi$, and one more independent sample from
$\chi$. Denote the distribution of this sum by $\chi'$. In the case of
a power-of-$2$ modulus $q$, the probability of decryption failure for
any single symbol is therefore the sum
\[ p = \sum_{e \notin [-q/2^{B+1},q/2^{B+1})}\chi'(e) \enspace. \]
The probability of decryption failure for the entire message can then
be obtained using the union bound.

For the distributions~$\chi$ we use, which have rather small
support~$S_{\chi}$, the distribution $\chi'$ can be efficiently
computed exactly. The probability that a product of two independent
samples from~$\chi$ equals~$e$ (modulo~$q$) is simply
\[ \sum_{(a,b)\in S_\chi\times S_\chi\; :\; a b = e \bmod q}
  \chi(a)\cdot \chi(b) \enspace. \] Similarly, the probability that
the sum of two entries assumes a certain value is given by the
standard convolution sum.  \autoref{sec:params:sets} reports the
failure probability for each of the selected parameter sets.

\subsection{Transform from $\MINDCPA$ PKE to $\MINDCCA$ KEM}\label{sec:cca-transform}

The Fujisaki--Okamoto transform \cite{C:FujOka99} constructs an \INDCCATwo-secure public-key encryption scheme, in the classical random oracle model, from a one-way-secure public-key encryption scheme (assuming the distribution of ciphertexts for each plaintext is sufficiently ``well spread'').
Targhi and Unruh \cite{TCC:TarUnr16} gave a variant of the Fujisaki--Okamoto transform and showed its \INDCCATwo security against a quantum adversary in the quantum random oracle model under similar assumptions.
The results of both \cite{C:FujOka99} and \cite{TCC:TarUnr16} proceed under the assumption that the public-key encryption scheme has perfect correctness, which is often not the case for lattice-based schemes (including ours).
Hofheinz, H{\" o}velmanns, and Kiltz (HHK)~\cite{TCC:HofHovKil17} gave a variety of constructions, in a modular fashion, that in particular allow for a small probability of incorrect decryption.

\cpnote{We need to be clearer about where SFO was defined, and what we've modified.}
In this work, we propose a variant called Modified Salted Fujisaki--Okamoto with implicit rejection ($\SFOnotperpprime$) transform, which concatenates a uniformly random, public salt of bit length $\ell = \lengthsalt$. The salt value is made
public as part of the ciphertext output by encapsulation. The security of the resulting KEM against multi-challenge attacks in the ROM was proven in a Master's thesis by Glabush~\cite{GlabushThesis}. In that same work, a QROM bound is proposed, with an accompanying proof sketch. A full proof was given in private communication~\cite{Multi-challenge}. Without the inclusion of a salt, with $n$ challenge ciphertexts, and $N$ ROM queries, an adversary can break $\MINDCCA$ security with probability $\frac{nN}{|\M|}$. With the inclusion of the salt, this advantage shrinks to $\frac{nN}{|\M||\saltlist|}$, where $|\saltlist|$ is the number of distinct salt values sampled by the challenge oracle.

We apply the $\SFOnotperpprime$ (``Modified Salted FO with implicit
rejection'') transform, which constructs an $\MINDCCA$-secure key
encapsulation mechanism from an $\MINDCPA$ public-key encryption scheme
and three hash functions; following~\cite{EuroSP:Kyber}, we make the
following modifications (see \autoref{fig:kem-qfo} for notation),
denoting the resulting transform $\SFOnotperpprime$:\plnote{It's not clear where these modifications are applied to to get the $\SFOnotperpprime$ transform.}
\begin{itemize}
\item A single hash function (with longer output) is used to compute $\bfr$ and $\bfk$.
\item The computation of $\bfr$ and $\bfk$ also takes the public key $\pk$ as input.
% Lewis says this is not a change:
% \item The computation of the shared secret $\ssk$ also takes the encapsulation $c$ as input.
\end{itemize}

\begin{definition}[$\SFOnotperpprime$ transform]
  \label{def:qfo}
  Let $\PKE=(\KeyGen,\Enc,\Dec)$ be a public-key encryption scheme
  with message space $\MsgSp$ and ciphertext space $\CtxtSp$, where
  the randomness space of $\Enc$ is $\RandSp$.  Let
  $\lengthsalt, \lengths, \lengthK, \lengthpkhash, \lengthss$ be parameters.  Let
  $G_1\colon \{0,1\}^* \to \{0,1\}^\lengthpkhash$,
  $G_2\colon \{0,1\}^* \to \RandSp \times \{0,1\}^\lengthK$, and
  $F\colon \{0,1\}^* \to \{0,1\}^\lengthss$ be hash functions.  Define
  $\KEMnotperpprime = \SFOnotperpprime[\PKE,G_1,G_2,F]$ to be the key
  encapsulation mechanism as shown in \autoref{fig:kem-qfo}.
\end{definition}

As observed by Guo, Johansson, and Nilsson \cite{C:GuoJohNil20}, a timing side-channel enables key recovery if \cpnote{use label and ref here:} step 5 of $\KEMnotperpprime.\Decaps$ is not performed in constant time.

\begin{figure}[h]
\centering
\fbox{
\begin{minipage}[t]{0.4\textwidth}
\underline{$\KEMnotperpprime.\KeyGen()$:}
\vspace{-1em}
\begin{algorithmic}[1]
\STATE $(\pk, \sk) \getsr \PKE.\KeyGen()$
\STATE $\bfs \getsr \{0,1\}^\lengths$
\STATE $\pkh \gets G_1(\pk)$
\STATE $\sk' \gets (\sk, \bfs, \pk, \pkh)$
\RETURN $(\pk, \sk')$
\end{algorithmic}

\medskip

\underline{$\KEMnotperpprime.\Encaps(\pk)$:}
\vspace{-1em}
\begin{algorithmic}[1]
\STATE $\mu \getsr \MsgSp$, {\color{black} $\salt \getsr \{0,1\}^{\lengthsalt}$}
\STATE $(\bfr, \bfk) \gets G_2(G_1(\pk) \| \mu {\color{black} \| \salt})$
\STATE $c \gets \PKE.\Enc(\mu, \pk; \bfr)$
\STATE $\ssk \gets F(c {\color{black} \| \salt} \| \bfk)$
\RETURN $(c {\color{black} \| \salt}, \ssk)$
\end{algorithmic}
\end{minipage}
~
\begin{minipage}[t]{0.5\textwidth}
\underline{$\KEMnotperpprime.\Decaps(c {\color{black} \| \salt}, (\sk, \bfs, \pk, \pkh))$:}
%\vspace{-1em}
\begin{algorithmic}[1]
\STATE $\mu' \gets \PKE.\Dec(c, \sk)$
\STATE $(\bfr', \bfk') \gets G_2(\pkh \| \mu' {\color{black} \| \salt})$
\STATE $\ssk_0' \gets F(c {\color{black} \| \salt} \| \bfk')$
\STATE $\ssk_1' \gets F(c {\color{black} \| \salt} \| \bfs)$
\STATE (in constant time) $\ssk' \gets \ssk_0'$ if $c = \PKE.\Enc(\mu', \pk; \bfr')$ else $\ssk' \gets \ssk_1'$
\RETURN $\ssk'$
\end{algorithmic}
\end{minipage}
}
\caption{Construction of an \INDCCA-secure key encapsulation mechanism
  $\KEMnotperpprime=\SFOnotperpprime[\PKE,G_1,G_2,F]$ from a public-key
  encryption scheme $\PKE$ and hash functions $G_1$, $G_2$, and $F$.}
\label{fig:kem-qfo}
\end{figure}

\begin{remark}
If the use of $\salt$ is removed from the $\SFOnotperpprime$ transform in \autoref{fig:kem-qfo},
it defaults to $\FOnotperpprime$ based on the standard Fujisaki--Okamoto transform~\cite{EuroSP:Kyber,NISTPQC-R3:FrodoKEM20}.
$\FOnotperpprime$ is used to build the \eFrodoKEM variant, which restricts the number of ciphertexts generated per public key (see \autoref{sec:cca-kem} and \autoref{sec:strength:justification}).
\end{remark}

%%% Local Variables:
%%% mode: latex
%%% TeX-master: "Main"
%%% End:
